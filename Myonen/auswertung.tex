\section{Auswertung}
%zuerst !alle! errechneten werte entweder in ganzen s�tzen aufz�hlen, oder in tabellen (�bersichtlicher) dargestellen, sowie auf die verwendeten formeln verweisen (die referenzierung der formel kann in der �berschrift stehen)
%kurz erw�hnen (vor der tabelle), warum wir das ganze ausrechnen bzw. was wir dort ausrechnen
%danach histogramme und plots erstellen, wobei wenn m�glich funktionen durch die plots gelegt werden (zur not k�nnen auch splines benutzt werden, was aber angegeben werden muss)
%bei fits immer die funktion und das reduzierte chiquadrat mit angegeben, wobei auf verst�ndlichkeit beim entziffern der zehnerpotenzen geachtet werden muss z.b. f(x)=(wert+-fehler)\cdot10^{irgendeine zahl}\cdot x + (wert+-fehler)\cdot10^{irgendeine zahl}
%bei jedem fit erkl�ren, nach welchem zusammenhang gefittet wurde und warum!
%bei plots darauf achten, dass die achsenbeschriftung (auch die tics) die richtige gr��e haben und die legende im plot nicht die messwerte verdeckt
%kurz die aufgabenstellung abgehandeln
Die gemessenen Daten werden in von Abb. \ref{fig:Kanal_Eintr�ge_Fehler_Plot} dargestellt.
\begin{figure}[H]
\centering
\includegraphics[scale = 0.35]{Kanal_Eintraege_Fehler_Plot}
\caption{Plot der Eintr�ge gegen die Kan�le}
\label{fig:Kanal_Eintr�ge_Fehler_Plot}
\end{figure}
Die ersten 17 Kan�le sind dabei wegen starkem Rauschen unbrauchbar.
Die mittlere Lebensdauer soll nun mit der Maximum Likelihood Methode bestimmt werden. Daf�r wurde ein Programm in Python geschrieben, welches $\tau$ (= mittlere Lebensdauer) iterativ mittels Formel \ref{eqn:tau_rek} rekursiv bestimmt. Als Startwert wird der Mittelwert (Formel \ref{eqn:startwert}) verwendet. Der Python-Code kann im Anhang nachvollzogen werden.