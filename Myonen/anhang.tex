\newpage
\section{Anhang}
Python-Code zur Bestimmung der mittleren Lebensdauer von Myonen und dessen Fehler nach Formel \ref{eqn:tau_err}.
\begin{framed}
\begin{lstlisting}
import numpy as np
import pandas as pd

Kanalskip = 17
Kanal_Eintr�ge_Fehler = np.array(
    pd.read_csv('Kanal_Eintr�ge_Fehler.dat',
                skiprows= Kanalskip, sep = ' ')).T
A= [0.002106, 0.002136]
B= [0.126, 0]
mu = []
T = []
mu_err = []

print(Kanal_Eintr�ge_Fehler.T)
print(len(Kanal_Eintr�ge_Fehler[0]))
print(len(Kanal_Eintr�ge_Fehler[1]))

def arith_mittel(N_k,t_k):
    mu = 1/np.sum(N_k)*np.sum(N_k*t_k)
    return mu

def fehler_tau(N_k,t_k):
    mu = 1/np.sum(N_k)*np.sqrt(np.sum(N_k*t_k**2))
    return mu

for i in range (len(A)):
    mu.append(arith_mittel(
        Kanal_Eintr�ge_Fehler[1],
        (A[i]*(Kanal_Eintr�ge_Fehler[0]-Kanalskip)+B[i])*10**(-6)))
    mu_err.append(fehler_tau(
        Kanal_Eintr�ge_Fehler[1],
        (A[i]*(Kanal_Eintr�ge_Fehler[0]-Kanalskip)+B[i])*10**(-6)))
    T.append((A[i]*(
        Kanal_Eintr�ge_Fehler[0][-1]-Kanalskip)+B[i])*10**(-6))
print(mu, mu_err, T)

def tau_iter(T,mu_iter_1, mu,a=0):
    a += 1
    mu_iter_2 = mu + T*np.exp(-T/mu_iter_1)/(1-np.exp(-T/mu_iter_1))
    if abs(mu_iter_2-mu_iter_1) > 10**(-24) and a < 10**2:
        return tau_iter(T,mu_iter_2,mu,a)
    else:
        return mu_iter_2, a

for i in range (len(mu)):
    tau , anziter = tau_iter(T[i],mu[i],mu[i])
    print(anziter, tau)
    print(T[i]*np.exp(-T[i]/tau)/(1-np.exp(-T[i]/tau)))
\end{lstlisting}
\end{framed}