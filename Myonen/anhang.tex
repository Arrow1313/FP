\section{Anhang}
Der Python-Code zur Bestimmung der mittleren Lebensdauer von Myonen ist im folgenden dargestellt.
\begin{framed}
\begin{figure}[H]

\begin{lstlisting}
import numpy as np
import pandas as pd

cd /home/arrow13/Git/FP/Myonen/Messdate/

Kanalskip = 17
Kanal_Eintr�ge_Fehler = np.array(pd.read_csv(
'Kanal_Eintr�ge_Fehler.dat', skiprows= Kanalskip, sep = ' ')).T

print(Kanal_Eintr�ge_Fehler.T)
print(len(Kanal_Eintr�ge_Fehler[0]))
print(len(Kanal_Eintr�ge_Fehler[1]))

def arith_mittel(N_k,t_k):
    mu = 1/np.sum(N_k)*np.sum(N_k*t_k)
    return mu
    
mu = arith_mittel(Kanal_Eintr�ge_Fehler[1],
 (0.00211206*(Kanal_Eintr�ge_Fehler[0]-Kanalskip)+0.12)*10**(-6))
T = (0.0021206*(Kanal_Eintr�ge_Fehler[0][-1]-Kanalskip)+0.12)*10**(-6)

print(mu, T)

# Eigentliche Iteration:
def tau_iter(T,mu_iter_1,a=0):
    a += 1
    mu_iter_2 = mu + T*np.exp(-T/mu_iter_1)/(1-np.exp(-T/mu_iter_1))
    if abs(mu_iter_2-mu_iter_1) > 10**(-24) and a < 10**2:
        return tau_iter(T,mu_iter_2,a)
    else:
        return mu_iter_2, a

tau , anziter = tau_iter(T,mu)

print(anziter, tau)
print(T*np.exp(-T/tau)/(1-np.exp(-T/tau))) #Differenz des Letzten Schrittes
\end{lstlisting}
\end{figure}
\end{framed}