\section{Einleitung}
%einleitung zu dem experiment.
%auf die einstellungen, die vor dem versuch gemacht werden, eingehen, oder auf eine anleitung dazu verweisen.
%---------------------------------------------------------------------------------------------
%hinter der einleitung kann der allgemeine theoretische hintergrund in einer zus�tzlichen section erkl�rt werden
In diesem Versuch soll die mittlere Lebensdauer von Myonen der H�henstrahlung mithilfe von Plastik-Szintillationsz�hlern und Photovervielfachern kurzer Anstiegszeit bestimmt werden. Es werden also relativistische Myonen, welche aus Spallationsprozessen von kosmischer Strahlung mit Teilchen in der oberen Atmosph�re stammen, detektiert. Die relativ lange Lebendauer des Myons deutet dabei auf einen schwachen Zerfall hin. Ziel des Versuches ist es einen m�glichst geringen statistischer Fehler und ein kleinen systematischen Fehler zu erreichen. Mit den Szintillationsz�hlern ist es m�glich Zerfallszeiten bis zu \SI{10}{ns} zu bestimmen. Dabei ist die Eichung des Aufbaus eine entscheidende Vorausssetzung f�r eine erfolgreiche Messung.