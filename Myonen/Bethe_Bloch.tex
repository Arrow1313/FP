\subsection{Bethe-Bloch-Formel}
Die Bethe-Bloch-Formel beschreibt den Energieverlust pro Wegl�ngeneinheit schneller und\\ -- verglichen mit einem Elektron -- schwerer geladener Teilchen beim Durchgang durch Materie duch inelastische St��e mit den Elektronen, wodurch eine Anregung oder Ionisation des Materials bewirkt wird. Die genaue Formel wird in diesem Versuch nicht ben�tigt und kann ggf. in geeigneter Fachliteratur nachgeschlagen werden. Teilchen in der N�he des Minimums der Bethe-Bloch-Formel werden minimalionisierende Teilchen genannt, und ihr Energieverlust kann als n�herungsweise konstant angesehen werden. Myonen der H�henstrahlung k�nnen in diesem Versuch als minimalionisierende Teilchen betrachtet werden, da sie, bevor sie in den Eisen-Filter eintreten, eine Energie von ca. $\SI{1}{GeV}$ ($\sim$ $\SI{99,5}{\percent}$ der Lichtgeschwindigkeit) besitzen, welche in der Bethe-Bloch-Formel f�r Myonen etwas �ber dem Minimum liegt.