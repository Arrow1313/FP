\subsubsection{Maximum Likelihood Methode}
Um die mittlere Lebensdauer von Myonen zu bestimmen eignet sich die Maximum Likelihood Methode. Wie in Abschnitt 2.3 besprochen sind Zerfallszeiten mit Zeitunabh�ngiger Zerfallsrate exponentialverteilt, sodass sich die einparametrige Warscheinlichkeitsdichte
\begin{align}
P(t_i|\tau) = \frac{1}{\tau}\frac{e^{-\frac{t_i}{\tau}}}{e^{-\frac{T_1}{\tau}}-e^{-\frac{T_2}{\tau}}}
\end{align}
ergibt. Da die Messung des Zeitintervalls nach oben und unten beschr�nkt ist, wurde die Warschenlichkeitsdichte normiert, wobei $T_1$ die untere Schranke und $T_2$ die obere Schranke der Zeitmessung ist. Es wird davon ausgegangen, dass die einzelnen Zeitmessungen unabh�ngig voneinander sind, sodass sich f�r die gesamte Messung die n-dimensionale Warscheinlichkeitsdichte
\begin{align}
L = \prod_{i=1}^N P(t_i|\tau) = \prod_{i=1}^N \frac{1}{\tau}\frac{e^{-\frac{t_i}{\tau}}}{e^{-\frac{T_1}{\tau}}-e^{-\frac{T_2}{\tau}}}
\end{align}
ergibt. Es wird davon ausgegangen, dass die gemessenen Zerfallszeiten der wahrscheinlichsten Messung entsprechen, sodass die Messung einem Maximum der Warscheinlichkeitsdichtefunktion bez�glich $\tau$ entspricht. Um das Maximum von $L$ zu bestimmen betrachtet man die Funktion
\begin{align}
\ln{L} = \ln\left[\prod_{i=1}^N P(t_i|\tau)\right] = \sum_{i=1}^{N}\ln[P(t_i|\tau)] = -\left[\sum_{i=1}^{N}\ln(\tau) + \frac{t_i}{\tau} + \ln(e^{-\frac{T_1}{\tau}}-e^{-\frac{T_2}{\tau}})\right]
\end{align}
und bestimmt die Nullstelle der Ableitung nach $\tau$.
Schlie�lich erh�lt man die beste Approximation f�r die Lebensdauer $\tau$:
\begin{align}
\hat{\tau} = \frac{1}{N}\sum_{i=1}^{N} t_i - \frac{T_1e^{-\frac{T_1}{\tau}}-T_2e^{-\frac{T_2}{\tau}}}{e^{-\frac{T_1}{\tau}}-e^{-\frac{T_2}{\tau}}}
\end{align}
Da es nur M Kan�le und damit M m�gliche Zeiten $t_k$ gibt, kann die erste Summe umgeschrieben werden,
\begin{align}
\sum_{i=1}^{N} t_i = \sum_{k=1}^{M} N_k t_k \text{ wobei } N = \sum_{k=1}^{M} N_k
\end{align}
sodass der Fehler von $\hat{\tau}$ �ber den statistischen Fehler auf $N_k$, welcher $\sqrt{N_k}$ betr�gt, bestimmt werden kann.
Es ergibt sich also
\begin{align}
\hat{\tau} = \sum_{k=1}^{M} N_k t_k - \frac{T_1e^{-\frac{T_1}{\tau}}-T_2e^{-\frac{T_2}{\tau}}}{e^{-\frac{T_1}{\tau}}-e^{-\frac{T_2}{\tau}}}
\end{align}
mit einem Fehler von
\begin{align}
\Delta\hat{\tau} = \frac{1}{N}\sqrt{\sum_{k=1}^{M}N_kt_k^2}
\end{align}