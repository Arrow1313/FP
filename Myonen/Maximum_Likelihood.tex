\subsubsection{Maximum Likelihood Methode}
Um die mittlere Lebensdauer von Myonen zu bestimmen eignet sich die Maximum Likelihood Methode. Wie in Abschnitt 2.3 besprochen sind Zerfallszeiten mit Zeitunabh�ngiger Zerfallsrate exponentialverteilt, sodass sich die einparametrige Warscheinlichkeitsdichte
\begin{align}
P(t_i|\tau) = \frac{1}{\tau}\frac{e^{-\frac{t_i}{\tau}}}{e^{-\frac{T_1}{\tau}}-e^{-\frac{T_2}{\tau}}}
\end{align}
ergibt. Da die Messung des Zeitintervalls nach oben und unten beschr�nkt ist, wurde die Warschenlichkeitsdichte normiert, wobei $T_1$ die untere Schranke und $T_2$ die obere Schranke der Zeitmessung ist. Es wird davon ausgegangen, dass die einzelnen Zeitmessungen unabh�ngig voneinander sind, sodass sich f�r die gesamte Messung die n-dimensionale Warscheinlichkeitsdichte
\begin{align}
L = \prod_{i=1}^N P(t_i|\tau) = \prod_{i=1}^N \frac{1}{\tau}\frac{e^{-\frac{t_i}{\tau}}}{e^{-\frac{T_1}{\tau}}-e^{-\frac{T_2}{\tau}}}
\end{align}
ergibt. Es wird davon ausgegangen, dass die gemessenen Zerfallszeiten der wahrscheinlichsten Messung entsprechen, sodass die Messung einem Maximum der Warscheinlichkeitsdichtefunktion bez�glich $\tau$ entspricht. Um das Maximum von $L$ zu bestimmen betrachtet man die Funktion
\begin{align}
\ln{L} = \ln\left[\prod_{i=1}^N P(t_i|\tau)\right]
\end{align} 