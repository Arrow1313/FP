%deckblatt erstellen.

\begin{titlepage}

\begin{center}
% Oberer Teil der Titelseite:
\includegraphics[width=0.75\textwidth]{logo.png}\\[1cm]    	%Logo 

\textsc{\LARGE Bergische Universit\"at Wuppertal}\\[1.5cm]	%Institution

\textsc{\Large Fortgeschrittenen Praktikum}\\[0.5cm]				%Projekt


\newcommand{\HRule}{\rule{\linewidth}{0.5mm}}
\HRule \\[0.4cm]
{ \huge \bfseries Lebensdauer von Myonen der H�henstrahlung}\\[0.4cm]				%Titel

\HRule \\[1.5cm]

% Author und Tutor
\begin{minipage}{0.4\textwidth}
\begin{flushleft} \large
\emph{Verfasser:}\\
Henrik \textsc{J�rgens} \\
Frederik \textsc{Strothmann} \\
\end{flushleft}
\end{minipage}
\hfill
\begin{minipage}{0.4\textwidth}
\begin{flushright} \large
\emph{Tutoren:} \\
Max \textsc{Mustermann} \\
Max \textsc{Mustermann} \\
\end{flushright}
\end{minipage}

\vspace{1cm}

\begin{table}[H]
\centering
\begin{tabular}{|l|}
\hline \textbf{Abstract: } \\
	   Ziel dieses Versuches ist es die Myonen-Lebensdauer\\
	   mit einem statistischen Fehler von unter \SI{5}{\percent}\\
	   und m�glicht geringem systematischen Fehler zu bestimmen\\
\hline 
\end{tabular}
\end{table} 

\vspace{1cm}

\begin{table}[H]
\centering
\begin{tabular}{|c|c|c|}
\hline Dies & ist & ein \\ 
\hline Platz- & halter & f�r \\ 
\hline die & bewertungs & Tabelle \\ 
\hline 
\end{tabular} 
\end{table}

\vfill

% Unterer Teil der Seite/Datum
{\large \today}

\end{center}

\end{titlepage}