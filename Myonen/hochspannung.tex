\subsection{Hochspannung der Photomultiplier}
Ein exaktes einstellen der Photomulitplier ist ist essenziell f�r eine gute Messung. Die Spannungen von PM1, PM2 und PM4 sollen dabei einen Wert von 2100 V nicht �berschreiten. Der Arbeitspunkt von PM3 liegt im Bereich von 2600-2700 V. F�r die Bestimmung des optimalen Arbeitspunktes wird die Schwelle des Diskriminators auf einen m�glichst geringen Wert eingestellt. Es wird die die Z�hlrate in Abh�ngigkeit der Spannung untersucht und nach einem  Plateau im Bereich von 100 bis 1000 Counts/s gesucht, da sich der Photomulitplier dann am optimalen Arbeitspunkt befindet. Falls die Z�hlraten zu niedrig sind kann ein $^{60}Co$-Pr�parat verwendet werden, um die Z�hlrate zu erh�hen.

Die aufgenommenen Spannungskennlinien sind in Abbildung ?? bis ?? zu sehen.

%Auswertung der Spannungskennlinien, beschreibung der Plots

Die bestimmten Spannungen f�r die Photomulitplier sind in Tabelle \ref{tab:hochspannung} aufgetragen.

\begin{table}[H]
\centering
\caption{Verwendete Spannungen f�r die Photomuliplier}
\label{tab:hochspannung}
\begin{tabular}{|c|c|}
\hline Photomultiplier & Spannung[V] \\ \hline
\hline PM1 &  \\ 
\hline PM2 &  \\ 
\hline PM3 &  \\ 
\hline PM4 &  \\ 
\hline 
\end{tabular} 
\end{table}