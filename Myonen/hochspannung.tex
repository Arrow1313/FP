\subsection{Hochspannung der Photomultiplier}
Alle vier Photomuliplier werden mit einer Hochspannung versorgt, PM1, PM2 und PM4 sollen dabei den Maximalwert von 2100V nicht �berschreiten, da es sonst zu Besch�digungen kommen kann. Der Maximalwert f�r PM3 liegt bei 2700V. Die Hochspannung der einzelnen Photomuliplier soll so eingestellt werden, dass PM1, PM2 und PM4 m�glichstdie selben Z�hlraten liefern. PM3 wurde mit der maximal m�glichen Spannung von 2700V betrieben. Es wurde die Spannungen in Tabelle \ref{tab:hochspannung} verwendet.

\begin{table}[H]
\centering
\caption{Verwendete Spannungen f�r die Photomuliplier}
\label{tab:hochspannung}
\begin{tabular}{|c|c|}
\hline Photomultiplier & Spannung[V] \\ \hline
\hline PM1 &  \\ 
\hline PM2 &  \\ 
\hline PM3 &  \\ 
\hline PM4 &  \\ 
\hline 
\end{tabular} 
\end{table}