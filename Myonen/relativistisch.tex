\subsection{Relativistische Effekte}
Aus Tabelle \ref{tab:zerfallskan�le} liest man eine Myonen Lebensdauer von $\SI{2,1971}{\micro \second}$ ab. Klassisch h�tte das Myon mit ann�hernder Lichtgeschwindigkeit also eine Reichweite von ungef�hr \SI{600}{\meter}, sodass nur sehr wenige Myonen die Erdoberfl�che erreichen k�nnten. Allerdings spielen bei ca. \SI{99,5}{\percent} der Lichtgeschwindigkeit relativistische Effekte eine gro�e Rolle (vgl. \cite{wiki_zeitdila}). Durch die Zeitdilatation vergeht die Zeit des Myons aus der Sicht eines Beobachters im Laborsystem langsamer: 
\begin{align}
t' = \frac{t}{\sqrt{1-\frac{v^2}{c^2}}}
\end{align}
Dabei entspricht $t'$ der beobachteten Zeitspanne im Laborsystem, $t$ der vergangenen Zeitspanne im Ruhesystem und v der Geschwindigkeit des Myons im Laborsystem. Im System der Myonen ergibt sich dagegen aufgrund der hohen Relativgeschwindigkeit eine L�ngenkontraktion. Unabh�ngig von der Betrachtungsweise ergibt sich, dass die Warscheinlichkeit f�r das Auftreffen der Myonen auf der Erdoberfl�che deutlich steigt, sodass gen�gend Myonen die Erdoberfl�che erreichen. 