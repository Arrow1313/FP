\subsection{Szitillationsz�hler}
Szintillationsz�hler werden zum detektieren von Strahlung verwendet. Sie sind aus einem Materialien aufgebaut, welche von ionisierender Strahlung angeregt werden k�nnen und die absorbierte Energie in Form von Photonen abstrahlen. Sie m�ssen gegen �u�eren Lichteinfall gesch�tzt werden. Bestenfalls ist dabei die Intensit�t des abgestahlten Lichtes direkt proportional zur Energie der einfallenden Strahlung. Dabei ist zu beachten, dass jeder Szintillator eine gewisse Totzeit besitzt, sodass nur Ereignisse, deren Zeitdifferenz gr��er als die Totzeit des Szitillationsz�hlers ist, detektiert werden k�nnen. Die abgestrahlten Photonen werden meistens mit Photomultipliern verst�rkt, um die Lichtintensit�ten in messbare Stromst�rken, welche �ber eine Analog-Digital-Konverter digitalisiert und an den Computer weitergegeben werden k�nnen, umzusetzen.