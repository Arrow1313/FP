\subsection{Standardmodell}
Das Standartmodell der Teilchenphysik beinhaltet drei der vier Grundlegenden Wechselwirkungen (WW), die schwache WW, die elektromagnetische WW und die Starke WW.
Die Kr�fte wechselwirken �ber Vektorbosonen, welche einen ganzzahligen Spin haben.
In Tabelle \ref{tab:ww} ist eine �bersicht der drei Kr�fte zu sehen.

\begin{table}[H]
\caption{In der Tabelle sind die Grundlegenden WW (au�er der Gravitation) und ihre Eigenschaften aufgetragen (entnommen \cite{povh} Seite 274)}
\label{tab:ww}
\centering
\begin{tabular}{|c|c|c|c|c|}
\hline Wechselwirkung & koppelt an & Austauschteilchen & $\frac{m_0}{GeV}$ & J$^P$ \\ 
\hline stark & Farbe & 8 Gluonen (g) & 0 & 1$^-$ \\ 
\hline elektromagnetisch & elektrische Ladung & Photon ($\gamma$) & 0 & 1$^-$ \\ 
\hline schwach & schwache Ladung & W$^\pm$, Z$^0$ & $\approx 10^2$ & 1 \\ 
\hline 
\end{tabular} 
\end{table}
Neben den Bosonen gibt es noch zwei weitere Fundamentale Teilchenarten die Quarks und die Leptonen, welche die Grundbausteine der Materie darstellen.
Beide geh�ren zu den Fermionen, haben also einen halbzahligen Spin. Leptonen und Quarks werden mit aufsteigender Masse in drei Generationen aufgeteilt.
In Tabelle \ref{tab:fermi} sind Quarks und Leptonen mit ihren Eigenschaften dargestellt.

\begin{table}[H]
\caption{�bersicht der Grundlegenden Eigenschaften von Quarks und Leptonen}
\label{tab:fermi}
\centering
\begin{tabular}{|p{2cm}|p{2cm}|p{2cm}|p{1cm}|p{3.5cm}|p{1cm}|}
	\hline
	Fermionen & \hspace{0.25cm} Familie \newline 1 \hspace{0.3cm} 2 \hspace{0.3cm} 3                                               & elektrische Ladung    & Farbe  & schwacher Isospin \newline rechtsh. \hspace{0.5cm} linksh. & Spin \\ \hline
	Leptonen  & $\nu_e$ \hspace{0.17cm} $\nu_\mu$ \hspace{0.17cm} $\nu_\tau$ \newline e \hspace{0.3cm} $\mu$ \hspace{0.3cm} $\tau$ & \hspace{0.6cm} 0 \newline  \hspace*{0.6cm} -1      & ------ & 1/2 \hspace{1.5cm} --- \newline \hspace*{2.4cm} 0          & 1/2  \\ \hline
	Quarks    & u \hspace{0.3cm} c \hspace{0.3cm} t \newline d \hspace{0.3cm} s \hspace{0.3cm} b                                   & \hspace*{0.2cm} +2/3 \newline \hspace*{0.3cm} -1/3 & r,g,b  & 1/2 \hspace{1.64cm} 0 \newline \hspace*{2.4cm} 0           & 1/2  \\ \hline
\end{tabular} 
\end{table}

