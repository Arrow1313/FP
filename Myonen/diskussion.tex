\section{Diskussion}
%(immer) die gemessenen werte und die bestimmten werte �ber die messfehler mit literaturwerten oder untereinander vergleichen
%in welchem fehlerintervall des messwertes liegt der literaturwert oder der vergleichswert?
%wie ist der relative anteil des fehlers am messwert und damit die qualit�t unserer messung?
%in einem satz erkl�ren, wie gut unser fehler und damit unsere messung ist
%kurz erl�utern, wie systematische fehler unsere messung beeinflusst haben k�nnten
%(wichtig) zum schluss ansprechen, in wie weit die ergebnisse mit der theoretischen vorhersage �bereinstimmen
%--------------------------------------------------------------------------------------------
%falls tabellen mit den messwerten zu lang werden, kann die section mit den messwerten auch hinter der diskussion angef�gt bzw. eine section mit dem anhang eingef�gt werden.
Die gemessenen Werte und die Kalibration lieferten zum Teil ungew�hnliche Ergebnisse, welche im Folgenden diskutiert werden.\\
Die Arbeitspunkte der Photomultiplier konnten ohne gr��ere Probleme aus den Plots abgesch�tzt werden.\\
Die Diskriminatorschwellwerte konnten auch ohne Probleme bestimmt werden, wobei das Plateau zwischen den von $^{60}$Co emittierten Photonen erst durch die Aufnahme weiterer Messwerte bei hoher Diskriminatorschwelle identifiziert werden konnte.\\
F�r die Einstellung des Delays wurden die Counts der Szintillatoren in Abh�ngigkeit des Delays geplottet, wobei die Szintillatoren so geschaltet waren, dass nur bei gleichzeitigem Signal hochgez�hlt wurde. Aus dem ersten Plot f�r Szintillator 1 und 3 konnte leider kein Plateau abgelesen werden, was unter anderem durch die Eisen-Absorber erschwert wird. Im zweiten Plot f�r Szintillator 1 und 2 konnte dagegen ein Plateau identifiziert werden. Totzdem wurde das Plateau nicht nur f�r den ersten, sondern auch f�r den zweiten Szintillator zus�tzlich mit dem Oszilloskop bestimmt. Leider konnten nicht die Signale beider Szintillatoren gleichzeitig getriggert werden, sodass keine guten Oszilloskopbilder f�r den 1. Szintillator entstanden sind. Beim zweiten Szintillator konnte mit etwas Gl�ck ein gutes Oszilloskopbild aufgenommen werden, welches zeigt, dass die Signale der Szintillatoren �bereinander liegen. Auff�llig dabei war bei beiden Oszilloskopmessungen, dass zus�tzliche verschobene Signale zu sehen waren, welche eigentlich von den Abschlusswiderst�nden h�tten unterdr�ckt werden sollen.\\
Die Kanal-Zeit-Eichung konnte erfolgreich Durchgef�hrt werden und die linearen Fits passen gut an die Daten. Allerdings konnte die Impulsbreite der verwendeten Dualclock nur bis ca. 12 $\mu$s hochgedreht werden. Der Versuch h�here Impulsbreiten einzustellen scheiterte an zu starken Schwankungen. Die Kanal-Zeit-Eichung wurde deshalb nur bis ca. 12 $\mu$s durchgef�hrt wie man in Abb. \ref{fig:kanal_zeit_fit} und \ref{fig:kanal_zeit_fit2} sieht.\\
Die Iteration der Maximum-Likelihood Methode wurde erfolgreich mit den Messdaten mithilfe von selbstgeschriebenem Python-Code durchgef�hrt. Wie in der Versuchsanleitung beschrieben, wurden die ersten 17 verrauschten Kan�le nicht zur Bestimmung der mittleren Lebensdauer verwendet. F�r $\hat{\tau}$ ergeben sich so Werte von 2.98(15) $\mu$s mit Offset bei der Kanal-Zeit-Eichung und 2.89(15) $\mu$s ohne Offset, welche vom Literaturwert von $\tau_{lit} = \SI{2,1969811(22)}{$\mu$s}$ um 35 \% bzw. 31\% abweichen. Die Werte f�r die mittlere Lebensdauer von Myonen konnten also nur sehr schlecht bestimmt werden. Begr�ndet werden kann dies vermutlich durch die Probleme mit den Dualclocks bei der Kanal-Zeit-Eichung, bei der eine der beiden Uhren nach Angaben des Tutor schon l�nger unbrauchbar f�r diese ist. Zus�tzliche Systematische Fehler k�nnen daneben bei der Bestimmung der Delays entstanden sein, da wie gesagt auch verschobene Signale zu sehen waren, die nicht erkl�rt werden konnten.