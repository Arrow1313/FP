\section{Diskussion}
%(immer) die gemessenen werte und die bestimmten werte �ber die messfehler mit literaturwerten oder untereinander vergleichen
%in welchem fehlerintervall des messwertes liegt der literaturwert oder der vergleichswert?
%wie ist der relative anteil des fehlers am messwert und damit die qualit�t unserer messung?
%in einem satz erkl�ren, wie gut unser fehler und damit unsere messung ist
%kurz erl�utern, wie systematische fehler unsere messung beeinflusst haben k�nnten
%(wichtig) zum schluss ansprechen, in wie weit die ergebnisse mit der theoretischen vorhersage �bereinstimmen
%--------------------------------------------------------------------------------------------
%falls tabellen mit den messwerten zu lang werden, kann die section mit den messwerten auch hinter der diskussion angef�gt bzw. eine section mit dem anhang eingef�gt werden.
Die gemessenen Werte und die Kalibration lieferten zum Teil ungew�hnliche Ergebnisse, welche im Folgenden diskutiert werden. Die Arbeitspunkte der Photomultiplier konnten ohne gr��ere Probleme aus den Plots abgesch�tzt werden.