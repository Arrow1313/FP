\subsection{Delay}
Da PM1, PM2 und PM4 sind �ber eine logische Einheit verbunden, welche den TAC startet. Deshalb muss sichergestellt werde, dass die Signale der drei Photomuliplier Zeitgleich ankommen. Die Zeitversetzung (Delay) der Signale wir �ber die Kabell�nge der Photomuliplier zur logischen Einheit eingestellt.

Es kann angenommen werde, das die Signale von PM3 und PM4 zur selben Zeit ankommen, deshalb kann das Signal von PM3 gut als Referenz f�r die ersten beiden verwendet werden. F�r die Bestimmung des Delays werden PM1 (bzw. PM2) und PM3 an die logische Einheit angeschlossen, jedoch ohne ein Veto. Es wir erwartet, das sich ein Plateau ausbildet, das Delay in der Mitte des Plateaus ist die optimale Einstellung. Damit sich ein m�glichst kleines Plateau ausbildet, m�ssen die Pulse der Diskriminatoren minimal sein. In Abbildung \ref{fig:delay_1} ist zu sehen, dass f�r den ersten Photomulipier kein eindeutiges Plateau identifiziert werden konnte. Deshalb wurde das Delay mit dem Oszilloskop bestimmt, daf�r wurden die beide Signale beobachtet und das Delay so eingestellt, das sich die Signale �berlappen. F�r den ersten Photomuliplier wurde so ein Delay von 9ns bestimmt. In Abbildung \ref{fig:delay_2} ist sind die Messdaten f�r den zweiten Photomuliplier zu sehen. Im Bereich von 20 bis 28ns ist ein Plateau zu erkennen, das Delay wurde mit dem Wert (24ns) in der Mitte des Plateaus angenommen. Das Delay von 24ns f�r den zweiten Photomuliplier wurde zus�tzlich noch mit dem Oszilloskop bestimmt und konnte so noch einmal best�tigt werden.

\begin{figure}[H]
	\centering
  \includegraphics[scale=0.33]{delay_1.pdf}
	\caption{Counts in Abh�ngigkeit vom Delay f�r die ersten Photomuliplier, es ist kein eindeutiges Plateau zu erkennen. Deshalb wurde das Delay mit eine Oszilloskop vorgenommen und ein Delay von 9ns bestimmt.}
	\label{fig:delay_1}
\end{figure}


\begin{figure}[H]
	\centering
  \includegraphics[scale=0.33]{delay_2.pdf}
		\caption{Counts in Abh�ngigkeit vom Delay f�r die ersten Photomuliplier, es ist im Bereich von 20 bis 28 ns zu erkennen. Das Delay wurde mit 24 ns, in der Mitte des Plateaus angenommen, dieser Wert wurde mit den Oszilloskop zus�tzlich verifiziert.}
	\label{fig:delay_2}
\end{figure}

Es ergeben sich die Delays in Tabelle \ref{tab:delay}.

\begin{table}[H]
	\centering
	\caption{Optimal bestimmtes Delay f�r PM1 und PM2}
	\label{tab:delay}
	\begin{tabular}{|c|c|}
	\hline Photomuliplier & Delay [ns] \\ \hline
	\hline PM1 & 9 \\ 
	\hline PM2 & 24 \\ 
	\hline 
	\end{tabular} 
\end{table}

