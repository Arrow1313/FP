\section{Energieverlust von $\alpha$-Strahlung in Luft}
%kurz das ziel dieses versuchsteiles ansprechen, damit keine zwei �berschriften direkt �bereinander stehen!
%bei schwierigeren versuchen kann auch der theoretische hintergrund erl�utert werden. (mit formeln, herleitungen und erkl�rungen)
\label{sec:energieverlust}
In diesem Versuchsabschnitt soll der Energieverlust von $\alpha$-Strahlung in Luft, bei Normaldruck, untersucht werden.


\subsection{Kanal-Energie-Eichung}
Dann wird eine Kanal-Zeit-Eichung mit der Zerfallsreihe von $^{266}$Ra durchgef�hrt. Die Zerfallsreihe ist in Abb. \ref{fig:zerfall} zu sehen. Die Peaks werden �ber einen Multi-Gauss-Fit bestimmt. Der 5,49 MeV und der 5,30 MeV Peak liegen so nah bei einander, dass er von dem Detektor nicht aufgel�st werden kann, deshalb, werde bei dem als ein Peak interpretiert. Dieser besteht aus einer Summe von 4 Gaussverteilungen. Den so bestimmten Kan�len kann mit der Zerfallskette (Abb. \ref{fig:zerfall}) eine Energie zugeordnet werden. Die Werte werden mit einer linearen Funktion gefittet (Gleichung \ref{eqn:lin}).


\begin{figure}[H]
	\centering
  \includegraphics[scale=0.33]{zerfalls_reihe.png}
	\caption{Zerfallsreihe von $^{266}$Ra. Entnommen aus \cite{anleitung}}
	\label{fig:zerfall}
\end{figure}


\begin{align}
\label{eqn:lin}
E(k) = A \cdot k + B
\end{align}

Da nur Fehler auf den Kanal vorhanden sind, wird Gleichung \ref{eqn:lin} zu Gleichung \ref{eqn:lin_neu} umgeformt und damit die Fehler im Fit zu ber�cksichtigen.

\begin{align}
\label{eqn:lin_neu}
k = \frac{E-B}{A}
\end{align}

Der Multi-Gauss-Fit ist in Abbilung \ref{fig:multi_fit} zu sehen. Dabei ergaben sich die Fitparameter in Tabelle \ref{tab:multi-fit}. Die Kurve passt optisch gut zu den Messdaten, was durch ein reduziertes Chiquadrat $\chi_{red}^2$ von 1,717 best�tigt wird.

\begin{figure}[H]
	\centering
  \includegraphics[scale=0.33]{multi_fit.pdf}
	\caption{Messung des $^{226}$Ra-Zerfalls mit Multi-Gauss-Fit}
	\label{fig:multi_fit}
\end{figure}


\begin{table}[H]
\centering
\caption{$\mu$-Parameter des Multi-Gauss-Fits}
\label{tab:multi-fit}
\begin{tabular}{|c|c|c|}
\hline Gausskurve & Wert & Fehler \\ \hline
\hline 1  & 255,0 & 1,0 \\ 
\hline 2  & 315,5 & 0,6 \\ 
\hline 3  & 372,5 & 0,8 \\ 
\hline 4  & 548,6 & 0,3 \\ 
\hline 
\end{tabular} 
\end{table}

Mit den bestimmten Kan�len und den Energien ergibt sich der Plot in Abb. \ref{fig:linear_fit}. Die Ergebnisse des Fits sind in Tabelle \ref{tab:linear-fit} aufgetragen. Das reduzierte Chiquadrat $\chi_{red}^2$ hat einen Wert von 3.01. Die Energie eines Kanals ist durch Gleichung \ref{eqn:energie-kanal} gegeben.

\begin{figure}[H]
	\centering
  \includegraphics[scale=0.33]{linear_fit.pdf}
	\caption{Die bestimmten Kan�le gegen Zerfallsenergien mit linearem Fit}
	\label{fig:linear_fit}
\end{figure}


\begin{table}[H]
\centering
\caption{Parameter des Linearen-Fits}
\label{tab:linear-fit}
\begin{tabular}{|c|c|c|}
\hline Parameter & Wert & Fehler \\ 
\hline A [MeV] & 0,00949 & 0,00007 \\ 
\hline B [MeV]& 2,48 & 0,03 \\ 
\hline 
\end{tabular} 
\end{table}

\begin{align}
\label{eqn:energie-kanal}
E(k) = (0,00949 \pm 0,00007) \cdot k + (2,48 \pm 0,03)
\end{align}


\subsection{Druckmessung}
Die Kammer wird langsam bel�ftet und Spektren im Bereich von 100 Torr bis 800 Torr in 25 Torr Schritten aufgenommen, wobei der Druck w�hrend der Messung konstant gehalten wird. Ein Bar entspricht 750 Torr. Nach Gleichung \ref{eqn:reich_normal} kann die Strecke mit erh�htem Druck in die Strecke unter Normaldruck umgerechnet werden. Aus den Countrates in Abh�ngigkeit des Drucks kann der absolute Energieverlust und der Energieverlust pro Wegst�ck bei Normaldruck bestimmt werden. F�r den Energieverlust pro Wegst�ck wird ein Verhalten nach Gleichung \ref{eqn:reich_normal} erwartet. Die Peaks der Spektren werden wie zuvor mit einem Mulit-Gauss gefittet. Eine Messung wurde �ber einen Zeitraum von 180s durchgef�hrt.

\subsection{Energieverlust}
Der Energieverlust pro Strecke wird nach Gleichung \ref{eqn:verlust} berechnet.

\begin{align}
\label{eqn:verlust}
\frac{dE}{dx} = \frac{E_1 - E_2}{x_1 - x_2}
\end{align}

Dabei ergibt sich der Fehler nach Gleichung \ref{eqn:delta_verlust}.

\begin{align}
\label{eqn:delta_verlust}
\Delta \frac{dE}{dx} = \sqrt{\left( \frac{\Delta dE}{dx} \right)^2 + \left( \frac{\Delta dx \cdot dE}{dx^2} \right)^2 }
\end{align}

Die Wegdifferenz wurde mit Gleichung \ref{eqn:reich_normal} bestimmt, dabei ist P$_{normal}$ = 760 Torr und \mbox{x$_{normal}$ = 6 cm}. Die Energie wurde mit Gleichung \ref{eqn:energie-kanal} bestimmt. Die Energie wurde �ber den Mittelwert Gleichung \ref{eqn:e_mittel} bestimmt.  Der Fehler ist in Gleichung \ref{eqn:delta_e_mittel} angegeben.

\begin{align}
\label{eqn:e_mittel}
\bar{E} = \frac{E_1 + E_2}{2}
\end{align}

\begin{align}
\label{eqn:delta_e_mittel}
\Delta \bar{E} = \sqrt{ \left( \frac{\Delta E_1}{2} \right)^2 + \left( \frac{\Delta E_2}{2} \right)^2}
\end{align}

Die bestimmten Peakpositonen in Abh�ngigkeit vom Druck sind in Tabelle \ref{tab:multi-fit-erg} aufgetragen. Tr�gt man nun $\frac{dE}{dx}$ gegen $dE$ auf, erwartet man das von der Bethe-Bloch-Formel beschriebene Verhalten. Die Bethe-Bloch-Formel ist in Gleichung \ref{eqn:bethe-bloch} zu sehen. Da nur der Verlauf der Daten von intresse ist und nicht die Werte der einzelnen Parameter, k�nnen die Messdaten mit einem vereinfachten Modell, Gleichung \ref{eqn:bethe-bloch-einfach} gefittet werden. Dabei wird ein linearer Zusammenhang zwischen der Energie und dem Geschwindigkeitsquadrat der Teilchen angenommen ($E_{kin} \sim v^2$).

\begin{align}
\label{eqn:bethe-bloch-einfach}
\frac{dE}{dx} = \frac{A}{\beta^2} \left[ ln \left( B \cdot \beta^2 \right) - \beta^2 \right]
\end{align}

In Abb. \ref{fig:bethe_1} ist $\frac{dE}{dx}$ gegen $dE$ f�r den 4,871 MeV Peak aufgetragen, die Werte des Fits sind in Tabelle \ref{tab:bethe_fit_1} zu sehen. 

\begin{table}[H]
\centering
\caption{Fitwerte f�r den 4,871 MeV Peak nach Gleichung \ref{eqn:bethe-bloch-einfach}}
\label{tab:bethe_fit_1}
\begin{tabular}{|c|c|}
\hline Parameter & Wert \\ 
\hline A & 1,8 $\pm$ 0,2 \\ 
\hline B & 3,8 $\pm$ 0,5 \\ 
\hline $\chi_{red}^2$ & 8,4 \\ 
\hline 
\end{tabular} 
\end{table}

\begin{figure}[H]
	\centering
  \includegraphics[scale=0.33]{bethebloch_1.pdf}
	\caption{Es ist $\frac{dE}{dx}$ gegen $dE$ aufgetragen, es wird ein Verlauf nach der Bethe-Bloch-Formel (Gleichung \ref{eqn:bethe-bloch}) erwartet. Aus dem Fit mit Gleichung \ref{eqn:bethe-bloch-einfach} ergibt sich ein reduziertes Chiquadrat $\chi_{red}^2$ von 8,4.}
	\label{fig:bethe_1}
\end{figure}


In Abb. \ref{fig:bethe_2} ist $\frac{dE}{dx}$ gegen $dE$ f�r den 5,49 MeV Peak aufgetragen, die Werte des Fits sind in Tabelle \ref{tab:bethe_fit_2} zu sehen. 

\begin{table}[H]
\centering
\caption{Fitwerte f�r den 5,49 MeV Peak nach Gleichung \ref{eqn:bethe-bloch-einfach}}
\label{tab:bethe_fit_2}
\begin{tabular}{|c|c|}
\hline Parameter & Wert \\ 
\hline A & 2,1 $\pm$ 0,1 \\ 
\hline B & 3,2 $\pm$ 0,2 \\ 
\hline $\chi_{red}^2$ & 13,4 \\ 
\hline 
\end{tabular} 
\end{table}

\begin{figure}[H]
	\centering
  \includegraphics[scale=0.33]{bethebloch_2.pdf}
	\caption{Es ist $\frac{dE}{dx}$ gegen $dE$ f�r den 5,49 MeV Peak aufgetragen, es wird ein Verlauf nach der Bethe-Bloch-Formel (Gleichung \ref{eqn:bethe-bloch}) erwartet. Aus dem Fit mit Gleichung \ref{eqn:bethe-bloch-einfach} ergibt sich ein $\chi_{red}^2$ von 13,4.}
	\label{fig:bethe_2}
\end{figure}




In Abb. \ref{fig:bethe_3} ist $\frac{dE}{dx}$ gegen $dE$ f�r den 6 MeV Peak aufgetragen, die Werte des Fits sind in Tabelle \ref{tab:bethe_fit_3} zu sehen. 

\begin{table}[H]
\centering
\caption{Fitwerte f�r den 6 MeV Peak nach Gleichung \ref{eqn:bethe-bloch-einfach}}
\label{tab:bethe_fit_3}
\begin{tabular}{|c|c|}
\hline Parameter & Wert \\ 
\hline A & 2,0 $\pm$ 0,1 \\ 
\hline B & 3,5 $\pm$ 0,2 \\ 
\hline $\chi_{red}^2$ & 17,5 \\ 
\hline 
\end{tabular} 
\end{table}

\begin{figure}[H]
	\centering
  \includegraphics[scale=0.33]{bethebloch_3.pdf}
	\caption{Es ist $\frac{dE}{dx}$ gegen $dE$, f�r den 6 MeV Peak aufgetragen, es wird ein Verlauf nach der Bethe-Bloch-Formel (Gleichung \ref{eqn:bethe-bloch}) erwartet. Aus dem Fit mit Gleichung \ref{eqn:bethe-bloch-einfach} ergibt sich ein $\chi_{red}^2$ von 17,5.}
	\label{fig:bethe_3}
\end{figure}



In Abb. \ref{fig:bethe_4} ist $\frac{dE}{dx}$ gegen $dE$ f�r den 7,69 MeV Peak aufgetragen. Die Werte des Fits sind in Tabelle \ref{tab:bethe_fit_4} zu sehen. 

\begin{table}[H]
\centering
\caption{Fitwerte f�r den 7,69 MeV Peak nach Gleichung \ref{eqn:bethe-bloch-einfach}}
\label{tab:bethe_fit_4}
\begin{tabular}{|c|c|}
\hline Parameter & Wert \\ 
\hline A & 2,3 $\pm$ 0,2 \\ 
\hline B & 3,2 $\pm$ 0,3 \\ 
\hline $\chi_{red}^2$ & 30,9 \\ 
\hline 
\end{tabular} 
\end{table}

\begin{figure}[H]
	\centering
  \includegraphics[scale=0.33]{bethebloch_4.pdf}
	\caption{Es ist $\frac{dE}{dx}$ gegen $dE$ f�r den 7,69 MeV Peak aufgetragen, es wird ein Verlauf nach der Bethe-Bloch-Formel (Gleichung \ref{eqn:bethe-bloch}) erwartet. Aus dem Fit mit Gleichung \ref{eqn:bethe-bloch-einfach} ergibt sich ein reduziertes Chiquadrat $\chi_{red}^2$ von 30,9.}
	\label{fig:bethe_4}
\end{figure}



Das $\chi_{red}^2$ liegt bei allen Fits weit oberhalb von 1, dabei f�llt auf, dass die ersten Werte besser an die Behte-Bloch-Kurve passen. Ab einem bestimmten Energiewert steigt $\frac{dE}{dx}$ bei allen Peaks. Der Energiewert, ab dem $\frac{dE}{dx}$ steigt, ist bei jedem Peak anders, weshalb ein systematischer Fehler ab einer bestimmten Messung ausgeschlossen werden kann. Die Quelle des Fehlers ist Unbekannt. Wie sich in Abschnitt \ref{subsec:bragg} zeigt, ist der Verlauf der Bragg-Kurven bei allen Messungen wie erwartet.

\subsection{Bragg-Kurve}
\label{subsec:bragg}
Tr�gt man $\frac{dE}{dx}$ gegen die zur�ckgelgete Strecke auf, so erwartet man ein Verhalten wie in Abb. \ref{fig:braggkurve}.

In Abb. \ref{fig:bragg_1} ist f�r den 4,87 MeV Peak $\frac{dE}{dx}$ gegen $x$ zusehen.

\begin{figure}[H]
	\centering
  \includegraphics[scale=0.33]{bragg_kurve_1.pdf}
	\caption{Es wurde $\frac{dE}{dx}$ gegen $x$ aufgetragen. Der Anfang des Braggpeaks ist deutlich zu sehen}
	\label{fig:bragg_1}
\end{figure}

\noindent
In Abb. \ref{fig:bragg_2} ist f�r den 5,49 MeV Peak $\frac{dE}{dx}$ gegen $x$ aufgetragen.

\begin{figure}[H]
	\centering
  \includegraphics[scale=0.33]{bragg_kurve_2.pdf}
	\caption{$\frac{dE}{dx}$ gegen $x$ Aufgetragen. Der Braggpeak ist deutlich zu sehen}
	\label{fig:bragg_2}
\end{figure}

\noindent
In Abb. \ref{fig:bragg_3} ist f�r den 6 MeV Peak $\frac{dE}{dx}$ gegen $x$ aufgetragen.

\begin{figure}[H]
	\centering
  \includegraphics[scale=0.33]{bragg_kurve_3.pdf}
	\caption{Es wurde $\frac{dE}{dx}$ gegen $x$ aufgetragen. Der Braggpeak ist gut zu sehen}
	\label{fig:bragg_3}
\end{figure}


\noindent
In Abb. \ref{fig:bragg_4} ist f�r den 7,69 MeV Peak $\frac{dE}{dx}$ gegen $x$ zusehen.

\begin{figure}[H]
	\centering
  \includegraphics[scale=0.33]{bragg_kurve_4.pdf}
	\caption{$\frac{dE}{dx}$ wurde gegen $x$ aufgetragen. Der Anfang des Braggpeaks ist deutlich zu sehen, wobei die Herkunft des Ausrei�ers bei ca. 2,5 cm nicht klar ist.}
	\label{fig:bragg_4}
\end{figure}

Bei allen Plots ist die Form der Braggkurven zu erkennen. Vergleicht man Abb. \ref{fig:braggkurve} mit Abb. \ref{fig:bragg_2}, so f�llt auf, dass der erwartete Peak bei 3,7 cm liegt, der Peak in Abb. \ref{fig:bragg_2} jedoch bei ca. 1.8 cm liegt. Woher die Schiebung kommt ist nicht bekannt.