\section{Rutherford-Streuversuch}
%kurz das ziel dieses versuchsteiles ansprechen, damit keine zwei �berschriften direkt �bereinander stehen!
%bei schwierigeren versuchen kann auch der theoretische hintergrund erl�utert werden. (mit formeln, herleitungen und erkl�rungen)
In diesem Versuchsabschnitt soll die Streuung von $\alpha$-Strahlung an Goldfolie untersucht werden.

\subsection{Versuchsdurchf�hrung}
Die Streuung von $\alpha$-Teilchen an einer Goldfolie wird in einem Winkelbereich von -30$^\circ$ bis 30$^\circ$ in 5$^\circ$-Schritten untersucht. Neben der Goldfolie wird der Kollimator mit einer Spaltbreite von 1mm eingesetzt. Die Streukammer wird auf 35 mbar evakuiert. Die Messdaten werden mit dem Computer aufgenommen. F�r jeden Winkel wurde f�r einen Zeitraum von 3min gemessen. Die aufgenommen Histogramme werden mit der Poissonverteilung gefittet (Beispiel im Anhang). Der Fehler des Winkels wurde mit 2$^\circ$ angenommen. Die so Bestimmten Z�hlraten werden mit Gl. \ref{eqn:ruth} gefittet. Da das Winkelma� einen Offset besitzt, wird dies durch eine additive Konstante als Fitparameter ber�cksichtigt.

\begin{align}
\label{eqn:ruth}
f(x) = \frac{A}{sin^4 \left[ \frac{\pi (x-B)}{2\cdot 180} \right]}
\end{align}

\subsection{Auswertung}
In Abb. \ref{fig:rutherford_gold} sind die Messdaten mit dem Fit der Rutherfordstreuformel zu sehen. Die Rutherfordstreuuformel wurde nach GL. \ref{eqn:ruth} gefittet. Dabei ergaben sich f�r den Fit die Werte in Tabelle \ref{tab:fit_gold_ruth}.

\begin{figure}[H]
	\centering
  \includegraphics[scale=0.33]{rutherford_messung.pdf}
	\caption{Es sind die Messdaten aus der Streuung von $\alpha$-Strahlung an Gold zu sehen. Die Messdaten wurden mit  Gl. \ref{eqn:ruth} gefittet, dabei ergab sich ein $\chi_{red}^2$ von 2,48.}
	\label{fig:rutherford_gold}
\end{figure}

\begin{table}[H]
\centering
\caption{Fitparamter f�r die Goldfolie nach Gl. \ref{eqn:ruth}}
\label{tab:fit_gold_ruth}
\begin{tabular}{|c|c|}
\hline Paramter & Wert \\ 
\hline A & 0,00037 $\pm$ 0,00002 \\ 
\hline B & 2,62 $\pm$ 0,05 \\ 
\hline $\chi_{red}^2$ & 2,48 \\ 
\hline 
\end{tabular} 
\end{table}


Der Fit passt optisch gut zu den Daten, auch wenn das $\chi_{red}^2$ nur einen Wert von 2,48 hat.