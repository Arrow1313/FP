\section{Rutherford-Streuversuch}
%kurz das ziel dieses versuchsteiles ansprechen, damit keine zwei �berschriften direkt �bereinander stehen!
%bei schwierigeren versuchen kann auch der theoretische hintergrund erl�utert werden. (mit formeln, herleitungen und erkl�rungen)
In diesem Versuchsabschnitt soll die Streuung von $\alpha$-Strahlung an Goldfolie untersucht werden.

\subsection{Versuchsdurchf�hrung}
Es soll die Streuung von $\alpha$-Teilchen an Goldfolie in einem Winkelbereich von -30$^\circ$ bis 30$^\circ$, in 5$^\circ$-Schritten untersucht werden. Neben der Goldfolie wird noch der Kollimator mit einer Spaltbreite von 1mm eingesetzt. Die Messdaten werden mit dem Computer aufgenommen. F�r jeden Winkel wurde f�r einen Zeitraum von ??s gemessen.

\subsection{Auswertung}
In Abb. ?? sind die Messdaten mit dem Fit der Rutherfordstreuformel zu sehen. Die Rutherfordstreuuformel wurde nach GL. ?? gefittet. Dabei ergaben sich f�r den Fit die Werte in Tabelle ??.
