\section{Reichweitenbestimmung}
%kurz das ziel dieses versuchsteiles ansprechen, damit keine zwei �berschriften direkt �bereinander stehen!
%bei schwierigeren versuchen kann auch der theoretische hintergrund erl�utert werden. (mit formeln, herleitungen und erkl�rungen)
Es soll die Reichweite von $\alpha$-Strahlung bei Normaldruck untersucht werden.

\subsection{Versuchsdurchf�hrung}
Es werden keine Metallfolien oder Kollimationsspalte verwendet. Der Schwenkarm wird auf 0$^\circ$ eingestellt. Da der Abstand zwischen Dem $^{241}$Am-Pr�parat und der Quelle nicht ver�nderbar ist, kann die Abstandsabh�ngigkeit nicht direkt bestimmt werden. Stattdessen wird die Z�hlrate in Abh�ngigkeit des Luftdrucks aufgenommen. Da die Reichweite linear mit Anzahl der St��e mit den Luftmolek�len abh�ngt, h�ngt die Reichweite unter Annahme des idealen Gassesetztes auch linear von Druck ab. Daraus l�sst sich f�r die Reichweite unter Normaldruck Gl. \ref{eqn:reich_normal} folgern.

\begin{align}
\label{eqn:reich_normal}
x_{Normal} = x_{Messung} \frac{p_{Messung}}{p_{Normal}} 
\end{align}

Der Druck wird solange in Schritten von ?? erh�ht, bis die Countrate auf 0 abf�llt. Mit dem Zusammenhang, aus Gl. \ref{eqn:reich_normal} kann die Reichweite von $\alpha$-Strahlung in Luft bestimmt werden.

\subsection{Auswertung}
