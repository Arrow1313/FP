\subsection{Wirkunsquerschnitt}
Der Wirkungsquerschnitt ist eine wichtige Gr��e um Streuprozesse zu analysieren und zu verstehen. Er gibt die Reaktionswarscheinlichkeit normiert auf die Anzahl der Targetteilchen pro Fl�cheneinheit an,
\begin{align}
\sigma = \frac{w}{\frac{N_T}{F}}
\end{align}
wobei
\begin{align}
w = \frac{N_{Reaktion}}{N_{Gesamt}} = \frac{I_{gestreut}}{I_{einfallend}}
\end{align}
die Reaktionswarscheinlichkeit, also der Anteil der gestreuten/wechselwirkenden Teilchen an der Gesamtteilchenzahl bzw. der Anteil des Stromes der einfallenden Teilchen am Strom der gestreuten Teilchen, ist. (vgl. \cite{WQtot})\\
Da im Versuchsaufbau nur ein gewisser Raumwinkel $\Delta\Omega$ vermessen wird, ist es sinnvoll mit dem differentiellen Wirkungsquerschnitt $\frac{d\sigma}{d\Omega}$ zu arbeiten, welcher die Warscheinlichkeit angibt, das gestreute Teilchen in einem kleinen Winkelbereich d$\Omega$ zu finden.
