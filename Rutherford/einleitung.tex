\section{Einleitung}
%einleitung zu dem experiment.
%auf die einstellungen, die vor dem versuch gemacht werden, eingehen, oder auf eine anleitung dazu verweisen.
%---------------------------------------------------------------------------------------------
%hinter der einleitung kann der allgemeine theoretische hintergrund in einer zus�tzlichen section erkl�rt werden
In diesem Versuch soll die Streuung von $\alpha$-Teilchen mit Materie untersucht werden. Die Rutherfordsche Streuformel soll experimentell verifiziert und der Energieverlust von $\alpha$-Teilchen in Materie untersucht werden. Dieser kann durch die Bethe-Formel und die Bragg-Kurve beschrieben werden. F�r die Erzeugung von $\alpha$-Strahlung ($^4_2$He-Kerne) werden verschiedene radioaktive Pr�parate verwendet, welche an unterschiedlichen Targets gestreut werden. F�r die �berpr�fung der Rutherfordschen Streuformel wird zuerst eine d�nne Goldfolie bestrahlt und in einem kleinen Winkelbereich um \SI{0}{$^\circ$} gemessen, sowie die R�ckstreuung der $\alpha$-Teilchen qualitativ ausgewertet. Anschlie�end soll die Kernladungszahl von Aluminium bestimmt werden, indem der Versuch mit einem Aluminiumtarget (Alu-Folie) wiederholt und mit den Daten zum Goldtarget verglichen wird. Um die Reichweite von $\alpha$-Strahlung in Materie

