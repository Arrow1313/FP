\subsection{Rutherford-Streuung}
Die Streuung von geladenen Teilchen am Coulomb-Potential wird mit der Rutherfordschen Streuformel, welche den differentiellen Wirkungsquerschnitt f�r diese Art von Streuung charakterisiert. Die genaue Herleitung kann man z.B. dem Buch \cite{Landau} entnehmen. Die Streuformel beschreibt die Wahrscheinlichkeit, ein gestreutes Teilchen im Raumwinkelelement d$\Omega$ zu finden.
\begin{align}
\frac{d\sigma}{d\Omega} = \left(\frac{Z_1Z_2e^2}{16\pi\varepsilon_0E_0}\right)^2\frac{1}{\sin^4(\frac{\theta}{2})}
\label{eqn:rutherstreu}
\end{align}
\begin{table}[H]
\begin{tabular}{c|l}
$Z_i$  & Kernladung des Targets/Projektils \\ 
$e$  & Elementarladung \\
$\varepsilon_0$  & Elektrische Feldkonstante \\
$E_0$  & Energie des Teilchens vor der Wechselwirkung \\ 
$\theta$ & Streuwinkel \\
\end{tabular} 
\end{table}
Die angegebene Formel hat bei $\theta = 0$ einen Pol 4. Ordnung, sodass f�r die Berechung des totalen Wirkungsquerschnittes ein minimaler Streuwinkel $\theta_{min}$ angenommen werden muss. In diesem Versuch soll Formel \ref{eqn:rutherstreu} �berpr�ft werden.