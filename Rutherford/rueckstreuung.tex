\section{R�ckstreuung von $\alpha$-Teilchen}
%kurz das ziel dieses versuchsteiles ansprechen, damit keine zwei �berschriften direkt �bereinander stehen!
%bei schwierigeren versuchen kann auch der theoretische hintergrund erl�utert werden. (mit formeln, herleitungen und erkl�rungen)
Es soll qualitativ die R�ckstreuung von $\alpha$-Teilchen untersucht werden. Mit dem zuvor bestimmten Fit kann der Wert f�r einen Winkel von 150$^\circ$ bestimmt werden.

\subsection{Versuchsdurchf�hrung}
Die Goldfolie wird ohne Spalt in die Kammer eingesetzt und ein Winkel von 150$^\circ$ eingestellt. Da eine sehr geringe Z�hlrate erwartet wird, wird �ber einen Zeitraum von einer Stunde gemessen.

\subsection{Auswertung}

�ber den Zeitraum von 60 Minuten wurden 5 Counts gemessen, dies entspriche einer Rate von 0.0014 Counts/s. Die niedrige Z�hlrate zeigt, das die Rutherfordstreuuformel f�r gro�e Winkel nicht mehr zutrifft.