\section{Bestimmung der Kernladungszahl von Aluminium}
%kurz das ziel dieses versuchsteiles ansprechen, damit keine zwei �berschriften direkt �bereinander stehen!
%bei schwierigeren versuchen kann auch der theoretische hintergrund erl�utert werden. (mit formeln, herleitungen und erkl�rungen)
Es soll die Kernladungszahl von Aluminium bestimmt werden. Die Winkelverteilung der Z�hlraten soll mit denen der Goldfolie verglichen werden.

\subsection{Versuchsdurchf�hrung}
Die Aluminiumfolie und der 1mm Spalt werden eingesetzt. Dann werden die Z�hlraten f�r verschiedene Winkel �ber einen Zeitraum von ??s aufgenommen.

\subsection{Auswertung}
Die Kernladungszahl wird mit zwei verschiedenen Methoden bestimmt. In der ersten Methode wird die Rutherfordstreuuformel (Gl. ??) an die Z�hlraten gefittet. Dabei entspricht der Parameter Z$_2$ der Kernladungszahl von Aluminium. Die Messdaten mit dem Fit sind in Abb. ?? zu sehen. F�r den Fit ergaben sich die Werte in Tabelle ??.


F�r die zweite Methode wird Gl. ?? verwendet, dabei werden f�r die festen Parameter die Werte in Tabelle ?? verwendet.

\begin{table}[H]
\centering
\caption{Werte der festen Parameter f�r die Bestimmung der Kernladungszahl von Aluminium nach Gleichung ??}
\label{tab:alu_paras}
\begin{tabular}{|c|c|}
\hline Parameter & Wert \\ 
\hline $Z^2_{Au}$ & 79 \\ 
\hline $d_{Au}$ & 2 [$\mu$m] \\ 
\hline $d_{Al}$ & 7 [$\mu$m] \\ 
\hline \.{N}$_{Au}$ &  \\ 
\hline \.{N}$_{Al}$ &  \\ 
\hline 
\end{tabular} 
\end{table}