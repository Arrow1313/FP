\subsection{$\alpha$-Strahlung}
$\alpha$-Strahlung entsteht durch den spontanen Zerfall eines schweren Kernes, wobei ein $^4_2$He-Kern emittiert wird.
Bei diesem Prozess wird die Coulomb-Barriere des Kerns durchtunnelt, sodass die potentielle Energie des Kernes in kinetische Energie umgewandelt wird. Die Spallationswarscheinlichkeit ist bei schweren Kernen gr��er, da die Bindungsenergie pro Nukleon bei diesen abnimmt. Die in diesem Versuch verwendeten Pr�parate sind Americium und Radium, deren Eigenschaften und Zerfallsprodukte in Tabelle \ref{tab:Zerfall} dargestellt sind :
\begin{table}[H]
\centering
\caption{Eigenschaften der $\alpha$-Strahler (vgl. \cite{Americ} und \cite{Radi})}
\label{tab:Zerfall}
\begin{tabular}{c|c|c|c}
 Isotop & $\tau$/a & E/MeV & Zerfall \\ 
\hline  $^{241}_{95}$Am& 432,2 & 5,486 & $^{237}_{93}$Np \\ 
  $^{226}_{88}$Ra& 1602 & 4,871 & $^{222}_{86}$Rn \\ 
\end{tabular}
\end{table}
Americium kann neben dem $\alpha$-Zerfall auch zu einem verschwindend geringen Teil durch spontane Spaltung zerfallen.