\section{Fazit}
%im fazit nochmal alles zusammenfassen und den verlauf der messung absch�tzen
%gravierende sytematische probleme bei den messungen nochmal betonen und die wertigkeit unserer ergebnisse einordnen
Im Versuch wurden Eigenschaften von $\alpha$-Strahlung untersucht. Dabei wurde im ersten Teil Eigenschaften mittels der Rutherfordstreuung untersucht. F�r die Streuung wurde Gold und Aluminium verwendet. Es konnte gezeigt werde, dass f�r Winkel (bis 30$^\circ$) das Streuverhalten mit der Rutherfordstreuformel beschrieben werden kann. Danach wurde gezeigt, das f�r gro�e Winke, die Rutherfordsteuformel nicht mehr zutrifft. Mit der Erkenntnissen aus der Streuung an Gold wurde die Kernladungszahl von Aluminium bestimmt, diese konnte mit einer Abweichung von 097\% zu erwartetem Wert bestimmt werden. Die Reichweite von $\alpha$-Strahlung wurde mit 2,98 cm bestimmt, erwartet wurde ein Wert von 2,96 cm, damit weicht der bestimmte Wert um 0,67\% vom erwartetem Wert ab. Bei der Untersuchung des Energieverlusts von $\alpha$-Strahlung in Luft konnte der erwartet Verlauf der Kurven best�tigt werden. Im letztem wurde die Absorption von $\alpha$-Strahlung qualitativ f�r Papier, Biebelpapier und Alufolie beschrieben.