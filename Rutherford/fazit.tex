\section{Fazit}
%im fazit nochmal alles zusammenfassen und den verlauf der messung absch�tzen
%gravierende sytematische probleme bei den messungen nochmal betonen und die wertigkeit unserer ergebnisse einordnen
Im diesem Versuch wurden die Eigenschaften von $\alpha$-Strahlung untersucht. Die Rutherfordstreuung sollte im ersten Teil �berpr�ft werden. F�r die Streuung wurde Gold und Aluminium verwendet. Es konnte gezeigt werde, dass f�r Winkel (bis 30$^\circ$) das Streuverhalten mit der Rutherfordstreuformel beschrieben werden kann. Danach wurde gezeigt, dass die Rutherfordsteuformel f�r gro�e Winkel nicht gilt. Die Kernladungszahl von Aluminium wurde dann mit den Ergebnissen aus der Streuung an einer Goldfolie bestimmt. Diese konnte mit einer Abweichung von 0,97\% zu erwartetem Wert (13) bestimmt werden. Die Kanal-Energie-Eichung konnte erfolgreich durchgef�hrt werden, wobei sich ein Chiquadrat von 0.367 bei dem Gradenfit ergab. Die Reichweite von $\alpha$-Strahlung wurde mit 3,04 $\pm$ 0,37 cm bestimmt. Erwartet wurde ein Wert von 2,96 cm. Damit weicht der bestimmte Wert um 2,7\% vom erwartetem Wert ab. Bei der Untersuchung des Energieverlusts von $\alpha$-Strahlung in Luft konnte der erwartete Verlauf der Kurven best�tigt werden. Im letztem Abschnitt wurde die Absorption von $\alpha$-Strahlung qualitativ f�r Papier, Biebelpapier und Alufolie beschrieben.