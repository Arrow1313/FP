\section{Einleitung}
%einleitung zu dem experiment.
%auf die einstellungen, die vor dem versuch gemacht werden, eingehen, oder auf eine anleitung dazu verweisen.
%---------------------------------------------------------------------------------------------
%hinter der einleitung kann der allgemeine theoretische hintergrund in einer zus�tzlichen section erkl�rt werden


In diesem Versuch sollen verschieden Oberfl�chen mittels Rasttunnelmikroskops, auf unterschiedliche Eigenschaften untersucht werden. Untersucht werden die Oberfl�chenrauheit, die Elektronendichte und die atomare Gitterstruktur. Mit dem Rasttunnelmikroskop werden leitf�hige Materialien, unter Verwendung des Tunneleffekts untersucht. Dabei wird zwischen des Kopfs des Mikroskops und der Oberfl�che des Materials eine Spannung angelegt, sodass die Potentialdifferenz durchtunnelt werde kann und so ein Tunnelstrom flie�t. Durch systematisches ``abtasten'' kann die elektrische Oberfl�chenstrucktur untersucht werden. Es sollen drei unterschiedliche Proben untersucht werden, Graphit, (111)-Goldschicht und eine unbekannte Probe.

