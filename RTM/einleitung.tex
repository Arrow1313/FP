\section{Einleitung}
%einleitung zu dem experiment.
%auf die einstellungen, die vor dem versuch gemacht werden, eingehen, oder auf eine anleitung dazu verweisen.
%---------------------------------------------------------------------------------------------
%hinter der einleitung kann der allgemeine theoretische hintergrund in einer zus�tzlichen section erkl�rt werden


In diesem Versuch werden Oberfl�chen verschiedener Proben mittels Rasttunnelmikroskopie auf deren Gitterstruktur und morphologische Eigenschaften untersucht. Elektronendichte, Oberfl�chenrauheit und die atomare Gitterstruktur k�nnen mit dem Rastertunnelmikroskop analysiert werden. Der quantenmechanische Tunneleffekt wird genutzt, um leitende Materialien zu untersuchen. Indem zwischen einer einatomigen Platin-Iridum-Elektrode und der zu untersuchenden Probe eine Potentialdifferenz erzeugt wird, kommt es abh�ngig von der Entfernung der Pt-Ir-Elektrode zur Probe und dessen Elektronendichte zu einem Tunnelstrom, welcher R�ckschl�sse auf die Struktur der Probe erlaubt.