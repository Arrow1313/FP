\section{Fazit}
%im fazit nochmal alles zusammenfassen und den verlauf der messung absch�tzen
%gravierende sytematische probleme bei den messungen nochmal betonen und die wertigkeit unserer ergebnisse einordnen
In diesem Versuch wurden [HOP]-Graphit und eine Gold-(111)-Schicht mit einem Rastertunnelmikroskop untersucht. Es wurden die Oberfl�chen- und Linienrauheit, sowie die Gitterstruktur der Proben untersucht. Zuerst wurde die Graphitprobe untersucht. Die Oberfl�chenrauheit $S_a$  wurde mit 10,471pm bestimmt, die Linienrauheit R$_a$ wurde mit 7,016pm bestimmt. Bei der Untersuchung der Gitterstruktur konnte die Hexagonalstruktur deutlich gezeigt werden (vgl. Abbildung \ref{fig:graphit_1_1_nm_struktur}). Die Gitterkonstante wurde mit $\bar{a} = \SI{221(1)}{pm}$ bestimmt, was einer relativen Abweichung von 10\% entspricht. Die Abweichung konnte zum Gro�teil mit systematischen Fehlern erkl�rt werden.
Bei der Untersuchung der Goldprobe wurde die Oberfl�chenrauheit $S_a$ f�r einen Bereich von 202 x 202 nm mit 220,09 pm bestimmt. Die Linienrauheit wurde mit 57,353pm bestimmt. Der Netzebenenabstand der Gold(111)-Probe wurde mit 237(4)nm bestimmt. Dies entspricht einer relativen Abweichung von 0,42\%. Daneben waren viele Gitterdefekte, welche nicht Vielfache des Gitterabstandes von Gold waren, zu sehen. Insgesamt ergeben sich aus den Messungen akzeptable Resultate.