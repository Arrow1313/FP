\section{Fazit}
%im fazit nochmal alles zusammenfassen und den verlauf der messung absch�tzen
%gravierende sytematische probleme bei den messungen nochmal betonen und die wertigkeit unserer ergebnisse einordnen
Im Versuch wurden Graphit und Gold(111) mittels eines Rastertunnelmikroskops untersucht. Es wurden die Oberfl�chen- und Linienrauheit, sowie die Gitterstruktur der Proben untersucht. Zu erst wurde die Graphitprobe untersucht, dabei wurde die Oberfl�chenrauheit S$_a$ mit 10,471pm bestimmt, die Linienrauheit R$_a$ wurde mit 7,016pm bestimmt. Bei der Untersuchung der Gitterstruktur konnte die Hexagonalestrktur deutlich gezeigt werden (vgl. Abbildung \ref{fig:graphit_1_1_nm_struktur}). Die Gitterkonstanten wurde mit ?? bestimmt.
Bei der Untersuchung der Goldprobe wurde die Oberfl�chenrauheit S$_a$ f�r einen Bereich von 202 x 202 nm mit 220,09 pm bestimmt. Die Linienrauheit wurde mit 57,353pm bestimmt. Der Netzebenenabstand der Gold(111)-Probe wurde mit 237(4)nm bestimmt, dies entspricht einer relativen Abweichung von 0,42\%.