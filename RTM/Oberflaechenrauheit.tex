\subsection{Oberfl�chenrauheit}
Die Oberfl�chenrauheit oder kurz Rauheit ist in der Oberfl�chenphysik eine Gr��e zur Charakterisierung der Unebenheit der Oberfl�chenh�he.
Beschrieben wird sie mit der mittleren betragsm��igen Abweichung von der mittleren Oberfl�chenh�he, oder mit der mittleren quadratischen Abweichung von der Oberfl�chenh�he. Die Formeln sind im Zweidimensionalen, wie auch im Dreidimensionalen, auf Hyperfl�chen anwendbar. F�r die Berechung der Rauheit wird Formel \ref{eqn:R_a} und Formel \ref{eqn:R_q} f�r das arithmetische Mittel $R_a$ und f�r das quadratische Mittel $R_q$ verwendet.
\begin{align}
R_a = \frac{1}{N}\sum_{i=1}^{N}|z_i-\bar{z}|
\label{eqn:R_a}
\\
R_q = \frac{1}{N}\sqrt{\sum_{i=1}^{N}(z_i-\bar{z})^2}
\label{eqn:R_q}
\end{align}
Das quadratische Mittel gewichtet gr��ere Abweichungen st�rker als das arithmetische Mittel.
Der Mittelwert $\bar{z}$ wird �ber Formel \ref{eqn:bar_z} bestimmt.
\begin{align}
\bar{z} = \frac{1}{N}\sum_{i=1}^{N}z_i
\label{eqn:bar_z}
\end{align}