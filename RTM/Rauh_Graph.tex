\subsubsection{Rauheit}
Die Rauheit der Graphitschicht konnte mit der Software f�r das RTM bestimmt werden, nachdem die Graphitschicht im CC-Modus mit der Spitze des RTM abgerastert wurde. Die Verkippung wurde bestimmt und korrigiert. Das Fenster Topography-Scan ist f�r die Messung nicht wichtig, da es die Oberfl�chenbeschaffenheit ausschlie�lich auf H�he des schwarzen Pfeils angibt. In Abbildung \ref{fig:lin_rau_graph} ist die Messung zur Bestimmung der mittleren Linienrauheit dargestellt.
\begin{figure}[H]
\centering
\includegraphics[scale = 0.75]{snipping_122_linienrauheit}
\caption{ Mittlere Linienrauheit von Graphit ($R_a = \SI{7,0159}{pm}$ und $R_q = \SI{9,155}{pm}$) }
\label{fig:lin_rau_graph}
\end{figure}
Ebenso wurde die mittlere Fl�chenrauheit der Graphitprobe bestimmt, welche in Abbildung \ref{fig:flae_rau_graph} dargestellt ist.
\begin{figure}[H]
\centering
\includegraphics[scale = 0.75]{snipping_122_flaechenrauheit}
\caption{ Mittlere Fl�chenrauheit von Graphit ($S_a = \SI{10,471}{pm}$ und $S_q = \SI{13,1}{pm}$) }
\label{fig:flae_rau_graph}
\end{figure}