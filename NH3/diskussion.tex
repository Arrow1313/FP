\subsection{Discussion}
%(immer) die gemessenen werte und die bestimmten werte �ber die messfehler mit literaturwerten oder untereinander vergleichen
%in welchem fehlerintervall des messwertes liegt der literaturwert oder der vergleichswert?
%wie ist der relative anteil des fehlers am messwert und damit die qualit�t unserer messung?
%in einem satz erkl�ren, wie gut unser fehler und damit unsere messung ist
%kurz erl�utern, wie systematische fehler unsere messung beeinflusst haben k�nnten
%(wichtig) zum schluss ansprechen, in wie weit die ergebnisse mit der theoretischen vorhersage �bereinstimmen
%--------------------------------------------------------------------------------------------
%falls tabellen mit den messwerten zu lang werden, kann die section mit den messwerten auch hinter der diskussion angef�gt bzw. eine section mit dem anhang eingef�gt werden.
In this section the result will be discussed and the measured values will be compared with the literature values.

In the first part of the experiment all the 39 peaks for the NH$_3$ spectrum where measured. The relative absorption coefficients can be seen in figure \ref{fig:alpha}, the vales are given in table \ref{tab:data_39_1},\ref{tab:data_39_2}. All peaks where found and the the peaks, that where expected to be the highest where measured as the highest.

In the second part of the experiment the quadrupole moment should be calculated using the hyperfine structure. The quadrupole moment was calculated with the distance of the side peaks to the main peak, with equation \ref{eqn:Hyperfine}. The quadrupole moment was determent with 5.02(9), the literature vale is 4.14(6) \cite{hf_book}. The relative deviation is 17.53\%. The source of the high deviation is not know but seems to be a systematic error, an indication for hat is the weak signal of the hyperfine structure, see figure \ref{fig:hf_4_4}.

In the last part of the experiment the broadening of the peak width depended on the pressure. The line broadening $\Delta$ was determent with 28(1) $\frac{\text{MHz}}{\text{mmHg}}$. The literature value is 29(2)$\frac{\text{MHz}}{\text{mmHg}}$ \cite{examenarbeit}. The measurement can be called a success because the literature value lies in the 1-$\sigma$ interval of the measured value. 