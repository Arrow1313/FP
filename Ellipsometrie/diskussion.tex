\subsection{Diskussion}
%(immer) die gemessenen werte und die bestimmten werte �ber die messfehler mit literaturwerten oder untereinander vergleichen
%in welchem fehlerintervall des messwertes liegt der literaturwert oder der vergleichswert?
%wie ist der relative anteil des fehlers am messwert und damit die qualit�t unserer messung?
%in einem satz erkl�ren, wie gut unser fehler und damit unsere messung ist
%kurz erl�utern, wie systematische fehler unsere messung beeinflusst haben k�nnten
%(wichtig) zum schluss ansprechen, in wie weit die ergebnisse mit der theoretischen vorhersage �bereinstimmen
%--------------------------------------------------------------------------------------------
%falls tabellen mit den messwerten zu lang werden, kann die section mit den messwerten auch hinter der diskussion angef�gt bzw. eine section mit dem anhang eingef�gt werden.

Die Reihenfolge in der die prozentualen Abweichungen angegeben sind ist immer Abweichung-Fit, Abweichung-Diagramm.


F�r Kupfer wurde mit dem Fit ein Wert von n' = 0,31(2) und $\kappa$ = 2,2(2) bestimmt, mit dem Diagramm wurde f�r n' ein Wert von 0,25 und f�r $\kappa$ ein Wert von 2,2 bestimmt. Der Brechungsindex liegt im Bereich der Werte aus der Datenbank (0,08-0,4), der Absorptionsindex weicht jedoch um 42,11\% von dem Werten aus den Datenbank ab.


Bei Aluminium wurde durch den Fit ein n' von 0,70(8) und ein $\kappa$ von 4,4(2) bestimmt. Aus dem Diagramm ergab sich f�r n' = 0,6 und $\kappa$ = 4. In der Datenbank war f�r n' Werte von 1,36 bis 1,7 angegeben, das entspricht eine Abweichung von 48,53\% und 55,88\%. F�r $\kappa$ waren Werte zwischen 7 und 8,1 angegeben, dies entspricht eine Abweichung von 37,14\% und 42,85\%.


F�r Germanium wurde n' mit 4(2) und $\kappa$ mit 2(2) durch den Fit bestimmt. Durch das Diagramm wurde f�r n' eine Werte von 3,7 bestimmt, f�r $\kappa$ ergab sich eine Wert von 2,2. Die n' weichen um 21,56\% und 27,45\% ab. Die Abweichung von $\kappa$ betr�gt 300\% und 340\%.


Mit dem Fit wurde ein n' von 1,467(6) und ein $\kappa$ von 0,04(12) f�r Glas bestimmt. Mit dem Graphen ergab sich f�r n' ein Wert von 1,4 und f�r $\kappa$ ein Wert von 0. In der Datenbank war f�r n' Werte zwischen 1,456 und 1,550 angegeben somit stimmt der durch den Fit bestimmt Wert mit den Erwartungen �berein und der aus dem Graphen bestimmte Wert weicht um 3,84\% ab. $\kappa$ weicht um 2000\% ab, der Fehler ist jedoch gr��er als der bestimmte Wert und liegt mit Fehler im erwartetem Bereich. Der aus dem Diagramm bestimmte Wert f�r $\kappa$ kann keine Abweichung bestimmt werden, da der Wert 0 ist.


F�r das Kochsalz wurde n' mit 1,51(1) und $\kappa$ mit 0,19(5) bestimmt. Aus dem Diagramm ergab sich f�r n' ein Wert von 1,6 und f�r $\kappa$ ein Wert von 0. In der Datenbank wurde kein Wert f�r $\kappa$ angegeben und f�r n' wurde nur ein Wert von 1,54 angegeben. Der Wert aus dem Fit weicht um 1,94\% und der Wert aus dem Diagramm um 2,59\% ab.

Bei Silizium wurde n' mit 3,97(8) und $\kappa$ mit 0(10) durch den Fit bestimmt. Aus dem Diagramm wurde f�r n' ein Wert von 3,8 und f�r $\kappa$ ein Wert von 0 bestimmt. F�r n' wurden Werte zwischen 3,81 und 3,92 angegeben, damit liegen beide Werte im erwartetem Bereich. F�r $\kappa$ wurden Werte von 0,013 bis 0,017 angegeben. Da f�r den Fit und das Diagramm ein Wert von 0 bestimmt wurde, kann eine relative Abweichung angegeben werden.


Bei den ersten Drei Materialien liegen die relativen Abweichungen von n' im Bereich von 20\% bis 58\%, abgesehen von n' f�r Kupfer, welches im erwartetem Bereich liegt. Dies Abweichungen k�nnen durch Verschmutzungen und Oxidationen der Oberfl�che kommen.
Die Kupferprobe wurde nach schlechten Messergebnissen gereinigt, wodurch sich der Wert f�r n' von 0,41(5) auf 0,31(1) �nderte und der f�r $\kappa$ von 0,23(6) auf 2,2(1) �nderte. Das l�sst vermuten dass f�r einen korrekte Bestimmung des Absorptionsindex die Reinheit der Oberfl�chen von sehr wichtig ist. Wenn man sich alle Messungen anguckt so f�llt auch auf, das die Abweichung der Brechungsindexe geringer ist, als die Abweichung der Absorptionsindexe.