\section{Versuchsdurchf�hrung}
%erkl�ren, !was! wir machen, !warum! wir das machen und mit welchem ziel
%(wichtig) pr�zize erkl�ren, wie bei dem versuch vorgegangen und was gemacht wurde

Die Durchf�hrung setzt sich mehren Teilen zusammen, der erste Teil ist die Justage, welche sich in zwei Schritte aufteilt. Die Justage der Probe muss f�r jede Probe einzeln wiederholt werden. Nach der Justage k�nnen die Messungen gemacht werden. Bei der ersten Messung wird die Intensit�t in Abh�ngigkeit der Polarisationsebene untersucht.

\subsection{Justage}
Bevor Messungen gemacht werden k�nnen muss die Messapertur justiert werden.

Zuerst wird der Detektorarm justiert/seine Nulllate festzulegen. Daf�r wird der Laserstrahl auf die Si-Photodiode fokussiert und der Rotationstisch 1 solange gedreht, bis sich ein Maximum in der Intensit�t ergibt.

Danach werden die Proben mit dem Laser justiert, dieser Justageschritt ist bei jedem Wechsel der Probe n�tig. Dieser Schritt ist besonders wichtig, da eine gute Messung von ihm Abh�ngt. Zuerst wird Rotationstisch 2 auf einen Einfallswinkel von $\theta$ = 0$^\circ$ eingestellt und eine d�nne Glasscheibe angebracht. Dabei sollte die im Detektor die maximale Intensit�t zu messen sein. Die Probe wird dann �ber den Lineartisch so weit in den Strahl gefahren, dass am Detektor nur noch die H�lfte der maximalen Intensit�t gemessen wird (die Positon x$_{1/2} $). Dann wird mit dem ersten Goniometer, die Neigung der Probenoberfl�che senkrecht zum Strahl so ge�ndert, dass sich eine maximale Intensit�t einstellt (der Winkel $\Phi_0$). Es wird �berpr�ft, ob die x$_{1/2}$-Position sich nicht ge�ndert hat. Falls sie sich doch ge�ndert hat muss die Justierung des Goniometers wiederholt werden. Der Rotationstisch 1 wird um 90$^\circ$ gedreht und das zweite Goniometer so eingestellt, dass der Strahl exakt reflektiert wird.

\subsection{Messungen}
Es werden die Durchf�hrungen der drei verschiedenen Messungen beschrieben.

\subsubsection{Winkelabh�ngigkeit der Intensit�t}
Mit der Messung soll �berpr�ft werden, wie sich die Intensit�t bei Drehung der Polarisationsebene und Einsatz des Linearpolarisators �ndert. Daf�r wird der Polarisationsdreher auf der optischen Bank befestigt und die Glasscheibe aus dem Strahl gefahren. Dann wird die Spannung in Abh�ngigkeit des Winkels des Polarisators aufgenommen.

\subsubsection{Parallele und senkrechte Polarisation}
Es ist wichtig, f�r die weiteren Messungen die Winkeleinstellung am Polarisator, unter dem der Strahl parallel oder senkrecht zur Einfallsebene Polarisiert ist. Bei einem 90$^\circ $ Winkel wird erwartet, dass der Strahl parallel zur Einfallsebene polarisiert ist. Um zu �berpr�fen, ob der Strahl parallel zur Einfallsebene polarisiert ist, wird die Glasscheibe wieder in den Strahl gestellt und auf die x$_{1/2}$-Position gestellt. Dann wird der Winkel von Rotationstisch 2 auf den theoretischen Brewsterwinkel (57$^\circ$) des Glases eingestellt. Um zu verifizieren, dass der gesuchte Winkel bei 90$^\circ$ liegt wird der Winkelbereich von 70$^\circ$ bis 110$^\circ$ durchgemessen und das Minimum der Intensit�t gesucht.

\subsubsection{Reflexionmessung}
Es wird der gew�nschte Winkel an Rotationstisch 2 eingestellt und die Position des reflektieren Strahls gemessen. Dann werden an dem Polarisationsdreher die zuvor bestimmten Werte f�r parallel und senkrecht Ausrichtung des Strahls zur Einfallsebene eingestellt.