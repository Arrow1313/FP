\section{Fazit}
%im fazit nochmal alles zusammenfassen und den verlauf der messung absch�tzen
%gravierende sytematische probleme bei den messungen nochmal betonen und die wertigkeit unserer ergebnisse einordnen

Im dem Versuch wurden der Brechungsindex n' und der Absorptionsindex $\kappa$ f�r Kupfer, Aluminium, Germanium, Glas, Kochsalz und Silizium bestimmt. F�r Kupfer, Glas, Kochsalz und Silizium stimmte der bestimmte Wert von n' mit den erwarteten Werten �berein oder wich um weniger als 2,6\% ab. Bei den anderen Materialien lag die Abweichung von den erwarteten Werten im Bereich von 20\% bis 58\%. Die Werte von $\kappa$ weichen zwischen 37\% und 2000\% ab. Bei den Werten, bei denen die Abweichung im Bereich von einigen 100\%-2000\% liegen sind die bestimmten und zu erwartenden Werte alle sehr klein und der Fehler auf die bestimmten Werte ist immer so gro�e dass der erwartet Wert im ersten Fehlerintervall des bestimmten Wertes liegt.