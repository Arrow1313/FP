\section{Einleitung}
%einleitung zu dem experiment.
%auf die einstellungen, die vor dem versuch gemacht werden, eingehen, oder auf eine anleitung dazu verweisen.
%---------------------------------------------------------------------------------------------
%hinter der einleitung kann der allgemeine theoretische hintergrund in einer zus�tzlichen section erkl�rt werden
Ziel des Versuches ist die Bestimmung der komplexen Brechzahl $ n = n' + i\kappa$ f�r verschiedene Metalle, Halbmetalle, sowie Isolatoren. F�r die Bestimmung der Konstanten $n'$ und $\kappa$ wird in diesem Praktikumsversuch die von D.G. Avery entwickelte $R_p/R_s$-Methode herangezogen. Die Fresnelschen Formeln, welche aus den Maxwellgleichungen und den Grenzbedinungen der Felder E, D, B und H beim �bergang zwischen Medien verschiedener Permittivit�t und Permeabilit�t (in diesem Versuch $\mu_r = 1$) folgen, liefern einen analytischen Zusammenhang zwischen den Reflektivit�ten bei paralleler und senkrechter Polarisation (zur Einfallsebene) und den Konstanten $n'$ und $\kappa$.