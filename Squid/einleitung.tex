\section{Einleitung}
%einleitung zu dem experiment.
%auf die einstellungen, die vor dem versuch gemacht werden, eingehen, oder auf eine anleitung dazu verweisen.
%---------------------------------------------------------------------------------------------
%hinter der einleitung kann der allgemeine theoretische hintergrund in einer zus�tzlichen section erkl�rt werden

In diesem Versuch werden Magnetfelder mit Hilfe eines rf-SQUID untersucht. Ein SQUID (Superconducting Quantum Interference Device) kann mit Hilfe von Supraleitung Magnetfeld�nderungen in Form von Flussquanten messen. Ziel des Versuches ist es die Funktionsweise des rf-SQUID, sowie die Grundlagen der Supraleitung und der elektromagnetischen Abschirmung zu erarbeiten, und sich mit dem rf-SQUID vertraut zu machen, wobei sich der Ablauf des Versuches in f�nf Teile gliedern l�sst. Zuerst wird das SQUID durch Abk�hlen in Fl�ssigstickstoff und anschlie�ender Justage in Betrieb genommen, sodass die Empfindlichkeit des SQUID mit einer Stromdurchflossenen Leiterschleife abgesch�tzt werden kann. Danach soll das SQUID mithilfe eines wohldefinierten Magnetfeldes kalibriert werden, um anschlie�end das Magnetfeld eines fahrenden Aufzugs im Universit�tsgeb�ude zu vermessen. Zuletzt soll die Curie-Temperatur von Gadolinium (Gd) bestimmt werden, indem die Magnetisierung eines Gd-W�rfels in Abh�ngigkeit von der Temperatur aufgenommen wird.
