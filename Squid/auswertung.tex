\section{Versuchsdurchf�hrung und Auswertung}
%die messwerte in !�bersichtlichen! tabellen angegeben
%zu viele kleine tabellen in gro�e tabellen �berf�hren!
%zu gro�e tabellen mit dem [scale]-befehl scalieren oder (falls zu lang) in zwei kleinere tabellen aufteilen
%(wichtig) vor !jeder! tabeelle sagen, was gemessen wurde und wie die fehler gew�hlt wurden und ausreichend !erkl�ren!, !warum! wir unsere fehler grade so gew�hlt haben

%zuerst !alle! errechneten werte entweder in ganzen s�tzen aufz�hlen, oder in tabellen (�bersichtlicher) dargestellen, sowie auf die verwendeten formeln verweisen (die referenzierung der formel kann in der �berschrift stehen)
%kurz erw�hnen (vor der tabelle), warum wir das ganze ausrechnen bzw. was wir dort ausrechnen
%danach histogramme und plots erstellen, wobei wenn m�glich funktionen durch die plots gelegt werden (zur not k�nnen auch splines benutzt werden, was aber angegeben werden muss)
%bei fits immer die funktion und das reduzierte chiquadrat mit angegeben, wobei auf verst�ndlichkeit beim entziffern der zehnerpotenzen geachtet werden muss z.b. f(x)=(wert+-fehler)\cdot10^{irgendeine zahl}\cdot x + (wert+-fehler)\cdot10^{irgendeine zahl}
%bei jedem fit erkl�ren, nach welchem zusammenhang gefittet wurde und warum!
%bei plots darauf achten, dass die achsenbeschriftung (auch die tics) die richtige gr��e haben und die legende im plot nicht die messwerte verdeckt
%kurz die aufgabenstellung abgehandeln

Der Versuch besteht aus f�nf Teile, welche im folgendem beschrieben und ausgewertet werden.

\subsection{Inbetriebnahme}
Vor dem einstellen des SQUID muss das Dewar-Gef�� in ein Bad von fl�ssigem Stickstoff getaucht werden, damit der Sensor nicht durch Feuchtigkeit zerst�rt wird und die Supraleitung nicht unterbrochen wird. Das Bad von fl�ssigem Stickstoff muss eventuell w�hrend dem Versuch nachgef�llt werden. Der Abk�hlprozess dauert ca. 20 min, danach k�nnen Messungen gestartet werden. Zu erst soll die Amplitude I$_{rf}$ (VCA, voltage controlled atteuator) und die Auslesefrequenz (VCO voltage controlled oscillator) des rf-Signals so ein maximales Signal-zu-Rausch-Verh�ltnis von U($\Phi$) erreicht wird. F�r die Einstellungen wird der Testmodus, mit eingeschaltetem Generator verwendet. Zu erst wird f�r VCA ein Werte von 900 eingestellt.
% Kapazit�ts- und Widerstandsangaben hinzuf�gen
Dann wird der VCO Wert zwischen 0 und 4095 so variiert, das m�glichst deutlich ein Dreiecksignal zu sehen ist. Der VCA Wert wird nun nochmal variiert um das Signal weiter zu optimieren.

%  Bild und die Werte einf�gen

F�r die Einstellung des Arbeitspunkts wird das Offsets kalibriert. Das SQUID wird im Messmodus betrieben um die zeitliche Magnetfeld�nderung auf dem Oszilloskop zu beobachten. Es wird ein Kompensations-Widerstand von 20k$\Omega$ gew�hlt. Falls das Offset richtig eingestellt ist sollten keine Peaks auf dem Oszilloskop zu sehen sein. Falls Peaks nach oben zu sehen sind, ist das Offset zu hoch eingestellt. Bei Peaks, die nach unten zeige, ist das Offset zu niedrig eingestellt.