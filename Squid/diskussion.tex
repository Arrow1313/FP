\subsection{Diskussion}
%(immer) die gemessenen werte und die bestimmten werte �ber die messfehler mit literaturwerten oder untereinander vergleichen
%in welchem fehlerintervall des messwertes liegt der literaturwert oder der vergleichswert?
%wie ist der relative anteil des fehlers am messwert und damit die qualit�t unserer messung?
%in einem satz erkl�ren, wie gut unser fehler und damit unsere messung ist
%kurz erl�utern, wie systematische fehler unsere messung beeinflusst haben k�nnten
%(wichtig) zum schluss ansprechen, in wie weit die ergebnisse mit der theoretischen vorhersage �bereinstimmen
%--------------------------------------------------------------------------------------------
%falls tabellen mit den messwerten zu lang werden, kann die section mit den messwerten auch hinter der diskussion angef�gt bzw. eine section mit dem anhang eingef�gt werden.

In diesem Versuch wurden Magnetfelder mittels eines rf-SQUID untersucht.
Daf�r wurde das SQUID zu erst justiert und es wurde nach St�rsignalen gesucht. Bei 50Hz wurde ein St�rsignal gefunden, welches wahrscheinlich auf Netzbrummen zur�ck zu f�hren ist. Nach der Justierung waren VCO auf 2777, VCA auf 547 und das Offset auf 2123 eingestellt.
Im zweitem Teil des Versuchs sollte die Empfindlichkeit des SQUID untersucht werden, daf�r wurde zu erste eine qualitative Analyse der Empfindlichkeit durchgef�hrt. Es wurde die Auswirkung des Schwenken eines Stuhls und eines Permanentmagneten untersucht. Zus�tzlich wurde noch der U $\propto \frac{1}{r^3}$ Zusammenhang untersucht. F�r den Fit ergab sich ein $\chi_{red}^2$ von 2.318, was darauf hindeuten k�nnte, das der Fehler f�r den Abstand zu gering angenommen wurde. Zudem musste der Permanentmagnet bei jedem Abstand mit der selben Geschwindigkeit gedreht werden, um ein vergleichbares Ergebnis zu erhalten, was eine weitere Fehlerquelle ist. Damit spiegelt das $\chi_{red}^2$ von 2.318 eine gute Messung wieder. F�r die quantitative Analyse wurde der minimale Strom gesucht, der eine �nderung der Spannung verursacht, was bei einem Strom von 5mA passierte, was einen viertel Flussquant entspricht.
Im dritten Teil wurde eine Eichung durchgef�hrt um im nachfolgenden Versuchsteil das Dipolmoment des Aufzuges zu bestimmen. Daf�r wurde der Zusammenhang zwischen Strom und Spannung f�r die einzelnen Widerst�nde untersucht, in dem ein linearer Zusammenhang gefittet wurde. Die $\chi_{red}^2$ ergaben sich zwischen 0.3 bis 2.1 wobei die meisten nahe bei 1 waren, was f�r gute Messungen spricht. Aus den bestimmten Empfindlichkeiten wurde durch einen linearen Fit der Zusammenhang zwischen Empfindlichkeit und dem Widerstand untersucht. Dabei ergab sich ein $\chi_{red}^2$ von 4.41, was auf nicht ber�cksichtigte St�rquellen schlie�en l�sst. Damit wurde die Eichung f�r den 20k$\Omega$ bestimmt, es ergab sich ein Wert von 163(2)V$\mu$T.
Im letztem Versuchsteil wurde das Dipolmoment des Aufzuges untersucht, dabei ergab sich ein Wert von 2295Am$^2$.