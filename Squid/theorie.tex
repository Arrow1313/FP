\section{Theorie}
% Es sollen die wichtigsten theoretischen Formeln und Zusammenh�nge einmal ausf�hrlich erkl�rt werden

In diesem Abschnitt sollen die theoretischen Grundlagen des Versuchs erarbeitet werden.

\subsection{Supraleitung}

Der Effekt der Supraleitung tritt bei vielen Metallen und Legierungen auf, welche bei sehr niedrigen Temperaturen(T$_c$) ihren elektrischen Widerstand verlieren. Der Effekt wurde 1911 von Kammerlingh Onnes in Leiden entdeckt. Im Versuch wird YBa$_2$Cu$_3$O$_7$ verwendet, bei dem Material handelt es sich um ein Hochtemperatursupraleiter, welcher unterhalb von 77K betrieben wird.

\subsubsection{Mei�ner-Ochsenfeld-Effekt}
Der Mei�ner-Ochsenfeld-Effekt wurde 1993 entdeckt und beschreibt das magnetische Verhalten eines Supraleiters im einem �u�erem magnetischen Feld. Das �u�ere Magnetfeld wird aus dem Supraleiter heraus gedr�ngt(siehe Abb. ??).

% Bild einf�gen

Es gilt innerhalb des Supraleiters $B=0$ und $\overset{\cdot}{B}=0$. Dadurch verh�lt sich der Supraleiter wie ein perfekter Diamagnet. Es hat sich herausgestellt, das man Supraleiter in zwei verschiedene Arten unterscheiden kann (siehe Abb. ??). Supraleiter erster Art verhalten sich wie oben beschrieben und die Magnetisierung f�llt direkt auf 0 ab. Bei Supraleiter zweiter Art f�llt die Magnetisierung nicht direkt ab sonder hat einen exponentiellen Abfall.

% Bild einf�gen

\subsubsection{BCS-Theorie}
Eine quantenmechanische Theorie der Supraleitung wurde 1957 von Bardeen, Cooper und Schrieffer aufgestellt, diese wird die BCS-Theorie genannt.