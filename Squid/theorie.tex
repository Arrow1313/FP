\section{Theorie}
% Es sollen die wichtigsten theoretischen Formeln und Zusammenh�nge einmal ausf�hrlich erkl�rt werden

In diesem Abschnitt werden die theoretischen Grundlagen des Versuchs dargestellt.

\subsection{Supraleitung}

Der Effekt der Supraleitung tritt bei vielen Metallen und Legierungen auf, welche bei sehr niedrigen Temperaturen (T$_c$) ihren elektrischen Widerstand verlieren. 1911 wurde er von \textit{Heike Kammerlingh Ones} in Leiden entdeckt. Im Versuch wird YBa$_2$Cu$_3$O$_7$ verwendet. Bei diesem Material handelt es sich um einen Hochtemperatursupraleiter, welcher unterhalb von \SI{77}{K} supraleitend wird.

\subsubsection{Mei�ner-Ochsenfeld-Effekt}
Der Mei�ner-Ochsenfeld-Effekt wurde 1993 entdeckt und beschreibt das magnetische Verhalten eines Supraleiters in einem �u�eren magnetischen Feld. Das �u�ere Magnetfeld wird aus dem Supraleiter 'herausgedr�ngt' (siehe Abb. ??).

% Bild einf�gen

Innerhalb des Supraleiters gilt $B=0$ und $\overset{\cdot}{B}=0$. Dadurch verh�lt sich der Supraleiter wie ein perfekter Diamagnet. Es hat sich herausgestellt, dass man Supraleiter in zwei verschiedene Arten unterteilen kann (siehe Abb. ??). Supraleiter erster Art verhalten sich wie oben beschrieben und die Magnetisierung f�llt direkt auf 0 ab. Bei Supraleitern zweiter Art f�llt die Magnetisierung nicht direkt ab sondern hat einen exponentiellen Abfall.

% Bild einf�gen

\subsubsection{BCS-Theorie}
Eine quantenmechanische Theorie der Supraleitung wurde 1957 von Bardeen, Cooper und Schrieffer aufgestellt, diese wird die BCS-Theorie genannt.

\subsection{SQUID}
In diesem Abschnitt werden kurz der grundlegende Aufbau des SQUIDs beschrieben und dann genauer auf das rf-SQUID eingegangen.
Ein SQUID besteht aus einer supraleitenden Schleife mit ein(rf-SQUID) oder zwei(dc-SQUID) Schwachstellen. Die Schwachstelle hat eine begrenzte Stromtragf�higkeit, wodurch sie bei gro�en externen �nderungen des Flusses resistiv werden kann. Dadurch wird der Suprastrom f�r eine sehr kurze Zeit unterbrochen, was mit einem Resonanzschwingkreis gemessen werden kann.

%Bild einf�gen

Der schematische Aufbau eines rf-SQUIDs ist in Abb. ?? zu sehen. Der Supraleitende Ring hat nur eine Schwachstelle und ist induktiv an einen Schwingkreis gekoppelt. Das rf-SQUID wird mit einem Wechselfeld mit einer Frequenz zwischen \SI{100}{MHz} und \SI{1}{GHz} betrieben.

Durch den Suprastrom alle �nderungen des �u�eren Feldes unterhalb des n�chsten Vielfachen eines Flussquants kompensiert. Beim anliegen eines �u�eren Feldes kann das Verhalten des SQUIDs in Abh�ngigkeit von I$_{rf}$ untersucht werden. Der Zusammenhang ist in Abb. ?? dargestellt. 