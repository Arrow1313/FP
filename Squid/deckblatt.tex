%deckblatt erstellen.

\begin{titlepage}

\begin{center}
% Oberer Teil der Titelseite:
\includegraphics[width=0.75\textwidth]{logo.png}\\[0.8cm]    	%Logo 

\textsc{\LARGE Bergische Universit�t Wuppertal}\\[1.cm]	%Institution

\textsc{\Large Fortgeschrittenen Praktikum}\\[0.4cm]				%Projekt


\newcommand{\HRule}{\rule{\linewidth}{0.4mm}}
\HRule \\[0.4cm]
{ \huge \bfseries RF-SQUID}\\
\textbf{R}adio \textbf{F}requency \textbf{S}uperconducting \textbf{Qu}antum \textbf{I}nterference \textbf{D}evice\\[0.4cm]				%Titel

\HRule \\[1.cm]

% Author und Tutor
\begin{minipage}{0.4\textwidth}
\begin{flushleft} \large
\emph{Verfasser:}\\
Henrik \textsc{J�rgens} \\
Frederik \textsc{Strothmann} \\
\end{flushleft}
\end{minipage}
\hfill
\begin{minipage}{0.4\textwidth}
\begin{flushright} \large
\emph{Tutor:} \\
Stephan \textsc{Mingels} \\
\end{flushright}
\end{minipage}

\vspace{0.8cm}

\begin{table}[H]
\centering
\begin{tabular}{|l|}
\hline \textbf{Abstract: } \\
Das SQUID ist derzeit das sensibelste Messinstrument f�r Magnetfelder. Mit diesem k�nnen\\
Magnetfelder bis \SI{e-15}{T} gemessen werden. Das rf-SQUID besteht aus einer supraleitenden\\ 
Schleife mit eingebauter Schwachstelle, einem Flusstransformer und geeigneter Ausleseelektronik.\\
\hline 
\end{tabular}
\end{table} 

\vspace{0.8cm}

{
%  \fbox{
%    \begin{turn}{-90}
    \begin{tabular}{|l|c|l|c|}
    \hline
    Bereich               & max. \% & +/0/- &  erreicht  \% \\  \hline \hline
    Einleitung \& Theorie & 15     &       & \\                \hline
    Durchf�hrung        &        &       & \\
    Auswertung		      & 70     &       & \\ 
    phys. Diskussion      &        &       & \\ 
    Zusammenfassung       &        &       & \\ \hline
    Formales              & 15     &       & \\ \hline \hline
    Note			      & \multicolumn{3}{|c|}{\quad} \\ \hline
    \end{tabular}
    
%    \end{turn}
%  }
}

\vfill

% Unterer Teil der Seite/Datum
{\large \today}

\end{center}

\end{titlepage}