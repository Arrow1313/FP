\section{Versuchsdurchf�hrung und Auswertung}
%zuerst !alle! errechneten werte entweder in ganzen s�tzen aufz�hlen, oder in tabellen (�bersichtlicher) dargestellen, sowie auf die verwendeten formeln verweisen (die referenzierung der formel kann in der �berschrift stehen)
%kurz erw�hnen (vor der tabelle), warum wir das ganze ausrechnen bzw. was wir dort ausrechnen
%danach histogramme und plots erstellen, wobei wenn m�glich funktionen durch die plots gelegt werden (zur not k�nnen auch splines benutzt werden, was aber angegeben werden muss)
%bei fits immer die funktion und das reduzierte chiquadrat mit angegeben, wobei auf verst�ndlichkeit beim entziffern der zehnerpotenzen geachtet werden muss z.b. f(x)=(wert+-fehler)\cdot10^{irgendeine zahl}\cdot x + (wert+-fehler)\cdot10^{irgendeine zahl}
%bei jedem fit erkl�ren, nach welchem zusammenhang gefittet wurde und warum!
%bei plots darauf achten, dass die achsenbeschriftung (auch die tics) die richtige gr��e haben und die legende im plot nicht die messwerte verdeckt
%kurz die aufgabenstellung abgehandeln

\subsection{Beschichten der Glasprismen}
Bevor man die Oberfl�chenplasmonen nachweisen kann m�ssen die Glasprismen mit eine Goldschicht bedeckt werden. Vor der Beschichtung der Glasprismen m�ssen diese gereinigt werden. Falls noch eine Goldschicht auf den Prismen vorhanden ist, kann dies mit einem speziellem Tuch entfernt werden. Danach werden die Prismen in einem Ethanol und Aceton Bad mit Ultraschall gereinigt. Danach werden die Prismen noch unter eine Druck von 100 bar mit eisf�rmigen CO$_2$-Partikeln abgestrahlt um weiter Verunreinigungen zu entfernen.

Die Goldschichten sollen eine Dicke von 20nm bis 60nm haben. Als Beschichtungsmethode wird das Vakuumbeschichtungsverfahren der Kathodenzerst�ubung (engl. Sputtern) verwendet. Bei dem Prozess werden durch eine Glimmentladung zwischen zwei Elektroden freie Ionen erzeugt. Die freien Ionen werden dann �ber elektrischen Feld auf die Kathode beschleunigt. An der Kathode kommt es zu Sto�wechselwirkung, wodurch Kathodenmaterial emittiert wird. Das emittierte Kathodenmaterial setzt sich dann als d�nner Film auf Substrat ab. F�r den Versuch werden d�nne Platten aus Silber und Gold als Kathoden verwendet. Die Anode besteht aus einer geerdeten Halterung f�r die Glasprismen. 

Das Sputtering wird mit der Magnetron-Sputterapparatur Polaron SC7620 der Firma VG durchgef�hrt. Die Glasprismen werden jeweils in einen speziellen Beh�lter gelegt. �ber eine Membranvorpumpe und eine Turbomolekularpumpe wird der Druck in der Kammer auf unter 10$^{-5}$ mbar abgesenkt, dieser Prozess kann bis zu 15 min dauern. Danach wird �ber ein Feindosierventiel Argon in die Kammer gelassen, bis der Druck auf einige 10$^{-2}$ mbar angestiegen ist. Danach kann die Beschichtung gestartet werden, dabei liegt zwischen Anode und Kathode eine Spannung von 1000V an. Der Sputterstrom ergibt sich dann bei einem Druck von 2$\cdot$10$^{-2}$mbar mit ca. 20mA. Der Sputterstrom l�sst sich �ber den Druck regeln. Die Schichtdicke ist von Sputterstrom abh�ngig, bei eine Sputterstrom von 20mA ergibt sich nach einer Minute eine Schichtdicke von ca. 10nm. 

\subsection{Nachweis der Oberfl�chenplasmonen}
Nach dem Pr�parieren der Prismen, muss die Messapparatur noch justiert werden. 

Zuerst wird der Detektorarm justiert/seine Nulllage festgelegt. Daf�r wird der Laserstrahl auf die Si-Photodiode fokussiert und der Rotationstisch 1 solange gedreht, bis sich ein Maximum in der Intensit�t ergibt.

