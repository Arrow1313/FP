\section{Theorie}
% Es sollen die wichtigsten theoretischen Formeln und Zusammenh�nge einmal ausf�hrlich erkl�rt werden
Die theoretischen Grundlagen f�r den Versuch sollen kurz zusammengefasst werden.
\subsection{Anregung von Oberfl�chen-Plasma-Wellen}
Oberfl�chenplasmonen k�nnen durch Bestrahlung einer d�nnen Goldschicht mit Photonen erzeugt werden. Die zur Oberfl�che parallele Komponente des Wellenvektors kann, falls die Welle parallel zur Einfallsebene polarisiert ist, ein Oberfl�chenplasmon  anregen. Dabei muss die Energie und Impulserhaltung erf�llt sein. Das bedeutet, dass die zur Metalloberfl�che parallele Komponente des Photonwellenvektors ($\equiv k_{Photon}$) mit der Wellenzahl des Oberfl�chenplasmons ($\equiv k_{Plasmon}$) �bereinstimmen muss. Mit der Dispersionsrelation der Oberfl�chen-Plasma-Wellen ergibt sich eine Resonanzkurve mit einem Absorbtionsminimum bei:
\begin{align}
k_{Photon} = \frac{\omega}{c}n_{Prism}\sin(\alpha) = k_{Plasmon} = \frac{\omega}{c}\sqrt{\frac{\epsilon}{1+\epsilon}}
\end{align}
\begin{table}[H]
\begin{tabular}{l|l}
$k_{Photon}$ & zur Metalloberfl�che parallele Komponente des Photonwellenvektors \\
$k_{Plasmon}$  & Plasmonenwellenzahl \\
$n_{Prism}$ & Brechungsindex des verwendeten Glasprisma $n_{Prism} \equiv \sqrt{\epsilon_{Prism}}$ \\ 
$\alpha$ & Einfallswinkel \\
$\epsilon$ & von der Photonenfrequenz abh�ngige dielektrische Funktion des Metalls\\
$\omega$ & Photonenfrequenz
\end{tabular}
\end{table}
Damit das Minimum durch Variation des Einfallswinkels $\alpha$ eingestellt werden kann, muss $k_{Photon} \leq k_{Plasmon}$ gelten.
\subsection{Methode der abgeschw�chten Totalreflektion}
\subsection{Kathodenzerst�ubung (Sputtern)}