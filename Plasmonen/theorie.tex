\section{Theorie}
% Es sollen die wichtigsten theoretischen Formeln und Zusammenh�nge einmal ausf�hrlich erkl�rt werden
Die theoretischen Grundlagen f�r den Versuch sollen kurz zusammengefasst werden.
Zuerst wird auf die Anregung von Oberfl�chen-Plasma-Wellen eingegangen, als zweites wird die im Versuch verwendete Methode der abgeschw�chten Totalreflektion erkl�rt und zuletzt wird die Funktionsweise des Sputterns erl�utert.
\subsection{Anregung von Oberfl�chen-Plasma-Wellen}
Oberfl�chenplasmonen k�nnen durch Bestrahlung einer d�nnen Goldschicht mit Photonen erzeugt werden. Die zur Oberfl�che parallele Komponente des Wellenvektors kann, falls die Welle parallel zur Einfallsebene polarisiert ist, ein Oberfl�chenplasmon  anregen. Dabei muss die Energie und Impulserhaltung erf�llt sein. Das bedeutet, dass die zur Metalloberfl�che parallele Komponente des Photonwellenvektors ($\equiv k_{Photon}$) mit der Wellenzahl des Oberfl�chenplasmons ($\equiv k_{Plasmon}$) �bereinstimmen muss. Mit der Dispersionsrelation der Oberfl�chen-Plasma-Wellen ergibt sich eine Resonanzkurve mit einem Absorbtionsminimum bei:
\begin{align}
k_{Photon} = \frac{\omega}{c}n_{Prism}\sin(\alpha) = k_{Plasmon} = \frac{\omega}{c}\sqrt{\frac{\epsilon}{1+\epsilon}}
\label{eqn:Resonanzfall}
\end{align}
\begin{table}[H]
\begin{tabular}{l|l}
$k_{Photon}$ & zur Metalloberfl�che parallele Komponente des Photonwellenvektors \\
$k_{Plasmon}$  & Plasmonenwellenzahl \\
$n_{Prism}$ & Brechungsindex des verwendeten Glasprisma $n_{Prism} \equiv \sqrt{\epsilon_{Prism}}$ \\ 
$\alpha$ & Einfallswinkel \\
$\epsilon$ & von der Photonenfrequenz abh�ngige dielektrische Funktion des Metalls\\
$\omega$ & Photonenfrequenz
\end{tabular}
\end{table}
Damit das Minimum durch Variation des Einfallswinkels $\alpha$ eingestellt werden kann, muss $k_{Photon} \leq k_{Plasmon}$ gelten:
%Abbildung einf�gen
??
\subsection{Methode der abgeschw�chten Totalreflektion}
Die Methode der abgeschw�chten Totalreflexion wird zur Anregung der Oberfl�chenplasmonen verwendet. Dabei wird parallel (p) und senkrecht (s) zur Einfallsebene polarisiertes Laserlicht durch ein rechtwinkliges Glasprisma, auf dessen Grundfl�che eine Goldschicht aufgesputtert wurde, eingestrahlt. Entscheidend ist das an der Grenzschicht entstehende evaneszente Feld, welches bei Totalreflektion aufgrund der aus den Maxwellgleichungen folgenden Stetigkeitsbedingungen f�r das E-Feld entsteht. Die Amplitude des evaneszenten Feldes f�llt exponentiell mit der Eindringtiefe ab. Innerhalb der Eindringtiefe besteht die M�glichkeit der Wechselwirkung mit der aufgesputterten Goldschicht, sodass es m�glich ist, bei dem richtigen Einfallswinkel $\alpha$ Oberfl�chenplasmonen anzuregen. Da Oberfl�chenplasmonen Transversalwellen entlang der Oberfl�che sind, kann nur die parallel zur Einfallsebene polarisierte Komponente des Lichtes, sowie dessen k-Verktorkomponente parallel zur Oberfl�che des Materials, Oberfl�chenplasmonen anregen. Die Energie des s-polarisierten Lichtes str�mt daher ohne Energieverlust zur�ck. Die Intensit�t der p-polarisierten reflektierten Welle ($R_p$) wird durch das durch die s-polarisierte reflektierte Welle gegebene Referenzsignal ($R_s$) geteilt, um die Intensit�tsschwankungen, welche aus den Fresnelschen-Formeln folgen, zu ignorieren. Insgesamt ergibt sich bei Variation des Einfallswinkels $\alpha$ eine Resonanzkurve f�r $\frac{R_p}{R_s}(\alpha)$, welche ein Minimum im Resonanzfall, im Fall $k_{plasmon} = k_{photon}$ (\ref{eqn:Resonanzfall}), hat. Experimentell kann dieses Verfahren �ber zwei verschiedene Anordnungen realisiert werden, die Kretschmann- und die Otto-Konfiguration. Im Falle der Otto-Konfiguration ist das Metall durch einen Luftspalt in der Breite der Eindringtiefe $\lambda$ des evaneszenten Feldes von dem Glasprisma getrennt, sodass an der Metalloberfl�che Oberfl�chenplasmonen erzeugt werden k�nnen. F�r eine konsistente Messung muss darauf geachtet werden, dass der Abstand w�hrend der Messung nicht variiert. Im Falle der in unserem Versuch verwendeten Kretschmann-Konfiguration durchdringt das evaneszente-Feld die aufgesputterte Metallschicht und kann bei ausreichend d�nner Metallschicht an der Grenzschicht Metall/Luft Oberfl�chenplasmonen anregen.
\subsection{Kathodenzerst�ubung (Sputtern)}
Mit der Methode der Kathodenzerbest�ubung (engl. Sputtern) k�nnen d�nne Metallschichten auf Materialien aufgebracht werden, im Versuch auf Glasprismen. Die Beschichtung findet statt, indem Ionen eines Gases auf eine Targetmaterial beschleunigt werden, welche Atome aus dem Targetmaterial schlagen. Die herausgeschlagenen Atome werden dann mit einer Spannung auf das Prisma beschleunigt, auf welchem sie eine Schicht bilden. Daf�r wird eine Gleichspannung verwendet, dabei bildet das Targematerial die Kathode und die Halterung f�r das Prisma die Anode, da das Prisma nicht leitend ist. Zwischen den beiden Elektrode befindet sich das Prozessgas, in diesem Versuch Argon. Durch Sto�ionisation wird das Argon zu einem Plasma. Die Argon-Ionen werden dann auf das Targetmaterial beschleunigt und l�sen dort die Atome aus. Damit die Atome aus dem Targetmaterial auch das Prisma erreichen wird der Druck auf den Bereich von 10$^{-2}$mbar reduziert. Die Schichtdicke wird �ber den Sputterstrom reguliert. Der Sputterstrom h�ngt von dem Druck ab und kann dar�ber reguliert werden.