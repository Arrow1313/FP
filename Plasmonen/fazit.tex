\section{Fazit}
%im fazit nochmal alles zusammenfassen und den verlauf der messung absch�tzen
%gravierende sytematische probleme bei den messungen nochmal betonen und die wertigkeit unserer ergebnisse einordnen
In dem Versuch wurden Oberfl�chenplasmonen an Goldschichten auf Glasprismen untersucht. Daf�r wurde Glaspismen durch Supptering beschichtet, was gut gelang und es wurden 5 Schichten verschiedener Dicken erzeugt. Danach wurde die Messaperatur justiert, wobei die Nullage des Detektors, so wie die Position des Polarisationsdrehers f�r paralleles und senkrechtes Licht bestimmt wurden. Dann wurde noch der Brechungsindex des Prismas bestimmt, wodurch die sp�ter die Wellenzahl der Oberfl�chenplasmonen bestimmt werde konnte. Mit der aufgenommen Plasmonenresonanzkurve konnte dann die Wellenzahl der Oberfl�chenplasmonen bestimmt werden.