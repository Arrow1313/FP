\subsection{Diskussion}
%(immer) die gemessenen werte und die bestimmten werte �ber die messfehler mit literaturwerten oder untereinander vergleichen
%in welchem fehlerintervall des messwertes liegt der literaturwert oder der vergleichswert?
%wie ist der relative anteil des fehlers am messwert und damit die qualit�t unserer messung?
%in einem satz erkl�ren, wie gut unser fehler und damit unsere messung ist
%kurz erl�utern, wie systematische fehler unsere messung beeinflusst haben k�nnten
%(wichtig) zum schluss ansprechen, in wie weit die ergebnisse mit der theoretischen vorhersage �bereinstimmen
%--------------------------------------------------------------------------------------------
%falls tabellen mit den messwerten zu lang werden, kann die section mit den messwerten auch hinter der diskussion angef�gt bzw. eine section mit dem anhang eingef�gt werden.

Um die Diskussion konsistent zu halten, werden nur die Messdaten unserer Kommilitonen Diskutiert. 

Beim Sputtern wurden Prismen mit 5 verschiedenen Dicken Goldschichten erstellt 18nm, 27nm, 36nm, 45nm und 63nm. Bevor die Messung begonnen werde konnten, musste die Messapparatur justiert werden. Zuerst wurde die Position des Detektors justiert, dabei wurde eine Offset von 1,9(2) $^\circ$ bestimmt. Damit ergibt sich die Nulllage bei 181,9 $^\circ$. Bei der Bestimmung des Winkels, bei dem parallel bzw. senkrecht Polarisiertes Licht erzeugt wird, ergab sich ein Winkel von 89$^\circ$. Entsprechend ist der Winkel f�r senkrecht polarisiertes Licht -1$^\circ$. Der Brechungsindex des Prismas wurde �ber den Brewsterwinkel bestimmt, dabei ergab sich ein Wert von n$_{prisma}$=1.51. Nach der Justierung konnte die R$_p$/R$_s$-Kurve aufgenommen werden. Bei den allen Schichtdicken au�er der 63nm Schicht ist die Plasmonenresonanzkurve deutlich zu sehen. �ber eine Fit konnte der Resonanzwinkel bestimmt werden, aus welchem die Wellenzahl der Oberfl�chenplasmonen bestimmt werden konnte. Es ergab sich eine maximale Wellenzahl von 9,96(11) $\cdot$ 10$^6$ m$^{-1}$.

