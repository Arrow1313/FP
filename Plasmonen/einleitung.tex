\section{Einleitung}
%einleitung zu dem experiment.
%auf die einstellungen, die vor dem versuch gemacht werden, eingehen, oder auf eine anleitung dazu verweisen.
%---------------------------------------------------------------------------------------------
%hinter der einleitung kann der allgemeine theoretische hintergrund in einer zus�tzlichen section erkl�rt werden
Ziel des Versuches ist es Plasmaschwingungen an d�nnen gesputterten Metallschichten nachzuweisen. Plasmonen geh�ren zu den Quasiteilchen, mit denen sich in Festk�rpern verschiedene Ph�nomene beschreiben lassen. In diesem Fall steht das Plasmon f�r die Plasmaschwingung, genauer einen Plasmaschwingungsquant. Die Zerfalls/Abklingzeit eines Oberfl�chenplasmons liegt im Bereich weniger Femtosekunden. Das Plasmon hat daruch nur lokal Einfluss auf den Brechungsindex des Metalles, sodass sich zeitgleich an verschiedenen Stellen Plasmonen anregen lassen. Mit der Methode der abgeschw�chten Totalreflexion soll die Existenz des Oberfl�chenplasmons experimentell nachgewiesen und die Wellenzahl der Plasmaschwingung k$_{Plasmon}$ m�glichst genau bestimmt werden. Um Oberfl�chenplasmonen durch Einstrahlung von Photonen anzuregen, muss die Frequenz und Wellenzahl der einfallenden Photonen mit der des zu erzeugenden Plasmons �bereinstimmen. Durch Variation der entscheidenden Impulskomponente des Photons(, bei fester Photonfrequenz,) sollen Oberfl�chenplasmonen angeregt werden.