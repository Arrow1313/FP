\subsection{Kathodenzerst�ubung (Sputtern)}
Mit der Methode der Kathodenzerbest�ubung (engl. Sputtern) k�nnen d�nne Metallschichten auf Materialien aufgebracht werden, im Versuch auf Glasprismen. Die Beschichtung findet statt, indem Ionen eines Gases auf eine Targetmaterial beschleunigt werden, welche Atome aus dem Targetmaterial schlagen. Die herausgeschlagenen Atome werden dann mit einer Spannung auf das Prisma beschleunigt, auf welchem sie eine Schicht bilden. Daf�r wird eine Gleichspannung verwendet, dabei bildet das Targematerial die Kathode und die Halterung f�r das Prisma die Anode, da das Prisma nicht leitend ist. Zwischen den beiden Elektrode befindet sich das Prozessgas, in diesem Versuch Argon. Durch Sto�ionisation wird das Argon zu einem Plasma. Die Argon-Ionen werden dann auf das Targetmaterial beschleunigt und l�sen dort die Atome aus. Damit die Atome aus dem Targetmaterial auch das Prisma erreichen wird der Druck auf den Bereich von 10$^{-2}$mbar reduziert. Die Schichtdicke wird �ber den Sputterstrom reguliert. Der Sputterstrom h�ngt von dem Druck ab und kann dar�ber reguliert werden.