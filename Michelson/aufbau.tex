\section{Aufbau}
Es wird der Aufbau des Michelson-Interferometer beschrieben, ein Skizze des Aufbaus ist in Abbildung ?? zu sehen. Das Michelson-Interferometer besteht  aus einem Strahlteiler, der den eintreffenden Strahl in zwei Teilstrahlen teil. Die Teilstrahlen werde jeweils von einem Spiegel reflektiert. Einer der der beiden Spiegel kann �ber einen Hebelmechanismus, durch einem Elektromotor mit nahezu konstanter Geschwindigkeit bewegt werden. Der reflektierte Strahl wird durch eine CaF$_2$-Sammellinse auf eine pyroelektrischen Detektor geworfen. Der Detektor wandelt den Strahl in eine Spannung proportional zur Intensit�t des Strahls um. Die Spannung wird mit einem Lock-In-Verst�rker bearbeitet und �ber ein AD-Box an den Computer weitergeleitet. Am Computer wird aus dem Signal mit der Software Lab-View ein Interferogramm erstellt. 
Das �bersetzungsverh�ltnis k des Hebelmechanismus, f�r den Gangunterschied liegt bei ca. 5 (Quelle:??). Da der Teilstrahl, der auf den bewegten Spiegel trifft das Medium des Strahlteilers drei mal durchqueren muss, wird zwischen dem Strahlteiler und dem festem Spiegel noch ein Kompensationspl�ttchen angebracht.
Als Strahlenquelle wird ein HeNe-Laser und ein Muffelofen verwendet. Der Laser hat eine Wellenl�nge von \SI{3,39}{$\mu$m}, der Muffelofen und der Laser k�nnen �ber den Folienspiegel gekoppelt werden. Zwischen dem Muffelofen und dem Folienspiegel ist noch ein Chopper, zum Filtern bestimmter Wellenl�ngen eingebaut. Der Chopper zwischen 0-70 Hz filtern. Durch den Chopper registriert der Detektor immer die Temperatur- und Strahlungs�nderung (Untergrund) und die zu messende Strahlung (mit Untergrund) abwechselnd. Die Trennung der beiden Signal wird mittels eines phasenempfindlichen Log-In-Verst�rker und eines anhand Referenzsignals vorgenommen. Der Muffelofen kann bis 900$^\circ$C erhitzt werden, die dabei emittierte Strahlung durch eine Blende kollimiert. Es kann ein Schmalbandfilter verwendet werden, um die Wellenl�nge der Strahlung aus dem Muffelofen auf ca. \SI{3,31}{$\mu$m} zu begrenzen.