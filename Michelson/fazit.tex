\section{Fazit}
%im fazit nochmal alles zusammenfassen und den verlauf der messung absch�tzen
%gravierende sytematische probleme bei den messungen nochmal betonen und die wertigkeit unserer ergebnisse einordnen
Zu Beginn des Versuches wurde die Detektorspannung in Abh�ngigkeit der Chopperfrequenz bestimmt. Mit einem reduzierten Chiquadrat von 1,22 ist der lineare Fit an die logarithmierten Daten gut gelungen. F�r den Rest des Versuches wurde eine Chopperfrequenz von \SI{30}{Hz} verwendet. Im zweitem Teil sollte die Wellenl�nge des Lasers und das �bersetzungsverh�ltnis f�r die Millimeterschraube bestimmt werden. F�r den He-Ne-Laser wurde eine Wellenl�nge von \SI{3.732(9)}{$\mu$m} bestimmt, welche relativ um \SI{10.08}{\percent} von der erwarteten Wellenl�nge abweichte. Der Korrekturfaktor f�r das �bersetzungsverh�ltnis der Millimeterschraube wurde mit $k = \SI{5.42(1)}{}$ bestimmt. Bei der Bestimmung des Wei�lichtpunkts wurden unerwarteterweise zwei Wei�lichtpunkte gefunden. Der erste Wei�lichtpunkt wurde bei einer Schraubenposition von \SI{4.7545(80)}{mm} bestimmt, der zweite Punkt wurde bei \SI{5.770(3)}{mm} bestimmt. Nach der Theorie d�rfte es nur einen Wei�lichtpunkt geben. Die Herkunft des zweiten Wei�lichtpunktes ist nicht bekannt. Da auch nicht hinreichend Zeit vorhanden war, um den Aufbau weiter zu untersuchen, k�nnen keine Vermutungen gemacht werden. Im vorletzten Teil sollte der Schmalbandfilter untersucht werden. Daf�r wurde die Wellenl�nge aus dem Interferogramm bestimmt. F�r die Wellenl�nge ergab sich ein Wert von \SI{2.91}{$\mu$m}. Dieser weicht relativ um ca. 12\% von der erwarteten Wellenl�nge \SI{3,31}{$\mu$m} ab. Der bestimmte Wert konnte durch Analyse mit FFT best�tigt werden. Die Breite des e$^{-1}$-Abfalls des Wellenzahlspektrums wurde mit $\Delta \lambda = \SI{0,067(8)}{$\mu$ m}$ bestimmt und weicht um 12\% von der erwarteten Breite (0.060 $\mu$m) ab. Der theoretische Wert liegt innerhalb des Fehlers des experimentellen Wertes. Die halbe Breite des e$^{-1}$ Abfalls des Signals wurde mit $a= \SI{0,0129(1)}{mm}$ bestimmt. Die Spektrale Verteilung des Schmalbandfilters ist in Abbildung \ref{fig:spektrum_filter} zu sehen. Im letztem Versuchsteil sollte die Eigenschaft der Schwebung untersucht werden. Mittels FFT konnten die beiden Wellenl�ngen 3.392(12) $\mu$m und 3.31(2) $\mu$m aus den Messdaten bestimmt werden. Diese entsprechen den Wellenl�ngen des Lasers und des Muffelofens. Bei der manuellen Bestimmung der Schwebungswellenl�nge ergab sich ein Wert von 275.6(3) $\mu$m. Die aus den Messdaten bestimmte spektrale Verteilung der Schwebung ist in Abbildung \ref{fig:spektrum_schwebung} zu sehen.