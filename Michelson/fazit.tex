\section{Fazit}
%im fazit nochmal alles zusammenfassen und den verlauf der messung absch�tzen
%gravierende sytematische probleme bei den messungen nochmal betonen und die wertigkeit unserer ergebnisse einordnen
Zu Beginn des Versuches wurde die Detektorspannung in Abh�ngigkeit der Chopperfrequenz bestimmt. Mit einem reduzierten Chiquadrat von 1,22 ist der lineare Fit an die logarithmierten Daten gut gelungen. F�r den Rest des Versuches wurde eine Chopperfrequenz von \SI{30}{Hz} verwendet. Im zweitem Teil sollte die Wellenl�nge des Lasers und das �bersetzungsverh�ltnis f�r die mm-Schraube. F�r den HeNe-Lasers wurde eine Wellenl�nge von 3.732(9) $\mu$m bestimmt und weicht relativ um 10.08\%. Der Korrekturfaktor f�r das �bersetzungsverh�ltnis der mm-Schraube wurde mit 5.42(1) bestimmt. Bei der Bestimmung des Wei�lichtpunkts wurden unerwarteter weise zwei Wei�lichtpunkte gefunden. Der erste Wei�lichtpunkt wurde bei einer Schraubenposition von 4.7545(80) mm bestimmt und der zweite Punkt wurde bei 5.770(3) mm bestimmt. Nach der Theorie d�rfte es nur einen Wei�lichtpunkt geben. Die Herkunft des zweiten Wei�lichtpunktes ist nicht bekannt. Da auch nicht hinreichend Zeit vorhanden war um den Umstand weiter zu untersuchen, k�nnen auch keine Vermutungen gemacht werden. Im Vorletztem Teil sollte der Schmalbandfilter untersucht werden. Daf�r wurde die Wellenl�nge aus dem Interferogramm bestimmt. F�r die Wellenl�nge ergab sich ein Wert von 2.91 $\mu$m, dieser Weicht relativ um 12\% ab. Der bestimmte Wert konnte durch Analyse mit FFT best�tigt werden. Die Breite des e$^{-1}$-Abfalls wurde mit 0.067(8) $\mu$m bestimmt und weicht um 12\% von der erwarteten Breite (0.060 $\mu$m) ab. Der experimentelle Wert liegt mit dem Fehler immer noch innerhalb des theoretischen Wertes. Die Spektrale Verteilung des Schmalbandfilters ist in Abbildung \ref{fig:spektrum_filter} zu sehen. Im letztem Versuchsteil sollte die Eigenschaft der Schwebung untersucht werden. Mittels FFT konnten die beiden Wellenl�ngen 3.392(12) $\mu$m und 3.31(2) $\mu$m aus den Messdaten bestimmt werden. Diese entsprechen den Wellenl�ngen des Lasers und des Muffelofen. Bei der Bestimmung der Wellenl�nge der Schwebung ergab sich ein Wert von 275.6(3) $\mu$m. Die aus den Messdaten bestimmte spektrale Verteilung der Schwebung ist in Abbildung \ref{fig:spektrum_schwebung} zu sehen.