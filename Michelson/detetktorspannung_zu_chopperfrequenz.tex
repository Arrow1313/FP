\section{Detektorspannung in Abh�ngigkeit der Chopperfrequenz}

Zu erste soll die der Zusammenhang zwischen der Detektorspannung U und der Chopperfrequenz f untersucht werden. Die Chopperfrequenz wird im Bereich von 20Hz bis 50Hz variiert. Bei Werten unter 20Hz ist die Schwankung der Werte zu gro�, so dass die Signale nicht vom Untergrund unterschieden werden k�nnen. Die Detektorspannung wird am Log-In-Verst�rker abgelesen, dabei wurde ein Fehler von ?? V angenommen, da die Anzeige in diesem Bereich schwankte. Der Plot der Messdaten mit linearen Achsen ist in Abbildung ?? zu sehen. In Abbildung ?? sind die Messdaten mit doppelt logarithmischen Achsen geplotted. Da ein linearer Zusammenhang zwischen log(U) und log(f) zu sehen ist, werden die Messdaten mit  Gleichung ?? gefitted.

\begin{align}
U(f) = exp[A \cdot log(f) + B]
\end{align}

Dabei sind A und B freie Parameter, die durch den Fit bestimmt werden. Das Ergebnis des Fits ist in Tabelle ?? zu sehen.
