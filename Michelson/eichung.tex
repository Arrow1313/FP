\section{Eichung und Bestimmung der Wellenl�nge des Lasers}
In diesem Abschnitt soll das �bersetzungsverh�ltnis zwischen Schraube und Spiegel und die Wellenl�nge des Lasers bestimmt werden. Zuvor muss noch die x-Position in Lab-View und der Millimeterschraube geeicht werden.

\subsection{Eichung}
\label{Eichung}
Die x-Werte in Lab-View m�ssen den realen Positionen der Millimeterschraube zugeordnet werden. Bei der Bestimmung wird von einem linearem Zusammenhang ausgegangen. F�r die Eichung wird die Schraube ??mm gedreht, dabei wird alle $\frac{1}{10}$mm die Position im Interferogramm mit Lab-View aufgenommen. Der Fehler der Schraubenposition wird mit 0,5$\mu$m angenommen (eine halbe Skaleneinheit). F�r den Fit wurde Gleichung \ref{eqn:lin_fit} verwendet, A beschreibt dabei die Steigung und B das Offset.

\begin{align}
\label{eqn:lin_fit}
f(x) = A \cdot x + B
\end{align}

%Fit einf�gen

Aus dem Fit ergebe sich die Parameter in Tabelle ??.

\subsection{Bestimmund der Wellenl�nge des Lasers}
Mit der zuvor durchgef�hrten Eichung kann nun die Wellenl�nge $\lambda$ des Lasers bestimmt werden. Die Wellenl�nge wird aus dem Gangunterschied s, der Anzahl der Interferenzmaxima n und dem �bersetzungsverh�ltnis k bestimmt. F�r das �bersetzungsverh�ltnis wird ein Wert von k=5 angenommen. Die Wellenl�nge ergibt sich nach Gleichung ??.

\begin{align}
\lambda = \frac{2s}{5n}
\end{align}

Das aufgenommen Interferogramm ist in Abbildung ?? zu sehen.

%Interferogramm einf�gen

F�r den Gangunterschied wird \framebox{Auswahl der Maxima/Minima begr�nden} . Mit der zuvor durchgef�hrten Eichung wird der Gangunterschied der x-Position in eine reelle Wegdifferenz umgerechnet, der Fehler ergibt sich nach Fehlerfortpflanzung auf die Geradengleichung. Die Anzahl der Maxima wurde als Fehlerlos angenommen. Um ein besseres Ergebnis zu erhalten, wird das eigentliche �bersetzungsverh�ltnis $k_e$ mit Gleichung \ref{eqn:uebersetzungsverhaeltniss} bestimmt. Dabei wurde f�r $\lambda$ der Literaturwert von 3,39$\mu$m verwendet (vlg ??).

\begin{align}
\label{eqn:uebersetzungsverhaeltniss}
k_e = \frac{2s}{n\lambda}
\end{align}

Die Ergebnisse f�r ersten 10 \framebox{Minima/Maxima} sind in Tabelle ?? zu sehen. Dabei ergab sich ein Mittelwert von ?? f�r die Wellenl�nge des Lasers. \framebox{Auswertung und Diskussion}