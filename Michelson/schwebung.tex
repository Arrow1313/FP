\section{Analyse von Schwebungsinterferogrammen}
Eine Schwebung tritt bei der �berlagerung von zwei Schwingungen mit fast gleicher Frequenz auf (vlg. Abschnitt ??). Im Experiment ist diese Bedingung durch den Laser mit einer Wellenl�nge von $\lambda_L$ = ?? und einer Wellenl�nge des Muffelofen von $\lambda_M$ = ?? gegeben. Um die Wellenl�nge des Lasers an die des Muffelofen anzupassen, werden Polythylenfolien verwendet, welche in den Strahlengang gesetzt werden. Zur Feinjustierung k�nnen die Folien noch gedreht werden.
Nach dem anpassen der Amplituden der Strahler wird das Interferogramm aufgenommen. Das Schwebungsinterferogramm ist in Abbildung ?? zu sehen.