\newpage
\section{Fazit}
%im fazit nochmal alles zusammenfassen und den verlauf der messung absch�tzen
%gravierende sytematische probleme bei den messungen nochmal betonen und die wertigkeit unserer ergebnisse einordnen
Aus dem Diffraktogramm des vorgegebenen Pulvers konnte dessen Zusammensetzung mit gro�er Sicherheit abgelesen werden, sodass sogar die Werte $[nh,nk,nl]$ exakt bestimmt werden konnten. Es handelte sich um Siliciumpulver. Die relative Abweichung der nur �ber die Messung bestimmten Netzebenenabst�nde von den aus den Werten $[nh,nk,nl]$ und dem Literaturwert f�r die Gitterkonstante von Silicium bestimmten Netzebenenabst�nden ist �berall geringer als ein Zweihundertstel bzw. ein halbes Prozent. Daneben wurde eine m�gliche Begr�ndung f�r den in den Messdaten gefundenen n�herungsweise konstanten Offset der Maxima der Peaks zu den simulierten Maxima gefunden. Die Messdaten sind im Mittel um 0,1712 $^{\circ}$ nach links verschoben. Zuletzt konnte �ber die Scherrer Gleichung eine grobe Absch�tzung f�r die Korngr��e gemacht werden, sodass die Korngr��e mit \SI{38}{\angstrom} bis \SI{82}{\angstrom} abgesch�tzt werden konnte. Da die Gitterkonstante von Silicium bei \SI{5,431}{\angstrom} liegt, untersch�tzt dieser Wert die wirkliche Korngr��e mit gro�er Warscheinlichkeit. 