\newpage
\section{Fazit}
%im fazit nochmal alles zusammenfassen und den verlauf der messung absch�tzen
%gravierende sytematische probleme bei den messungen nochmal betonen und die wertigkeit unserer ergebnisse einordnen
Aus den Diffraktorammen des Si111-Einkristalls konnte die Energie der $K_{\alpha_{1,2}}$- und $K_\beta$-Linie bestimmt werden. Dabei wurde f�r die $K_{\alpha_1}$-Linie eine Energie von 8052 $\pm$ 5 eV, f�r die $K_{\alpha_2}$-Linie eine Energie von 8032 $\pm$ 5 und f�r die $K_\beta$-Linie wurde eine Energie von 8910 $\pm$ 6 eV bestimmt. Alle bestimmten Energien haben eine relative Abweichung von 0,1\%, was ein gutes Ergebnis ist. Bei der Messung mit Ni-Filter wurde das Signal-Rausch-Verh�ltnis bestimmt, dabei zeigt sich, dass bei h�heren Ordnungen das Signal-Rausch-Verh�ltnis schlechter wird. Bei der Bestimmung der Netzebenenabst�nde von Si311 und Ge111 wurde der Netzebenenabstand von Si11 mit 1,24603 $\pm$ 0,00006 \text{\AA } bestimmt. Der Netzebenenabstand von Ge111 wurde mit 3,27474 $\pm$ 0,00005\text{\AA } bestimmt. Die bestimmten Netzebenabst�nde weichen relativ um 0,01\% und 0,03\% ab, was ein gutes Ergebnis ist.
Aus dem Diffraktogramm des vorgegebenen Pulvers konnte dessen Zusammensetzung mit gro�er Sicherheit abgelesen werden, sodass sogar die Werte $[nh,nk,nl]$ exakt bestimmt werden konnten. Es handelte sich um Siliciumpulver. Die relative Abweichung der nur �ber die Messung bestimmten Netzebenenabst�nde von den aus den Werten $[nh,nk,nl]$ und dem Literaturwert f�r die Gitterkonstante von Silicium bestimmten Netzebenenabst�nden ist �berall geringer als ein Zweihundertstel bzw. ein halbes Prozent. Daneben wurde eine m�gliche Begr�ndung f�r den in den Messdaten gefundenen n�herungsweise konstanten Offset der Maxima der Peaks zu den simulierten Maxima gefunden. Die Messdaten sind im Mittel um 0,1712 $^{\circ}$ nach links verschoben. Zuletzt konnte �ber die Scherrer Gleichung eine grobe Absch�tzung f�r die Korngr��e gemacht werden, sodass die Korngr��e mit \SI{38}{\angstrom} bis \SI{82}{\angstrom} abgesch�tzt werden konnte. Da die Gitterkonstante von Silicium bei \SI{5,431}{\angstrom} liegt, untersch�tzt dieser Wert die wirkliche Korngr��e mit gro�er Warscheinlichkeit. 