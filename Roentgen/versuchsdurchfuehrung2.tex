\subsection{Versuchsdurchf�hrung}
Um die Zusammensetzung einer unbekannten Pulverprobe zu bestimmen, werden mit dem Diffraktometer Diffraktogramme erstellt, aus denen mittels einer Datenbank qualitativ die Probenzusammensetzung ermittelt werden kann. Ebenfalls sollen unabh�ngig von der Auswertung mithilfe der Datenbank einige Netzebenenabst�nde d aus den Diffraktogrammen manuell bestimmt werden. Dies geschieht nach der Braggschen Gleichung analog zum Versuchsteil 2.1.3 Formel ??. Nachdem graphisch gezeigt wurde, dass die gefundene Kristallstruktur mit den Diffraktogrammen vertr�glich ist, soll aus den Daten eine Absch�tzung f�r die mittlere Kristallitgr��e gemacht werden, welche aus der Scherrergleichung bestimmt werden kann.
\begin{align}
\delta(2\Theta)_{Korn} = \frac{K \lambda }{B cos{\Theta_0}}
\end{align}
Dabei ist $\delta(2\Theta)$ die volle Halbwertsbreite (FWHM) des Reflexes im Bogenma�, $\Theta_0$ das Maximum des Reflexes, $K$ der Scherrer Formfaktor mit $K \approx 0.89$ und B die gesuchte Korngr��e.
Es ergibt sich also F�r die Korngr��e B, wenn man beachtet, dass $\delta(2\Theta)_{Korn} = \delta(2\Theta)_{Pulver} - \delta(2\Theta)_{Einkristall} := $ und $E = \frac{hc}{\lambda}$ gilt:
\begin{align}
B = \frac{0.89 hc}{(\delta(2\Theta)_{Pulver} - \delta(2\Theta)_{Einkristall})Ecos{\Theta_0}}
\end{align}
Es ergibt sich ein Fehler von:
\begin{align}
\Delta B = B\sqrt{\left(\frac{\Delta (\delta(2\Theta)_{Pulver} - \delta(2\Theta)_{Einkristall})}{\delta(2\Theta)_{Pulver} - \delta(2\Theta)_{Einkristall}}\right)^2+\left(\frac{\Delta E}{E}\right)^2+\left(\frac{\Delta \Theta_0}{\cot{\Theta_0}}\right)^2}
\end{align}