\subsection{Versuchsdurchf�hrung}
Um die Zusammensetzung einer unbekannten Pulverprobe zu bestimmen, werden mit dem Diffraktometer Diffraktogramme erstellt, aus denen mittels einer Datenbank qualitativ die Probenzusammensetzung ermittelt werden kann. Ebenfalls sollen unabh�ngig von der Auswertung mithilfe der Datenbank einige Netzebenenabst�nde d aus den Diffraktogrammen manuell bestimmt werden. Dies geschieht nach der Braggschen Gleichung analog zum Versuchsteil 3.1.2 Formel \ref{eqn:netzebenen}. Nachdem graphisch gezeigt wurde, dass die gefundene Kristallstruktur mit den Diffraktogrammen vertr�glich ist, soll aus den Daten eine Absch�tzung f�r die mittlere Kristallitgr��e gemacht werden, welche aus der Scherrergleichung bestimmt werden kann.
\begin{align}
\delta(2\Theta)_{Korn} = \frac{K \lambda }{B cos{\Theta_0}}
\end{align}
Dabei ist $\delta(2\Theta)$ die volle Halbwertsbreite (FWHM) des Reflexes im Bogenma�, $\Theta_0$ das Maximum des Reflexes, $K$ der Scherrer Formfaktor mit $K \approx 0.94$ (vgl. \cite{scherrer}) und B die gesuchte Korngr��e.
Es ergibt sich f�r die Korngr��e B, wenn man beachtet, dass $E = \frac{hc}{\lambda}$ gilt:
\begin{align}
\label{eqn:scherrer_korngroesse}
B = \frac{0.94 hc}{\delta(2\Theta)_{Korn}Ecos{\Theta_0}}
\end{align}