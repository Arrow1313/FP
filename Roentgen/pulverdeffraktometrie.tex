\section{Pulverdiffraktometrie}
%kurz das ziel dieses versuchsteiles ansprechen, damit keine zwei überschriften direkt übereinander stehen!
%bei schwierigeren versuchen kann auch der theoretische hintergrund erläutert werden. (mit formeln, herleitungen und erklärungen)
Nun soll mit der zuvor verwendeten Methode die Zusammensetzung unbekannter Pulverproben bestimmt werden. Aus den bestimmten Diffraktogrammen soll mittels einer Datenbank die Zusammensetzung bestimmt, so wie die Netzebenabst\"ande berechnet werden. Graphisch soll auch die Vertr\"aglichkeit der gefundenen Kristallstruktur mit dem Diffraktogramm gezeigt werden und die mittlere Kristallgr\"o\ss e ermittelt werden.
