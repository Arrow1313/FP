\section{Pulverdiffraktometrie}
%kurz das ziel dieses versuchsteiles ansprechen, damit keine zwei überschriften direkt übereinander stehen!
%bei schwierigeren versuchen kann auch der theoretische hintergrund erläutert werden. (mit formeln, herleitungen und erklärungen)
Nun soll mit der zuvor verwendeten Methode die Zusammensetzung unbekannter Pulverproben bestimmt werden. Aus den bestimmten Diffraktogrammen soll mittels einer Datenbank die Zusammensetzung bestimmt, so wie die Netzebenabst\"ande berechnet werden. Graphisch soll auch die Vertr\"aglichkeit der gefundenen Kristallstruktur mit dem Diffraktogramm gezeigt werden und die mittlere Kristallgr\"o\ss e ermittelt werden.
\subsection{Verwendete Materialien}
%(immer) eine skizze oder ein foto einfügen, die geräte/materialien !nummerieren! und z.b. eine legende dazu schreiben
%falls am anfang des versuches nicht klar ist, was alles verwendet wird, wenn möglich erst am ende ein großes foto von den verwendeten materialien machen!
\subsection{Versuchsaufbau}
%skizze zum versuchsaufbau (oder foto) einfügen,   es muss erklärt werden wie das ganze funktioniert und welche speziellen einstellungen verwendet wurden (z.b. welche knöpfe an den geräten für die messung verdreht wurden)