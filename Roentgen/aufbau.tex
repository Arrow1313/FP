%Aubau
\section{Aufbau}
F\"ur die Spektroskopie wird ein Vollschutz-Diffraktometer "X'Pert" der Firma Philips verwendet.
Vorteile des Diffraktometers sind der modulare Aufbau, welcher ein flexibles und einfaches handhaben erm\"oglicht, so wie die hohe Messgenauigkeit.
Das Diffraktometer wird \"uber einen \"uber eine Computer gesteuert, an welchem die Beschleunigungsspannung in einem Bereich von \unit[15]{keV} bis \unit[50]{keV} eingestellt werden kann, bei einem Strom von \unit[5]{mA} bis \unit[36]{mA}.
Der Winkelbereich liegt bei 8$^\circ$ bis 140$^\circ$.
Nach dem einstellen der Beschleunigungsspannung, des Stroms und des Winkelbereichs verl\"aft die Messung vollautomatisch.
Die Messdaten werden automatisch digitalisiert und auf dem Computer gespeichert.
\\
??? \\
Bild und kurze beschreibung noch hizuf\"ugen.  \linebreak
??? \linebreak

Eine schematische Skizze des Strahlengans im Deffraktometer ist in Abb ?? zu sehen.
 