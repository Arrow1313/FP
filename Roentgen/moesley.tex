\subsection{Charakteristische R�ntgenstrahlung}
Die charakteristische R�ntgenstrahlung l�sst sich mit dem Moesleyschem Gesetz beschreiben. Das Moesleysche Gesetz h�ngt nur von der Ordnungszahl des Materials und den Schalen des �bergangs ab, es beschreibt die Energie ausgesandter Photonen bei �berg�ngen von Elektronen.
Das Moesleysche-Gesetz lautet:

\begin{align}
f = f_R \cdot Z_{eff}^2 \cdot \left( \frac{1}{n_1^2} - \frac{1}{n_2^2} \right)
\end{align}

$Z_{eff}$ ist die effektive Kernlandung welche durch

\begin{align}
\label{eqn:z_eff}
Z_{eff} = Z - S
\end{align}

gegeben ist, wobei Z die Kernladung und S die Abschirmungskonstante ist.
$f_R$ die angepasste Rydberg-Frequenz, sie h�ngt von der Rydbergfrequenz (R) und der Kernmasse (Z) ab (Gl. \ref{eqn:r_f}).

\begin{align}
\label{eqn:r_f}
f_R = R \frac{1}{1 + \frac{m_e}{Z}}
\end{align}

Mit dieser Gleichung und Gleichung \ref{eqn:energie} l�sst sich nun die Energie der K$_\alpha$ und K$_\beta$ R�ntgenphotonen bestimmten, dabei wird f�r den K$_\alpha$ �bergang S mit 1 angenommen und f�r K$_\beta$ S = 1,8 angenommen.

\begin{align}
\label{eqn:energie}
E = h \cdot f
\end{align}