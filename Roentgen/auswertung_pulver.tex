\subsection{Auswertung}
Bei der Analyse des vorgegebenen Pulvers mittels Debye Scherrer Verfahren ergab sich das Diffraktogramm in Abb. ??. Zum Vergleich sind Diffraktogramme von Silicium und Germanium (Abb. ??) simuliert worden. Man sieht sofort, dass das Diffraktogramm von Silicium mit dem der untersuchten Probe sehr gut �bereinstimmt. Daneben stellt man einen Offset der simulierten Daten zu den gemessenen fest. Um diesen Offset zu bestimmen, wurde an alle drei Datens�tze ein Multivoigt gefittet. Die Voigtverteilung wird dabei numerisch approximiert, wobei die in Python bereits implementierte Voigt-Verteilung aus der Bibliothek "`lmfit"' verwendet wird. Der Fit an die Messdaten passt mit einem reduzierten Chiquadrat von 9,419 relativ gut, wenn man beachtet, dass auch bei kleineren Zahlraten ein Fehler von $\sqrt{N}$ verwendet wurde. Erstaunlicherweise passen die Fits, bei einem Fehler von $\sqrt{N}$, eher schlecht an die simulierten Daten. Man sieht aber, dass die Maxima trotzdem gut getroffen werden, was in diesem Versuchsteil das wichtigste Kriterium f�r die Auswertung ist. Es ergeben sich reduzierte Chiquadrate von 500 bis 20000, sodass diese Fits nicht als beseondes gut betrachtet werden k�nnen. Wie die Fits in der N�he einzelner Peaks aussehen, kann im Anhang anhand von Beispielen nachvollzogen werden. Alle 66 Peaks zu zeigen w�rde den Rahmen dieses Protokolls sprengen, wobei diese in unserem Repository zu finden sind unter den Abbildungen mit dem Zusatz \_pulver und \_simu (siehe \cite{git_repo}).