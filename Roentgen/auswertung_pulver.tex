\subsection{Auswertung}
Bei der Analyse des vorgegebenen Pulvers mittels Debye Scherrer Verfahren ergab sich das Diffraktogramm in Abb. ??. Zum Vergleich sind Diffraktogramme von Silicium und Germanium (Abb. ??) simuliert worden. Man sieht sofort, dass das Diffraktogramm von Silicium mit dem der untersuchten Probe sehr gut übereinstimmt. Daneben stellt man einen Offset der simulierten Daten zu den gemessenen fest. Um diesen Offset zu bestimmen, wurde an alle drei Datensätze ein Multivoigt gefittet. Die Voigtverteilung wird dabei numerisch approximiert, wobei die in Python bereits implementierte Voigt-Verteilung aus der Bibliothek "`lmfit"' verwendet wird.