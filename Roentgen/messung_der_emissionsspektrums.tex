\section{Messung des Emissionsspektrums}
%kurz das ziel dieses versuchsteiles ansprechen, damit keine zwei �berschriften direkt �bereinander stehen!
%bei schwierigeren versuchen kann auch der theoretische hintergrund erl�utert werden. (mit formeln, herleitungen und erkl�rungen)
Im ersten Versuchsteil wird das R�ntgenspektrum der Kupferanode mit einem Silicium(111)-Einkristall untersucht, f�r die Untersuchung werden drei verschieden Beschleunigungsspannungen und ein Ni-Filter verwendet. Untersucht werden die Z�hlraten in Abh�ngigkeit des Winkels, sowie die Lage aller Ordnungen der K$_{\alpha_{1,2}}$- und K$_\beta$-Linien von Kupfer und deren Verh�ltnisse. Dann wird noch das Signal-Rausch-Verh�ltnis untersucht und weitere Details der Spektren besprochen.
\subsection{Verwendete Materialien}
%(immer) eine skizze oder ein foto einf�gen, die ger�te/materialien !nummerieren! und z.b. eine legende dazu schreiben
%falls am anfang des versuches nicht klar ist, was alles verwendet wird, wenn m�glich erst am ende ein gro�es foto von den verwendeten materialien machen!
\subsection{Versuchsaufbau}
%skizze zum versuchsaufbau (oder foto) einf�gen,   es muss erkl�rt werden wie das ganze funktioniert und welche speziellen einstellungen verwendet wurden (z.b. welche kn�pfe an den ger�ten f�r die messung verdreht wurden)