\section{Messung des Emissionsspektrums von Kupfer und der Netzebenenabst�nde von Si(331) und Ge(111)}
%kurz das ziel dieses versuchsteiles ansprechen, damit keine zwei �berschriften direkt �bereinander stehen!
%bei schwierigeren versuchen kann auch der theoretische hintergrund erl�utert werden. (mit formeln, herleitungen und erkl�rungen)
Im ersten Versuchsteil wird das R�ntgenspektrum der Kupferanode mit einem Silicium(111)-Einkristall untersucht, f�r die Untersuchung werden drei verschieden Beschleunigungsspannungen und ein Ni-Filter verwendet. Untersucht werden die Z�hlraten in Abh�ngigkeit des Winkels, sowie die Lage aller Ordnungen der K$_{\alpha_{1,2}}$- und K$_\beta$-Linien von Kupfer und deren Verh�ltnisse. Dann wird das Signal-Rausch-Verh�ltnis untersucht und weitere Details der Spektren besprochen. Im Anschluss werden die Netzebenabst�nde anderer Einkristalle untersucht. Untersucht werden Si(331)- und Ge(111)-Einkristalle. Die bestimmten Netzebenabst�nde werden mit Literaturwerten abgeglichen.

%Versuchsdurchf�hrung
\subsection{Versuchsdurchf�hrung}
%erkl�ren, !was! wir machen, !warum! wir das machen und mit welchem ziel
%(wichtig) pr�zize erkl�ren, wie bei dem versuch vorgegangen und was gemacht wurde
\subsection{Hochspannung der Photomultiplier}
Alle vier Photomuliplier werden mit einer Hochspannung versorgt, PM1, PM2 und PM4 sollen dabei den Maximalwert von 2100V nicht �berschreiten, da es sonst zu Besch�digungen kommen kann. Der Maximalwert f�r PM3 liegt bei 2700V. Die Hochspannung der einzelnen Photomuliplier soll so eingestellt werden, dass PM1, PM2 und PM4 m�glichstdie selben Z�hlraten liefern. PM3 wurde mit der maximal m�glichen Spannung von 2700V betrieben. Es wurde die Spannungen in Tabelle \ref{tab:hochspannung} verwendet.

\begin{table}[H]
\centering
\caption{Verwendete Spannungen f�r die Photomuliplier}
\label{tab:hochspannung}
\begin{tabular}{|c|c|}
\hline Photomultiplier & Spannung[V] \\ \hline
\hline PM1 &  \\ 
\hline PM2 &  \\ 
\hline PM3 &  \\ 
\hline PM4 &  \\ 
\hline 
\end{tabular} 
\end{table}
\subsection{Verwendete Formeln}
%eine legende kann angefertigt werden, die selbstverst�ndlichen buchstaben m�ssen nicht extra erkl�rt werden
%mit knappen erkl�rungen die !verwendeten! formeln, sowie die zugeh�rige fehlerrechnung einf�gen.
%Auswertung des Emissionsspektrums
\subsection{Auswertung}
Es soll das Röntgenspektrum der Kupferanode bei drei unterschiedlichen Beschleunigungsspannungen und ein mit einem Ni-Filter, bei möglichst hoher Röntgenspannung untersucht werden.
Zum beugen der Röntgenstrahlen wird ein Si(111)-Einkristall verwendet, der Netzebenabstand liegt bei \SI{3,1356}{\angstrom}, entnommen von \cite{si_a}.
Gescannt wird ein Winkelbereich von ??$^\circ$ bis ??$^\circ$, dabei wurden für die Beschleunigungsspannung Werte von
\begin{itemize}
\item U = ?? und A = ??
\item U = ?? und A = ??
\item U = ?? und A = ??
\end{itemize}
 verwendet. Dabei ergeben sich die folgenden Plots.
 
 % Plots der Messunge für die Kupferanode mit unterschiedlichen Beschleunigungsspannungen
 
 
 Es ist deutlich zu erkennen, das bei steigender Beschleunigungsspannung und Strom die Counts größer werden und so die Peaks deutlich von dem Untergrund zu unterscheiden sind.

