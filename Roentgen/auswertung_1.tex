%Auswertung des Emissionsspektrums
\subsection{Auswertung}
Es soll das R�ntgenspektrum der Kupferanode bei drei unterschiedlichen Beschleunigungsspannungen und mit einem Ni-Filter, bei m�glichst hoher R�ntgenspannung untersucht werden.
Zum beugen der R�ntgenstrahlen wird ein Si(111)-Einkristall verwendet, der Netzebenabstand liegt bei \SI{3,1356}{\angstrom}, entnommen von \cite{si_a}.
Gescannt wird ein Winkelbereich von ??$^\circ$ bis ??$^\circ$, dabei wurden f�r die Beschleunigungsspannung Werte von
\begin{itemize}
\item U = ?? und A = ??
\item U = ?? und A = ??
\item U = ?? und A = ??
\end{itemize}
 verwendet. Dabei ergeben sich die folgenden Plots.
 
 % Plots der Messunge f\"ur die Kupferanode mit unterschiedlichen Beschleunigungsspannungen
 
 
Es ist deutlich zu erkennen, das bei steigender Beschleunigungsspannung und Strom die Counts gr��er werden und so die Peaks deutlich von dem Untergrund zu unterscheiden sind.
Um Informationen �ber den Winkel und die Intensit�t zu erhalten, werden die Peaks mit der Gau� verteilung gefittet.
Die Energie wird nach Gleichung ?? bestimmt, die Counts ergeben sich mit Gleichung ??. Der Fehler wird mit Gleichung ?? bestimmt.

Es ergeben sich die folgenden Ergebnisse:

\begin{table}[H]
\caption{In der Tabelle sind die Messdaten und die daraus bestimmten Energien f�r die unterschiedlichen Cu-Linien und Ordnungen}
\label{tab:ergebnisse_1}
\centering
\begin{tabular}{|c|c|c|c|c|c|c|c|c|}
\hline Cu-Linie & Ordung & 2$\theta^\circ$ & $\Delta$2$\theta^\circ$ & Counts & $\Delta$Counts & E$_{exp}$[eV] & $\Delta$E$_{exp}$[eV] & E$_{literatur}$ \\ 
\hline K$_\beta$ & 1 &  &  &  &  &  &  &  \\ 
\hline K$_{\alpha_1}$ & 1 &  &  &  &  &  &  &  \\ 
\hline K$_{\alpha_2}$ & 1 &  &  &  &  &  &  &  \\ 
\hline K$_\beta$ & 3 &  &  &  &  &  &  &  \\ 
\hline K$_{\alpha_1}$ & 3 &  &  &  &  &  &  &  \\ 
\hline K$_{\alpha_2}$ & 3 &  &  &  &  &  &  &  \\ 
\hline K$_\beta$ & 4 &  &  &  &  &  &  &  \\ 
\hline 
\end{tabular}
\end{table}

%Auswerten der ergebnisse in der Tabelle

Im folgenden sind noch die Gau�-fits mit den bestimmten Parametern zu sehen.

%Plots mit fits einf�gen und Parameter in die Caption schreiben

\subsubsection{Verh�ltnisse}
Aus den bestimmten Peaks sollen die Verh�ltnisse der Cu-Linien und der Ordnungen unter einander.

\begin{table}[H]
\caption{In der Tabelle sind die Verh�ltnisse der einzelnen Cu-Linie und die der Ordnungen zu einander aufgetragen.}
\label{tab:verhaeltnisse}
\centering
\begin{tabular}{|c|c|c|c|c|}
\hline Cu-Linie & Ordnung & Counts & \% der K$_{\alpha_1}$ & \% der 1. Ordnung \\ 
\hline K$_{\alpha_1}$ & 1 &  &  &  \\ 
\hline K$_{\alpha_2}$ & 1 &  &  &  \\ 
\hline K$_\beta$ & 1 &  &  &  \\ 
\hline K$_{\alpha_1}$ & 3 &  &  &  \\ 
\hline K$_{\alpha_2}$ & 3 &  &  &  \\ 
\hline K$_\beta$ & 3 &  &  &  \\ 
\hline K$_\beta$ & 4 &  &  &  \\ 
\hline 
\end{tabular} 
\end{table}

%auswertung der Verhaeltnisse

\subsubsection{Abschw�chung durch den Ni-Filter}
Es soll die Abschw�chung der K$_\beta$-Linie und das Signal-zu-Rausch Verh�ltniss der K$_{\alpha_{1,2}}$-Linie und deren Energie und Energiebreite bestimmt werden.
Mit Verwendung des Filters ergibt sich das Diffrakogramm in Abbildung ??.

%Diffraktogramme mit Fits einf�gen

Es ergibt sich eine Countrate von ?? mit Ni-Filter und ein Count von ?? ohne Ni-Filter. Daraus ergibt sich ein Verh�ltnis von ??.
Das Signal-zu-Rauschverh�ltnis (SNR) wir in [dB] berechnet:

\begin{align}
SNR = 10 \cdot lg \left( \frac{I_{max}}{I_{rausch}} \right)
\end{align}

\subsubsection{Netzebenabstand Si(331)}
Der Si(111)-Einkristall wird nun gegen einen Si(331)-Einkristall ausgetauscht.
Mit den zuvor bestimmten Energien der Cu-Linien soll nach Gleichung ?? der Netzebenabstand bestimmt werden, der Fehler wird mit Gleichung ?? berechnet.
Die Peaks wurden wie zuvor mit der Gau�verteilung gefittet.

%Plot der Peaks mit Fit 

Aus den Messdaten ergeben sich die Werte in Tabelle ??

\begin{table}[H]
\label{tab:si(331)}
\caption{In der Tabelle sind die Ergebnisse zur Bestimmung des Netzebenabstandes von Si(331)}
\centering
\begin{tabular}{|c|c|c|}
\hline Werte & K$_{\alpha_1}$ & K$_{\alpha_2}$ \\ 
\hline 2$\theta [^\circ]$ &  &  \\ 
\hline $\Delta 2\theta [^\circ]$ &  &  \\ 
\hline E[eV] &  &  \\ 
\hline $\Delta$E[eV] &  &  \\ 
\hline d$[\SI{}{\angstrom}]$ &  &  \\ 
\hline $\Delta$d$[\SI{}{\angstrom}]$ &  &  \\ 
\hline 
\end{tabular} 
\end{table}

%Vergleich mit den Literaturwerten


\subsubsection{Netzebenabstand Ge(111)}
Es soll ein Ge(111)-Einkristall untersucht werden, dabei wird wie zuvor vorgegangen.

Aus den Messdaten ergeben sich die Werte in Tabelle ??

\begin{table}[H]
\label{tab:ge(111)}
\caption{In der Tabelle sind die Ergebnisse zur Bestimmung des Netzebenabstandes von Ge(111)}
\centering
\begin{tabular}{|c|c|c|}
\hline Werte & K$_{\alpha_1}$ & K$_{\alpha_2}$ \\ 
\hline 2$\theta [^\circ]$ &  &  \\ 
\hline $\Delta 2\theta [^\circ]$ &  &  \\ 
\hline E[eV] &  &  \\ 
\hline $\Delta$E[eV] &  &  \\ 
\hline d$[\SI{}{\angstrom}]$ &  &  \\ 
\hline $\Delta$d$[\SI{}{\angstrom}]$ &  &  \\ 
\hline 
\end{tabular} 
\end{table}


%Vergleich mit den Literaturwerten



