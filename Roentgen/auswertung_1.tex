%Auswertung des Emissionsspektrums
\subsection{Auswertung}
Es soll das Röntgenspektrum der Kupferanode bei drei unterschiedlichen Beschleunigungsspannungen und ein mit einem Ni-Filter, bei möglichst hoher Röntgenspannung untersucht werden.
Zum beugen der Röntgenstrahlen wird ein Si(111)-Einkristall verwendet, der Netzebenabstand liegt bei \SI{3,1356}{\angstrom}, entnommen von \cite{si_a}.
Gescannt wird ein Winkelbereich von ??$^\circ$ bis ??$^\circ$, dabei wurden für die Beschleunigungsspannung Werte von
\begin{itemize}
\item U = ?? und A = ??
\item U = ?? und A = ??
\item U = ?? und A = ??
\end{itemize}
 verwendet. Dabei ergeben sich die folgenden Plots.
 
 % Plots der Messunge für die Kupferanode mit unterschiedlichen Beschleunigungsspannungen
 
 
 Es ist deutlich zu erkennen, das bei steigender Beschleunigungsspannung und Strom die Counts größer werden und so die Peaks deutlich von dem Untergrund zu unterscheiden sind.