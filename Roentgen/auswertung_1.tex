%Auswertung des Emissionsspektrums
\subsection{Auswertung}
Es soll das R\"ontgenspektrum der Kupferanode bei drei unterschiedlichen Beschleunigungsspannungen und mit einem Ni-Filter, bei m\"oglichst hoher R\"ontgenspannung untersucht werden.
Zum beugen der R\"ontgenstrahlen wird ein Si(111)-Einkristall verwendet, der Netzebenabstand liegt bei \SI{3,1356}{\angstrom}, entnommen von \cite{si_a}.
Gescannt wird ein Winkelbereich von ??$^\circ$ bis ??$^\circ$, dabei wurden f\"ur die Beschleunigungsspannung Werte von
\begin{itemize}
\item U = ?? und A = ??
\item U = ?? und A = ??
\item U = ?? und A = ??
\end{itemize}
 verwendet. Dabei ergeben sich die folgenden Plots.
 
 % Plots der Messunge f\"ur die Kupferanode mit unterschiedlichen Beschleunigungsspannungen
 
 
Es ist deutlich zu erkennen, das bei steigender Beschleunigungsspannung und Strom die Counts gr\"o\ss er werden und so die Peaks deutlich von dem Untergrund zu unterscheiden sind.
Um Informationen \"uber den Winkel und die Intensit\"at zu erhalten, werden die Peaks mit der Gau\ss verteilung gefittet.
Die Energie wird nach Gleichung ?? bestimmt, die Counts ergeben sich mit Gleichung ??. Der Fehler wird mit Gleichung ?? bestimmt.

Es ergeben sich die folgenden Ergebnisse:

\begin{table}[H]
\caption{In der Tabelle sind die Messdaten und die daraus bestimmten Energien f\"ur die unterschiedlichen Cu-Linien und Ordnungen}
\label{tab:ergebnisse_1}
\centering
\begin{tabular}{|c|c|c|c|c|c|c|c|c|}
\hline Cu-Linie & Ordung & 2$\theta^\circ$ & $\Delta$2$\theta^\circ$ & Counts & $\Delta$Counts & E$_{exp}$[eV] & $\Delta$E$_{exp}$[eV] & E$_{literatur}$ \\ 
\hline K$_\beta$ & 1 &  &  &  &  &  &  &  \\ 
\hline K$_{\alpha_1}$ & 1 &  &  &  &  &  &  &  \\ 
\hline K$_{\alpha_2}$ & 1 &  &  &  &  &  &  &  \\ 
\hline K$_\beta$ & 3 &  &  &  &  &  &  &  \\ 
\hline K$_{\alpha_1}$ & 3 &  &  &  &  &  &  &  \\ 
\hline K$_{\alpha_2}$ & 3 &  &  &  &  &  &  &  \\ 
\hline K$_\beta$ & 4 &  &  &  &  &  &  &  \\ 
\hline 
\end{tabular}
\end{table}

%Auswerten der ergebnisse in der Tabelle

Im folgenden sind noch die Gau\ss -fits mit den bestimmten Parametern zu sehen.

%Plots mit fits einfügen und Parameter in die Caption schreiben

\subsubsection{Verh\"altnisse}
Aus den bestimmten Peaks sollen die Verh\"altnisse der Cu-Linien und der Ordnungen unter einander.

\begin{table}[H]
\caption{In der Tabelle sind die Verh\"altnisse der einzelnen Cu-Linie und die der Ordnungen zu einander aufgetragen.}
\label{tab:verhaeltnisse}
\centering
\begin{tabular}{|c|c|c|c|c|}
\hline Cu-Linie & Ordnung & Counts & \% der K$_{\alpha_1}$ & \% der 1. Ordnung \\ 
\hline K$_{\alpha_1}$ & 1 &  &  &  \\ 
\hline K$_{\alpha_2}$ & 1 &  &  &  \\ 
\hline K$_\beta$ & 1 &  &  &  \\ 
\hline K$_{\alpha_1}$ & 3 &  &  &  \\ 
\hline K$_{\alpha_2}$ & 3 &  &  &  \\ 
\hline K$_\beta$ & 3 &  &  &  \\ 
\hline K$_\beta$ & 4 &  &  &  \\ 
\hline 
\end{tabular} 
\end{table}

%auswertung der Verhaeltnisse

\subsubsection{Abschwächung durch den Ni-Filter}
Es soll die Abschwächung der K$_\beta$-Linie und das Signal-zu-Rausch Verhältniss der K$_{\alpha_{1,2}}$-Linie und deren Energie und Energiebreite bestimmt werden.
Mit Verwendung des Filters ergibt sich das Diffrakogramm in Abbildung ??.

%Diffraktogramme mit Fits einfügen

Es ergibt sich eine Count von ?? mit Ni-Filter und ein Count von ?? ohne Ni-Filter. Daraus ergibt sich ein Verhältnis von ??.
Das Signal-zu-Rauschverhältnis (SNR) wir in [dB] berechnet:

\begin{align}
SNR = 10 \cdot lg \left( \frac{I_{max}}{I_{rausch}} \right)
\end{align}




