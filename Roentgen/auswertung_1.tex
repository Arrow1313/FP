%Auswertung des Emissionsspektrums
\subsection{Auswertung}
Es soll das R\"ontgenspektrum der Kupferanode bei drei unterschiedlichen Beschleunigungsspannungen und mit einem Ni-Filter, bei m\"oglichst hoher R\"ontgenspannung untersucht werden.
Zum beugen der R\"ontgenstrahlen wird ein Si(111)-Einkristall verwendet, der Netzebenabstand liegt bei \SI{3,1356}{\angstrom}, entnommen von \cite{si_a}.
Gescannt wird ein Winkelbereich von ??$^\circ$ bis ??$^\circ$, dabei wurden f\"ur die Beschleunigungsspannung Werte von
\begin{itemize}
\item U = ?? und A = ??
\item U = ?? und A = ??
\item U = ?? und A = ??
\end{itemize}
 verwendet. Dabei ergeben sich die folgenden Plots.
 
 % Plots der Messunge f\"ur die Kupferanode mit unterschiedlichen Beschleunigungsspannungen
 
 
Es ist deutlich zu erkennen, das bei steigender Beschleunigungsspannung und Strom die Counts gr\"o\ss er werden und so die Peaks deutlich von dem Untergrund zu unterscheiden sind.
Um Informationen \"uber den Winkel und die Intensit\"at zu erhalten, werden die Peaks mit der Gau\ss verteilung gefittet.
Die Energie wird nach Gleichung ?? bestimmt, die Counts ergeben sich mit Gleichung ??. Der Fehler wird mit Gleichung ?? bestimmt.
