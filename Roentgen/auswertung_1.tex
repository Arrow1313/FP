%Auswertung des Emissionsspektrums
\subsection{Auswertung}
Es soll das R�ntgenspektrum der Kupferanode bei drei unterschiedlichen Beschleunigungsspannungen und mit einem Ni-Filter, bei m�glichst hoher R�ntgenspannung untersucht werden.
Zum beugen der R�ntgenstrahlen wird ein Si(111)-Einkristall verwendet, der Netzebenabstand liegt bei \SI{3,1356}{\angstrom}, entnommen von \cite{si_a}.
Gescannt wird ein Winkelbereich von 15$^\circ$ bis 130$^\circ$, dabei wurden f�r die Beschleunigungsspannung Werte von
\begin{itemize}
\item U = 30kV und A = 10mA
\item U = 40kV und A = 10mA
\item U = 40kV und A = 30mA
\end{itemize}
 verwendet. Dabei ergeben sich die folgenden Plots.
 
 % Plots der Messunge f\"ur die Kupferanode mit unterschiedlichen Beschleunigungsspannungen
In Abb. \ref{fig:spektrum_1} ist das Diffraktogramm f�r U = 30kV und A = 10mA.
 
\begin{figure}[H]
	\centering
  \includegraphics[scale=0.30]{spektrum_1.pdf}
	\caption{Diffraktogramm bei 30kV Beschleunigungsspannung und einem Anodenstrom von 10mA}
	\label{fig:spektrum_1}
\end{figure}

In Abb. \ref{fig:spektrum_2} ist das Diffraktogramm f�r U = 40kV und A = 10mA.
 
\begin{figure}[H]
	\centering
  \includegraphics[scale=0.30]{spektrum_2.pdf}
	\caption{Diffraktogramm bei 40kV Beschleunigungsspannung und einem Anodenstrom von 10mA}
	\label{fig:spektrum_2}
\end{figure}

In Abb. \ref{fig:spektrum_3} ist das Diffraktogramm f�r U = 40kV und A = 30mA.

\begin{figure}[H]
	\centering
  \includegraphics[scale=0.30]{spektrum_3.pdf}
	\caption{Diffraktogramm bei 40kV Beschleunigungsspannung und einem Anodenstrom von 30mA}
	\label{fig:spektrum_3}
\end{figure}
 

Es ist deutlich zu erkennen, das bei steigender Beschleunigungsspannung und Strom die Anzahl der Counts gr��er werden und so die Peaks deutlich von dem Untergrund zu unterscheiden sind.
Um Informationen �ber den Winkel und die Intensit�t zu erhalten, werden die Peaks mit der Voigtverteilung gefittet. Aus den Fitparametern kann die Intensit�t der einzelnen Peaks bestimmt und verglichen werden.

Der Fit f�r U=30kV und A=10mA ist in Abb. \ref{fig:30_10_fit} zu sehen, die Fitparameter ergaben sich die Werte in Tabelle \ref{tab:30_10_fit}.

\begin{figure}[H]
	\centering
  \includegraphics[scale=0.2]{30_10_voigt.pdf}
	\caption{Diffraktogramm bei 40kV Beschleunigungsspannung und einem Anodenstrom von 30mA}
	\label{fig:30_10_fit}
\end{figure}


\begin{table}[H]
\centering
\caption{Fitparameter f�r eine Beschleunigungsspannung von 30kV und einem Anodenstrom von 10mV}
\label{tab:30_10_fit}
\begin{tabular}{|c|c|c|c|c|c|c|}
\hline Peak & Paramter & Center & Amplitude & Sigma & Gamma & $\chi_{red}^2$ \\ 
\hline 1 & Wert & 25,6390 $\pm$ 0,0001 & 260 $\pm$ 5 & 0,0124 $\pm$ 0,0007 & 0,008 $\pm$ 0,001 & 1,08 \\ 
\hline 2 & Wert & 28,4278 $\pm$ 0,0007  & 790 $\pm$ 96 & 0,0187 $\pm$ 0,003 & -0,003 $\pm$ 0,006 & 16,11 \\ 
\hline 3 & Wert & 28,4996 $\pm$ 0,0007 & 425 $\pm$ 128 & 0,021 $\pm$ 0,006 & -0,01 $\pm$ 0,01 & 16,11 \\ 
\hline 
\end{tabular} 
\end{table}

Der Fit f�r U=40kV und A=10mA ist in Abb. \ref{fig:40_10_fit} zu sehen, die Fitparameter ergaben sich die Werte in Tabelle \ref{tab:40_10_fit}.

\begin{figure}[H]
	\centering
  \includegraphics[scale=0.2]{40_10_voigt.pdf}
	\caption{Diffraktogramm bei 40kV Beschleunigungsspannung und einem Anodenstrom von 10mA}
	\label{fig:40_10_fit}
\end{figure}


\begin{table}[H]
\centering
\caption{Fitparameter f�r eine Beschleunigungsspannung von 40kV und einem Anodenstrom von 10mV}
\label{tab:40_10_fit}
\begin{tabular}{|c|c|c|c|c|c|c|}
\hline Peak & Paramter & Center & Amplitude & Sigma & Gamma & $\chi_{red}^2$ \\ 
\hline 1 & Wert & 25,6400 $\pm$ 0,0001 & 425 $\pm$ 6 & 0,0121 $\pm$ 0,0007 & 0,0089 $\pm$ 0,0007 & 4,11 \\ 
\hline 2 & Wert & 28,4282 $\pm$ 0,0003  & 1149 $\pm$ 52 & 0,018 $\pm$ 0,001 & 0,0007 $\pm$ 0,002 & 14,46 \\ 
\hline 3 & Wert & 28,5008 $\pm$ 0,0003 & 621 $\pm$ 75 & 0,020 $\pm$ 0,003 & -0,006 $\pm$ 0,006 & 14,46 \\ 
\hline 
\end{tabular} 
\end{table}


Der Fit f�r U=40kV und A=30mA ist in Abb. \ref{fig:40_30_fit} zu sehen, die Fitparameter ergaben sich die Werte in Tabelle \ref{tab:40_30_fit}.

\begin{figure}[H]
	\centering
  \includegraphics[scale=0.2]{40_30_voigt.pdf}
	\caption{Diffraktogramm bei 40kV Beschleunigungsspannung und einem Anodenstrom von 30mA}
	\label{fig:40_30_fit}
\end{figure}


\begin{table}[H]
\centering
\caption{Fitparameter f�r eine Beschleunigungsspannung von 40kV und einem Anodenstrom von 30mV}
\label{tab:40_30_fit}
\begin{tabular}{|c|c|c|c|c|c|c|}
\hline Peak & Paramter & Center & Amplitude & Sigma & Gamma & $\chi_{red}^2$ \\ 
\hline 1 & Wert & 25.64521 $\pm$ 0,00001 & 498 $\pm$ 158 & 0,031 $\pm$ 0,005 & -0,03 $\pm$ 0,02 & 1,13 \\ 
\hline 2 & Wert & 28,433 $\pm$ 0,001  & 1046 $\pm$ 873 & 0,01 $\pm$ 0,04 & -0,05 $\pm$ 0,05 & 71,49 \\ 
\hline 3 & Wert & 28,507 $\pm$ 0,001 & 1814 $\pm$ 613 & 0,022 $\pm$ 0,005 & -0,004 $\pm$ 0,012 & 71,49 \\ 
\hline 
\end{tabular} 
\end{table}




\subsubsection{Verh�ltnisse}
Aus den bestimmten Peaks sollen die Verh�ltnisse der Cu-Linien und der Ordnungen unter einander. F�r den Voigt-Fit die eingebaute Funktion des Pythonpackages lmfit (cite) verwendet. Die Voigtverteilung wird �ber ein normiertes Intergral bestimmt, worduch der Fit Parameter f�r die Amplitude f�r die Verh�ltnisbestimmung verwendet werden.

\begin{table}[H]
\caption{In der Tabelle sind die Verh�ltnisse der einzelnen Cu-Linie und die der Ordnungen zu einander aufgetragen.}
\label{tab:verhaeltnisse}
\centering
\begin{tabular}{|c|c|c|c|c|}
\hline Cu-Linie & Ordnung & Counts & \% der K$_{\alpha_1}$ & \% der 1. Ordnung \\ 
\hline K$_{\alpha_1}$ & 1 &  &  &  \\ 
\hline K$_{\alpha_2}$ & 1 &  &  &  \\ 
\hline K$_\beta$ & 1 &  &  &  \\ 
\hline K$_{\alpha_1}$ & 3 &  &  &  \\ 
\hline K$_{\alpha_2}$ & 3 &  &  &  \\ 
\hline K$_\beta$ & 3 &  &  &  \\ 
\hline K$_\beta$ & 4 &  &  &  \\ 
\hline 
\end{tabular} 
\end{table}

%auswertung der Verhaeltnisse

\subsubsection{Abschw�chung durch den Ni-Filter}
Es soll die Abschw�chung der K$_\beta$-Linie und das Signal-zu-Rausch Verh�ltniss der K$_{\alpha_{1,2}}$-Linie und deren Energie und Energiebreite bestimmt werden.
Mit Verwendung des Filters ergibt sich das Diffrakogramm in Abbildung ??.

%Diffraktogramme mit Fits einf�gen

Es ergibt sich eine Countrate von ?? mit Ni-Filter und ein Count von ?? ohne Ni-Filter. Daraus ergibt sich ein Verh�ltnis von ??.
Das Signal-zu-Rauschverh�ltnis (SNR) wir in [dB] berechnet:

\begin{align}
SNR = 10 \cdot lg \left( \frac{I_{max}}{I_{rausch}} \right)
\end{align}

\subsubsection{Netzebenabstand Si(331)}
Der Si(111)-Einkristall wird nun gegen einen Si(331)-Einkristall ausgetauscht.
Mit den zuvor bestimmten Energien der Cu-Linien soll nach Gleichung ?? der Netzebenabstand bestimmt werden, der Fehler wird mit Gleichung ?? berechnet.
Die Peaks wurden wie zuvor mit der Gau�verteilung gefittet.

%Plot der Peaks mit Fit 

Aus den Messdaten ergeben sich die Werte in Tabelle ??

\begin{table}[H]
\label{tab:si(331)}
\caption{In der Tabelle sind die Ergebnisse zur Bestimmung des Netzebenabstandes von Si(331)}
\centering
\begin{tabular}{|c|c|c|}
\hline Werte & K$_{\alpha_1}$ & K$_{\alpha_2}$ \\ 
\hline 2$\theta [^\circ]$ &  &  \\ 
\hline $\Delta 2\theta [^\circ]$ &  &  \\ 
\hline E[eV] &  &  \\ 
\hline $\Delta$E[eV] &  &  \\ 
\hline d$[\SI{}{\angstrom}]$ &  &  \\ 
\hline $\Delta$d$[\SI{}{\angstrom}]$ &  &  \\ 
\hline 
\end{tabular} 
\end{table}

%Vergleich mit den Literaturwerten


\subsubsection{Netzebenabstand Ge(111)}
Es soll ein Ge(111)-Einkristall untersucht werden, dabei wird wie zuvor vorgegangen.

Aus den Messdaten ergeben sich die Werte in Tabelle ??

\begin{table}[H]
\label{tab:ge(111)}
\caption{In der Tabelle sind die Ergebnisse zur Bestimmung des Netzebenabstandes von Ge(111)}
\centering
\begin{tabular}{|c|c|c|}
\hline Werte & K$_{\alpha_1}$ & K$_{\alpha_2}$ \\ 
\hline 2$\theta [^\circ]$ &  &  \\ 
\hline $\Delta 2\theta [^\circ]$ &  &  \\ 
\hline E[eV] &  &  \\ 
\hline $\Delta$E[eV] &  &  \\ 
\hline d$[\SI{}{\angstrom}]$ &  &  \\ 
\hline $\Delta$d$[\SI{}{\angstrom}]$ &  &  \\ 
\hline 
\end{tabular} 
\end{table}


%Vergleich mit den Literaturwerten



