\subsection{Versuchsdurchf�hrung}
%erkl�ren, !was! wir machen, !warum! wir das machen und mit welchem ziel
%(wichtig) pr�zize erkl�ren, wie bei dem versuch vorgegangen und was gemacht wurde
Nach dem Braggschen Verfahren soll ein SI(111)-Einkristall untersucht werden. Genauer werden die R�ntgenspektren f�r mindestens drei verschiedene Beschleunigungsspannungen aufgenommen. Zus�tzlich wird ein Spektrum mit eingesetztem Ni-Filter bei einer hohen R�hrenspannung aufgenommen.
Bei jeder Messung wird die Intensit�t als Funktion des Winkels bestimmt und die Lage aller Ordnungen der Braggreflexe der K$_{\alpha_{1,2}}$- und K$_\beta$-Linien von Kupfer und deren Intensit�tsverh�ltnisse zueinander. In der Messung des Emissionsspektrums mit Ni-Filter wird zus�tzlich die Abschw�chung der K$_\beta$-Linie und das "`Signal zu Rausch"' Verh�ltnis f�r die K$_{\alpha_{1,2}}$ Linien, deren resultierende Energie und Energiebreite bestimmt.
Zuletzt sollen die Netzebenenabst�nde weiterer Einkristalle bestimmt werden, indem der SI(111)-Einkristall durch diese ersetzt wird und die Emissionsspektren aufgenommen werden. Die Resultate werden dann mit Literaturwerten f�r die Netzebenenabst�nde verglichen.
\subsection{Verwendete Formeln}
%eine legende kann angefertigt werden, die selbstverst�ndlichen buchstaben m�ssen nicht extra erkl�rt werden
%mit knappen erkl�rungen die !verwendeten! formeln, sowie die zugeh�rige fehlerrechnung einf�gen.
Um das Emissionsspektrum der Kupferanode nach dem Braggschen Verfahren an einem SI(111)-Einkristall zu bestimmen, muss ein Literaturwert f�r den Netzebenenabstand vorausgesetzt werden. Der Netzebenenabstand betr�gt (...)