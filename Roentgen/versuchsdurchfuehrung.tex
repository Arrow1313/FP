\subsection{Versuchsdurchf�hrung}
%erkl�ren, !was! wir machen, !warum! wir das machen und mit welchem ziel
%(wichtig) pr�zize erkl�ren, wie bei dem versuch vorgegangen und was gemacht wurde
Nach dem Braggschen Verfahren soll ein SI(111)-Einkristall untersucht werden. Genauer werden die R�ntgenspektren f�r mindestens drei verschiedene Beschleunigungsspannungen aufgenommen. Zus�tzlich wird ein Spektrum mit eingesetztem Ni-Filter bei einer hohen R�hrenspannung aufgenommen.
Bei jeder Messung wird die Intensit�t als Funktion des Winkels bestimmt und die Lage aller Ordnungen der Braggreflexe der K$_{\alpha_{1,2}}$- und K$_\beta$-Linien von Kupfer und deren Intensit�tsverh�ltnisse zueinander. Aufgrund des Strukturfaktors f�r das Diamantgitter (Silitium und Germanium), wird der Braggreflex zweiter Ordnung unterdr�ckt. In der Messung des Emissionsspektrums mit Ni-Filter wird zus�tzlich die Abschw�chung der K$_\beta$-Linie und das "`Signal zu Rausch"' Verh�ltnis f�r die K$_{\alpha_{1,2}}$ Linien, deren resultierende Energie und Energiebreite bestimmt.
Zuletzt sollen die Netzebenenabst�nde weiterer Einkristalle bestimmt werden, indem der Si(111)-Einkristall durch diese ersetzt wird und die Emissionsspektren aufgenommen werden. Die Resultate werden dann mit Literaturwerten f�r die Netzebenenabst�nde verglichen.
\subsubsection{Emissionsspektrum der Kupferanode}
%eine legende kann angefertigt werden, die selbstverst�ndlichen buchstaben m�ssen nicht extra erkl�rt werden
%mit knappen erkl�rungen die !verwendeten! formeln, sowie die zugeh�rige fehlerrechnung einf�gen.
Um das Emissionsspektrum der Kupferanode nach dem Braggschen Verfahren an einem Si(111)-Einkristall zu bestimmen, muss ein Literaturwert f�r den Netzebenenabstand vorausgesetzt werden. Der Netzebenenabstand betr�gt \SI{3,1356}{\angstrom} (vgl. \cite{si_a}). Damit k�nnen die Energien der charakteristischen Strahlung von Kupfer nach der Braggschen Gleichung und dem Zusammenhang zwischen Energie und Wellenl�nge aus den Winkeln maximaler Reflexion bestimmt werden:
\begin{align}
E = \frac{hc}{\lambda} = \frac{hc}{2d_{[nh,nk,nl]}sin{\Theta}}
\end{align}
d soll hierbei aufgrung des vernachl�ssigbaren Fehlers der Literaturangabe als fehlerlos angenommen werden.
Der Fehler f�r die Energie berechnet sich also aus dem Fehler f�r den Winkel bei maximaler Reflexion:
\begin{align}
\Delta E =  \lvert\frac{\partial E}{\partial \Theta}\Delta \Theta\rvert =  \lvert\frac{E}{\tan\Theta}\Delta \Theta\rvert
\end{align}
Die zugeh�rigen Counts k�nnen dann am Maximum der angefitteten Gausskurven abgelesen werden:
\begin{align}
I = \frac{a}{\sqrt{2\pi \sigma^2}}
\end{align}
Die Parameter $a$ und $\sigma$ sowie deren Fehler werden aus dem Fit berechnet.
Der Fehler von $I$ ist dann:
\begin{align}
\Delta I = \sqrt{\left(\frac{\Delta a}{\sqrt{2\pi \sigma^2}}\right)^2+\left(\frac{a\Delta \sigma}{\sqrt{2\pi \sigma^4}}\right)^2}
\end{align}
\subsubsection{Bestimmung der Netzebenenabst�nde von Si(331) und Ge(111)}
Um die Netzebenenabst�nde der Einkristalle Si(331) und Ge(111) zu bestimmen, werden diese in das Diffraktometer eingesetzt. Mit der Braggschen Gleichung werden nach der Aufnahme der Emissionsspektren die Netzebenenabst�nde bestimmt:
\begin{align}
d = n d_{[nh,nk,nl]} = \frac{n \lambda}{2 \sin{\Theta}} = \frac{n hc}{ 2 E \sin{\Theta}}
\end{align}
F�r den Fehler ergibt sich also:
\begin{align}
\Delta d = \sqrt{\left(\frac{\partial d}{\partial \Theta}\Delta \Theta\right)^2+ \left(\frac{\partial d}{\partial E}\Delta E\right)^2} = d \sqrt{\left(\frac{\Delta \Theta}{\tan{\Theta}}\right)^2+ \left(\frac{\Delta E}{E}\right)^2}
\end{align}