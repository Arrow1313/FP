\subsection{Monochromatoren und Detektoren}
Um die Funktionsweise von Detektoren und Monochromatoren zu verstehen, wird in diesem Abschnitt das Wichtigste besprochen.
Die entscheidenden Eigenschaften von Detektoren sind:
\begin{enumerate}
\item Effizienz\\
Die Effizienz des Detektors soll also hoch im Bereich der zu untersuchenden Wellenl�ngen liegen und idealerweise ungew�nschte Frequenzen filtern (Ineffizienz bei ungew�nschten Wellenl�ngen).
\item Linearit�t\\
Bestenfalls ist das Detektorsignal direkt Proportional zur Lichtintensit�t
\item Energieproportionalit�t\\
Idealerweise direkte Proportionalit�t des Detektorsignals zur Energie des Einfallenden Lichtquants
\item Aufl�sung\\
Photonen verschiedener Wellenl�nge/Energie sollten falls m�glich ein unterscheidbares Detektorsignal liefern. (Praktisch nicht realisierbar bei kontinuierlichem Spektrum)
\end{enumerate}
Der Siliciumdriftdetektor (SDD) beispielsweise basiert auf einer pn-Diode bzw. einer Photodiode, welche in Sperrichtung geschaltet ist, was eine Verbreiterung der Raumladungszone bewirkt. Einfallende R�ntgenstrahlung wird dort absorbiert und erzeugt Elektronen-Loch-Paare,  welche aufgrund der hohen Spannung voneinader getrennt werden, sodass sie in der Raumladungszone nicht rekombinieren k�nnen. Dieses Konzept wurde lange Zeit weiterentwickelt, sodass heutige Siliciumdriftdetektoren komplizierter aufgebaut sind, um z.B. die Effizienz deutlich zu steigern. Dies ist auch das Problem der Silitiumdriftdetektoren, denn sie besitzen oberhalb von \SI{10}{KeV} eine geringere Effizienz, sodass oberhalb dieser Grenze eine h�here Strahlungsintensit�t ben�tigt wird.