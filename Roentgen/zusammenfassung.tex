\documentclass[12pt,a4paper]{article}

\usepackage[onehalfspacing]{setspace}
\usepackage{natbib}
 
\usepackage{float}
%für feststellen der figures und tables [H] dranschreiben
\usepackage{units}
%wird so benutzt: 
%\unit[value/Zahl]{dimension/Einheit} oder 
%\unitfrac[value/Zahl]{dimension/Einheit num/Zähler}{dimension/Einheit denum/Nenner} oder
%\nicefrac[fontcommand/Schriftart]{dimension/Einheit num/Zähler}{dimension/Einheit denum/Nenner}
\usepackage{amssymb}
\usepackage{caption}
\usepackage{subcaption}

\usepackage{hyperref}

\usepackage[left=2cm,right=2cm,top=2cm,bottom=2cm]{geometry}
\usepackage[latin9]{inputenc}
\usepackage[ngerman]{babel}
\usepackage[T1]{fontenc}
\usepackage{lmodern}
\usepackage{amsmath}
\usepackage{graphicx}
\usepackage{textcomp}


\begin{document}

\tableofcontents

\section{Theorie}

\subsection{Erzeugen von R�ntgenstrahlung}
\label{subsec:erzeugen von r�ntgenstrahlung}
\begin{itemize}
\item R�ntgenr�hren
\begin{itemize}
\item Beschleunigen von Elektronen aus einem Gl�hdraht
\item Bremsen an der Kathode, wodurch R�ntgenstrahlung emittiert wird
\end{itemize}
\item Synchrotornstrahlung
\begin{itemize}
\item Undolator zum verlassen des Kreises
\item Monochromator
\item Focusierung
\item Spectrometer
\end{itemize}
\end{itemize}

\subsection{Charakteristische R�ntgenstrahlung}
\begin{itemize}
\item Atom spezifische R�ntgenstrahlung, die von den Schalen der Elektronen abh�ngen
\end{itemize}

\subsection{Bremsstrahlung}
\begin{itemize}
\item Die Strahlung entsteht durch abbremsen geladener Teilchen, siehe \ref{subsec:erzeugen von r�ntgenstrahlung}
\end{itemize}

\begin{align}
E_{photo} = h \cdot f = E_{kin} = e \cdot U
\end{align}

\subsection{Schalenmodell der Elektronenh�lle eines Atoms}

\subsection{Mosley'sches Gesetz}
\begin{itemize}
\item Beschreibt den �bergang eines Elektrons von der L in die K Schale, bzw. die Energie der K$_\alpha$ �berg�nge
\end{itemize}
\begin{align}
f = \frac{\lambda}{c} = f_R \cdot Z_{eff}^2 \left( \frac{1}{n_1^2} - \frac{1}{n_2^2} \right)
\end{align}
\begin{itemize}
\item $f_R$:	angepasste Rydberg-Frequenz
\item $Z_{eff}$:	effektive Kernladung
\item $n_1, n_2$:	Hauptquantenzahl der Zust�nde
\end{itemize}

\subsection{effektive Kernladungszahl}
\begin{align}
Z_{eff} = Z - S
\end{align}
\begin{itemize}
\item Z:	Kernladungszahl
\item S:	Abschirmungskonstante (heuristisch)
\end{itemize}
\subsection{Aufbau von Kristallen}
\begin{itemize}
\item bcc, fcc, sc
\end{itemize}
\subsection{Beugung von R�ntgenstrahlen an einem Kristallgitter}

\subsection{Bragg'sche Refelexionsbedingung}

\begin{align}
2 \cdot d \cdot sin(\theta) = n \cdot \lambda
\end{align}

\begin{itemize}
\item d:	Ebenenabstand
\item $\theta$:	Einfallswinkel (nicht zu Lot)
\item $\lambda$:	Wellenl�nge des R�ntgenstrahls
\item n:	Interferenz Bedingung 
\end{itemize}

\subsection{Begriff der Netzebene}

\subsection{Gitterkonstante und Netzebenenabstand eines Kristalls}

\subsection{Analyse von Pulvergemischen}

\subsection{Funktionsweise eines Monochromators}
\begin{itemize}
\item Verwendung der Bragg-Reflexion eines Kristalls, wodurch man monochromatische Strahlung erh�lt
\end{itemize}

\subsection{Arten von R�ntgendetektoren/Funktionsweise eines Detektors}
\begin{itemize}
\item Totzeit erkl�ren
\end{itemize}
\vspace{1cm}
\begin{itemize}
\item Halbleiterdetektoren
	\begin{itemize}
	\item Vorteile:
		\begin{itemize}
		\item hohe Energieaufl�sung
		\end{itemize}
	\item Nachteile:
		\begin{itemize}
		\item schwaches Signal
		\item k�hlung notwendig
		\end{itemize}
	\end{itemize}
\item Szintillator:
	\begin{itemize}
	\item Vorteile:
		\begin{itemize}
		\item hohe Quantenausbeute
		\item kurze Totzeit
		\item gute Linearit�t
		\end{itemize}
	\item Nachteile:
		\begin{itemize}
		\item schlechte Energieaufl�sung
		\end{itemize}
	\end{itemize}
\item R�ntgenfilm
	\begin{itemize}
	\item Vorteile:
		\begin{itemize}
		\item gro�fl�chig
		\item hohe Ortsaufl�sung
		\item Langzeitspeicher
		\end{itemize}
	\item Nachteile:
		\begin{itemize}
		\item hohe Strahlungsdosis
		\item nicht-lineare Schw�rzung
		\item nicht energieempfindlich
		\end{itemize}
	\end{itemize}
\end{itemize}

\subsection{Detektor-Totzeiten}
\begin{itemize}
\item Wenn ein Teilchen detektiert wurde, kann f�r kurze Zeit kein weiteres Teichen detektiert werde, diesen Zeitraum nennt man Totzeit
\end{itemize}
\subsection{Fluoreszenz}
\begin{itemize}
\item Spontane Licht Emission kurz nach der Anregung des Materials
\item Das emittierte Licht hat in der Regel eine geringere Energie als das zuvor emittierte
\item Spin ist erhalten
\end{itemize}


\section{Auswertung}

\subsection{Messung des Emissionsspektrums}
\subsubsection{Messung des Emissionsspektrums}
\begin{itemize}
\item Untersuchung von R�ntgenbeugung an einem Silizium(111)-Einkristall
\item Aufnahme des Emissionsspektrums der Kupferanode mittels Bragg'schen Verfahren
\begin{itemize}
\item Z�hlrate �ber Winkel auftragen (numpy dataten einlesen, matplotlib f�rs plotten)
\item Lage aller $K_{\alpha_{1,2}} \text{ und } K_\beta$ Linien von Cu und deren Intensit�tsverh�ltnisse (f�r jede Ordung und alle anderen Ordnungen)
\item Signal-zu-Rausch Verh�ltnis f�r $K_{\alpha_{1,2}}$ und $K_\beta$ Linien, SRV = $\frac{P_{Signal}}{P_{Rauschen}}$
\end{itemize}
\end{itemize}
\subsubsection{Netzebenenabst�nde anderer Kristalle}
\begin{itemize}
\item Bestimmen der Netzebenenabst�nde von Si(331) und Ge(111)
\item Vergleich mit Literaturwerten
\end{itemize}

\subsection{Pulverdeffraktometrie/Debye-Scherrer-Verfahren}
\begin{itemize}
\item Analyse einer unbekannten Pulverprobe
\item Daten aus dem Deffraktogramm mit Datenbank Abgleichen, zur Bestimmung der Pulverzusammensetzung
\item Netzebenenabst�nde aus dem Deffraktogramm bestimmen
\item Graphisch zeigen, das die Bestimmte Kristallstrucktur mit dem Diffraktogramm vertr�glich ist
\item Ermitteln der mittleren Kristallgr��e
\end{itemize}





\end{document}