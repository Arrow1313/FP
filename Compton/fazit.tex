\section{Fazit}
%im fazit nochmal alles zusammenfassen und den verlauf der messung absch�tzen
%gravierende sytematische probleme bei den messungen nochmal betonen und die wertigkeit unserer ergebnisse einordnen
Insgesamt kann die Messung nicht als erfolgreich angesehen werden, da weder mit der Messung des differentiellen Wirkungsquerschnittes, noch mit der Bestimmung der Elektronenmasse Ergebnisse erzielt werden konnten, die nah an den Theorie-/Literatur-Werten liegen. Der Wirkungsquerschnitt konnte nicht bis auf eine Genauigkeit von \SI{10}{\percent} bestimmt werden, da systematische Fehler, deren Ursprung nicht sicher begr�ndet werden k�nnen eine .ystematische Abweichung nach oben ergeben. Trotzdem konnte eindeutig die Proportionalit�t der gemessenen Wirkungsquerschnitte zum theoretischen Klein-Nishina Wirkungsquerschnitt gezeigt werden (vgl. Abb. \ref{fig:korr_WQ_all}).