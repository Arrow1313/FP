\section{Versuchsdurchf�hrung und Auswertung}
%zuerst !alle! errechneten werte entweder in ganzen s�tzen aufz�hlen, oder in tabellen (�bersichtlicher) dargestellen, sowie auf die verwendeten formeln verweisen (die referenzierung der formel kann in der �berschrift stehen)
%kurz erw�hnen (vor der tabelle), warum wir das ganze ausrechnen bzw. was wir dort ausrechnen
%danach histogramme und plots erstellen, wobei wenn m�glich funktionen durch die plots gelegt werden (zur not k�nnen auch splines benutzt werden, was aber angegeben werden muss)
%bei fits immer die funktion und das reduzierte chiquadrat mit angegeben, wobei auf verst�ndlichkeit beim entziffern der zehnerpotenzen geachtet werden muss z.b. f(x)=(wert+-fehler)\cdot10^{irgendeine zahl}\cdot x + (wert+-fehler)\cdot10^{irgendeine zahl}
%bei jedem fit erkl�ren, nach welchem zusammenhang gefittet wurde und warum!
%bei plots darauf achten, dass die achsenbeschriftung (auch die tics) die richtige gr��e haben und die legende im plot nicht die messwerte verdeckt
%kurz die aufgabenstellung abgehandeln

Der Versuch l�sst sich in 4 Teile aufteilen. Bevor Messungen gemacht werden k�nnen muss der Versuchsaufbau kalibriert werden. Im zweitem Teil wird die Absorption von $\gamma$-Strahlung durch Blei und Aluminium untersucht. Danach wird der Compton-Wirkungsquerschnitt gemessen und untersucht. Im letztem Teil wird die Elektronenmasse bestimmt.

\subsection{Kalibration}
Vor der Kalibrierung wird die Hochspannung des Photomulitpliers auf 600V gestellt, da sich nur f�r diese Spannung ein linearer Zusammenhang zwischen den Kanalnummern und der Energie der $\gamma$-Strahlen einstellt. Um den Arbeitspunk einzustellen, werden verschieden Eichpr�parate auf H�he der Kollimator�ffnung platziert. Die verwendeten Pr�parate und deren Eigenschaften sind in Tabelle ?? zu sehen \cite{??}. Die Verst�rkung wird so gew�hlt, dass der interessante Energiebereich abgedeckt wird. Um den Untergrund von den Messungen subtrahieren zu k�nnen, wird zuerst eine Leermessung gemacht. Danach wird mit den Elementen aus Tabelle ?? die Kalibrierung f�r den Zusammenhang zwischen Kanalnummer und Energie gemessen. Die Daten werden mit Gl. ?? gefittet.

\begin{align}
\label{equ:lin_fit}
f(x) = m \cdot x + b
\end{align}

\begin{table}
\centering
\caption{Eichpr�parate f�r die Kalibrierung}
\label{tab:kallibrierung}

\end{table}

\subsection{Absorbtion von $\gamma$-Strahlen durch Blei und Aluminium}
Es soll der Absorptionskoeffizient der 662keV Energie mit der Hilfe verschieden dicker Blei- und Aluminiumplatten bestimmt werden. Daf�r wird eine Cs-Quelle eingebaut und der Detektor auf die 0$^\circ$-Position gebracht. Es ist zu beachten, dass sich immer eine Bleiplatte von mindestens 2,5cm Dicke zwischen dem Detektor und der Quelle befindet, da sonst f�r mehrere Stunden nur Rausche zu messen ist.

\subsection{Compton-Wirkungsquerscnitt}
In diesem Versuchsabschnitt soll der differentielle Wirkungsquerschnitt an der Aluminiumplatte untersucht werden. Daf�r wird ein Winkelbereich von 20$^\circ$ bis 80$^\circ$ und 5$^\circ$ Schritten abgefahren, dabei befindet sich keine 2,5cm dicke Aluminiumplatte vor dem Detektor. Um den Untergrund aus den Messdaten raus rechnen zu k�nnen, wird f�r jede Winkelpostion einmal mit und einmal ohne Aluminiumplatte gemessen.

\subsection{Elektronenmasse}
Mit den zuvor bestimmten Daten l�sst sich nun �ber Gl. ?? die Elektromasse berechnen.