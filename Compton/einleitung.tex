\section{Einleitung}
%einleitung zu dem experiment.
%auf die einstellungen, die vor dem versuch gemacht werden, eingehen, oder auf eine anleitung dazu verweisen.
%---------------------------------------------------------------------------------------------
%hinter der einleitung kann der allgemeine theoretische hintergrund in einer zus�tzlichen section erkl�rt werden
Die Compton-Streuung behandelt den elastischen Sto� eines Photons mit einem freien (oder quasifreien bzw. schwach gebundenen) Elektron. Die Energie/Wellenl�nge des gestreuten Photons und die Richtung der Streuung folgen dabei ausschlie�lich aus der Kinematik. Aus der Wellenl�ngen�nderung kann die Compton-Wellenl�nge/Masse des Elektrons bestimmt werden. Der theoretische differentielle Wirkungsquerschnitt ist eine der ersten Folgerungen der Quantenelektrodynamik und wurde zuerst 1929 von Oskar Klein und Yoshio Nishina bestimmt (vgl. \cite{wikiWQkleinnishina}). Zur Detektion der gestreuten Photonen wird in diesem Versuch ein Detektor bestehend aus einem NaI-Kristall und nachgeschalteten Photomultipliern verwendet (vgl. \cite{Versuchsanleitung}). Zur Bestimmung des Wirkungsquerschnittes wird an einer etwa \SI{1}{cm} dicken Aluminiumplatte gestreut.