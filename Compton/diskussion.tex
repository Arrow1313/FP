\subsection{Diskussion}
%(immer) die gemessenen werte und die bestimmten werte �ber die messfehler mit literaturwerten oder untereinander vergleichen
%in welchem fehlerintervall des messwertes liegt der literaturwert oder der vergleichswert?
%wie ist der relative anteil des fehlers am messwert und damit die qualit�t unserer messung?
%in einem satz erkl�ren, wie gut unser fehler und damit unsere messung ist
%kurz erl�utern, wie systematische fehler unsere messung beeinflusst haben k�nnten
%(wichtig) zum schluss ansprechen, in wie weit die ergebnisse mit der theoretischen vorhersage �bereinstimmen
%--------------------------------------------------------------------------------------------
%falls tabellen mit den messwerten zu lang werden, kann die section mit den messwerten auch hinter der diskussion angef�gt bzw. eine section mit dem anhang eingef�gt werden.
Zu Beginn des Versuches musste die Energie-Kanal-Kalibrierung f�r den MCA durchgef�hrt werden. Die reduzierten Chiquadrate der Gaussfits an die gemessenen Peaks liegen im Bereich von 1, wobei das kleinste $\chi^2_{red}$ bei 0,80 und das gr��te $\chi^2_{red}$ bei 2,31 liegt. Insgesamt konnten auf diese Weise 8 Peaks angefittet werden, welche in Tabelle \ref{tab:lin_fit} angegeben sind. Bei der Energie-Kanal-Kalibrierung ergab sich die lineare Funktion $E(K) = K \cdot 0.345(6) - 9(2)$. Mit einem $\chi^2_{red}$ von 1,88 ist dieser Fit akzeptabel, wobei die Fehler entweder etwas zu klein sind, oder systematische Fehler das $\chi^2_{red}$ beeinflussen. Systematische Fehler k�nnten zum Beispiel bei der Abz�hlung der Peaks im jeweiligen Spektrum aufgetreten sein. Nach der Kalibrierung wurden $\gamma$-Stahlenabsorbtionsmessungen f�r die Pr�parate Blei und Aluminium durchgef�hrt. Bei verschiedener Dicke der zwischen Strahl und Detektor stehenden Blei Pr�parate, wurden Z�hlmessungen f�r den Zeitraum von \SI{300(1)}{\second} durchgef�hrt. Die Counts in Abh�ngigkeit der Dicke wurden mit einer Exponentialfunktion $ae^{\lambda x}$ gefittet. Es ergaben sich die Konstanten in Tabelle \ref{tab:absorb_blei_fit} bei einem $\chi^2_{red}$ von 1,63. Dabei ist die Amplitude stark mit der Absorbtionskonstante $\lambda$ korreliert, sodass eine systematische Abweichung vom eigentlichen Wert nicht ausgeschlossen werden kann. Die Amplitude ist dabei entscheidend f�r die Messung des differentiellen Wirkungsquerschnittes. Aufgrund des zu gro� geratenen geometrischen Wirkungsquerschnittes, der in Abb. \ref{fig:korr_WQ_all} dargestellt ist, kann davon ausgegangen werden, dass die so bestimmte Amplitude systematisch zu klein ist (vgl. Gleichung \ref{eqn:Differentieller_Wirkungsquerschnitt}). Die \SI{2,5}{cm} dicke Bleiplatte wurde f�r die Messung der Counts in Abh�ngigkeit der Dicke f�r das Target Aluminium zum Schutz des Detektors vor den Kollimator gestellt, sodass die Counts bei einer Dicke der Aluminiumplatte von \SI{0}{cm} den Counts f�r eine Bleidicke von \SI{2,5}{cm} entsprechen. Wieder wurden die Counts in Abh�ngigkeit der Dicke mit einer Exponentialfunktion gefittet. Es ergaben sich die Konstanten in Tabelle \ref{tab:absorb_alu_fit} bei einem $\chi^2_{red}$ von 0,03, was darauf schlie�en l�sst, dass die Fehler zu gro� gew�hlt wurden. Insgesamt ergeben sich dadurch Absorbtionskoeffizienten, die nah an den Literaturwerten liegen. Die Literaturwerte liegen innerhalb der Fehlerbalken der bestimmten Werte.